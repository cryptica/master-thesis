\chapter{Experimental evaluation}
\label{chap:experimental_evaluation}

We implemented our algorithms in a tool called
\emph{Pe\-tri\-ni\-zer}. 
Petrinizer is implemented as a script on top of the 
% combination of Bash and Prolog scripts,
Z3 SMT solver~\cite{MouraB08}.
It takes as input coverability problem instances encoded in the MIST
input format\footnote{\url{https://github.com/pierreganty/mist}}, and it
runs one of the selected methods. 
We implemented all possible combinations of
methods: with and without trap refinement, with rational and integer arithmetic,
with and without invariant construction, with and without invariant
minimization.

Our evaluation had two main goals. First, as the underlying
methods are incomplete, we wanted to measure their success rate on standard
benchmark sets. 
As a subgoal, we wanted to investigate
the usefulness and necessity of traps, the benefit of using integer arithmetic
over rational arithmetic, and the sizes of the constructed invariants.
The second goal was to measure Petrinizer's performance and to compare it
with state-of-the-art tools: 
IIC~\cite{KloosMNP13}, BFC\footnote{The most recent version of BFC at the time of
writing the paper was 2.0. However, we noticed it sometimes reports
inconsistent results, so we used version 1.0 instead. The tool can be
obtained at \url{http://www.cprover.org/bfc/}.}~\cite{KaiserKW12}, and
MIST\footnote{MIST consists of several methods, most of them based on
  EEC~\cite{GeeraertsRB06}. We used the abstraction
  refinement method that tries to minimize the number of places in the
  Petri net~\cite{GantyRB08}.}.

\paragraph{Benchmarks.} For the inputs used in the experiments, we collected 
coverability problem instances originating from various sources. The
collection contains 178 examples, out of which 115 are safe, and is
organized into five example suites. The first suite is a collection of
Petri net examples from the MIST toolkit. This suite contains a
mixture of 29 examples, both safe and unsafe. It contains both real-world and
artificially created examples. The second suite consists of 46 Petri nets
that were used in the evaluation of BFC~\cite{KaiserKW12}. They
originate from the analysis of concurrent C programs, and they are mostly
unsafe. The third and the fourth suites come from the
provenance analysis of messages in a medical system and a
bug-tracking system~\cite{MajumdarMW13}. The medical suite contains 12 safe
examples, and the bug-tracking suite contains 41 examples, all
safe except for one. The fifth suite contains 50 examples that
come from the analysis of Erlang programs~\cite{DOsualdoKO13}. We generated
them ourselves using an Erlang verification tool called Soter~\cite{DOsualdoKO13},
from the example programs found on Soter's
website\footnote{\url{http://mjolnir.cs.ox.ac.uk/soter/}}. Out of 50 examples in
this suite, 38 are safe. This suite also contains the largest example in the
collection, with 66,950 places and 213,635 transitions.
For our evaluation, only the 115 safe instances are interesting.

\begin{table}[t]
  \centering
  \caption[Safe examples that were successfully proved safe.] 
    {Safe examples that were successfully proved safe. Symbols
     $\mathbb{Q}$ and $\mathbb{Z}$ denote rational and integer numbers.}
\label{table:rate-of-success}
  \scriptsize
  \begin{tabular}{lrrrrrrrr}
    \toprule
    \multicolumn{1}{c}{Suite} &
    \multicolumn{1}{c}{Safety/$\mathbb{Q}$} &
    \multicolumn{1}{c}{Safety/$\mathbb{Z}$} &
    \multicolumn{1}{c}{Ref./$\mathbb{Q}$} &
    \multicolumn{1}{c}{Ref./$\mathbb{Z}$} &
    \multicolumn{1}{c}{IIC} &
    \multicolumn{1}{c}{BFC} &
    \multicolumn{1}{c}{MIST} &
    \multicolumn{1}{c}{Total} \\
    \midrule
    MIST        & 14 & 14& 20 & 20 & {\textbf 23} & 21 & 19 & 23 \\
    BFC          & {\textbf 2} & {\textbf 2} & {\textbf 2} & {\textbf 2} & {\textbf 2} & {\textbf 2} & {\textbf 2} & 2 \\
    Medical      & 4 & 4 & 4 & 4 & 9 & {\textbf 12} & 10 & 12 \\
    Bug-tracking & {\textbf 32} & {\textbf 32} & {\textbf 32} & {\textbf 32} & 0 & 0 & 0 & 40 \\
    Erlang       & 32 & 32 & 36 & {\textbf 38} & 17 & 26 & 2 & 38 \\
    \midrule
    Total        & 84 & 84 & 94 & {\textbf 96} & 51 & 61 & 33 & 115 \\
    \bottomrule
  \end{tabular}
\end{table}

\paragraph{Rate of success on safe examples.}
As shown in Table~\ref{table:rate-of-success}, even with the weakest of the
methods ---safety based on marking equation over rationals--- Petrinizer is
able to prove safety for 84 out
of 115 examples. Switching to integer arithmetic does not help: the number of
examples proved safe remains 84. 
Using refinement via traps, Petrinizer proves
safety for 94 examples. Switching to integer arithmetic in this case helps:
Another two examples are proved safe, totaling 96 out of 115 examples. 
In contrast to these numbers, the most successful existing tool turned out to be BFC, proving safety for
only 61 examples. 
% The small numbers are due to time and memory limits. 
Even though the methods these tools implement
are theoretically complete, the tools themselves are limited by the time and space they can
use.

Looking at the results accross different suites, we see that Petrinizer
performed poorest on the medical suite, proving safety for only 4 out of 12
examples. On the other hand, on the bug-tracking suite, which was completely
intractable for other tools, it proved safety for 32 out of 40 examples.
Furthermore, using traps and integer arithmetic, Petrinizer successfuly proved
safety for all safe Erlang examples. We find this result particularly
surprising, as the original verification problems for these examples seem non-trivial.

\begin{figure}[t]
  \centering
  \begin{tikzpicture}[gnuplot]
%% generated with GNUPLOT 4.6p3 (Lua 5.1; terminal rev. 99, script rev. 100)
%% Wed May 14 15:12:07 2014
\path (0.000,0.000) rectangle (12.500,7.000);
\gpcolor{color=gp lt color border}
\gpsetlinetype{gp lt border}
\gpsetlinewidth{1.00}
\draw[gp path] (1.424,4.485)--(1.604,4.485);
\node[gp node right] at (1.240,4.485) {{\supertiny $10^{0}$}};
\draw[gp path] (1.424,4.914)--(1.604,4.914);
\node[gp node right] at (1.240,4.914) {{\supertiny $10^{1}$}};
\draw[gp path] (1.424,5.343)--(1.604,5.343);
\node[gp node right] at (1.240,5.343) {{\supertiny $10^{2}$}};
\draw[gp path] (1.424,5.773)--(1.604,5.773);
\node[gp node right] at (1.240,5.773) {{\supertiny $10^{3}$}};
\draw[gp path] (1.424,6.202)--(1.604,6.202);
\node[gp node right] at (1.240,6.202) {{\supertiny $10^{4}$}};
\draw[gp path] (1.424,6.631)--(1.604,6.631);
\node[gp node right] at (1.240,6.631) {{\supertiny $10^{5}$}};
\draw[gp path] (1.424,4.485)--(1.424,4.665);
\node[gp node center] at (1.424,4.177) {{\supertiny $10^{0}$}};
\draw[gp path] (1.853,4.485)--(1.853,4.665);
\node[gp node center] at (1.853,4.177) {{\supertiny $10^{1}$}};
\draw[gp path] (2.282,4.485)--(2.282,4.665);
\node[gp node center] at (2.282,4.177) {{\supertiny $10^{2}$}};
\draw[gp path] (2.712,4.485)--(2.712,4.665);
\node[gp node center] at (2.712,4.177) {{\supertiny $10^{3}$}};
\draw[gp path] (3.141,4.485)--(3.141,4.665);
\node[gp node center] at (3.141,4.177) {{\supertiny $10^{4}$}};
\draw[gp path] (3.570,4.485)--(3.570,4.665);
\node[gp node center] at (3.570,4.177) {{\supertiny $10^{5}$}};
\draw[gp path] (1.424,6.631)--(1.424,4.485)--(3.570,4.485)--(3.570,6.631)--cycle;
\node[gp node center,rotate=-270] at (0.350,5.558) {Size of invariant};
\node[gp node center] at (2.497,3.715) {Number of places};
\gpcolor{color=gp lt color 2}
\gpsetpointsize{2.00}
\gppoint{gp mark 7}{(1.887,4.914)}
\gppoint{gp mark 7}{(1.724,4.914)}
\gppoint{gp mark 7}{(2.362,5.420)}
\gppoint{gp mark 7}{(2.045,5.092)}
\gppoint{gp mark 7}{(2.455,5.515)}
\gppoint{gp mark 7}{(2.455,5.515)}
\gppoint{gp mark 7}{(2.164,5.218)}
\gppoint{gp mark 7}{(1.916,4.743)}
\gppoint{gp mark 7}{(2.000,4.614)}
\gppoint{gp mark 7}{(2.000,4.743)}
\gppoint{gp mark 7}{(2.070,4.743)}
\gppoint{gp mark 7}{(2.161,4.743)}
\gppoint{gp mark 7}{(1.963,4.914)}
\gppoint{gp mark 7}{(1.758,4.914)}
\gppoint{gp mark 7}{(1.941,4.743)}
\gppoint{gp mark 7}{(1.871,4.948)}
\gppoint{gp mark 7}{(1.941,4.977)}
\gppoint{gp mark 7}{(1.834,4.895)}
\gppoint{gp mark 7}{(1.916,5.024)}
\gppoint{gp mark 7}{(1.902,4.743)}
\gppoint{gp mark 7}{(2.494,4.614)}
\gppoint{gp mark 7}{(2.494,4.614)}
\gppoint{gp mark 7}{(2.494,4.614)}
\gppoint{gp mark 7}{(2.494,4.614)}
\gppoint{gp mark 7}{(2.659,5.002)}
\gppoint{gp mark 7}{(2.659,5.002)}
\gppoint{gp mark 7}{(2.659,5.002)}
\gppoint{gp mark 7}{(2.659,5.002)}
\gppoint{gp mark 7}{(2.659,4.614)}
\gppoint{gp mark 7}{(2.659,4.614)}
\gppoint{gp mark 7}{(2.659,4.614)}
\gppoint{gp mark 7}{(2.659,4.614)}
\gppoint{gp mark 7}{(2.659,4.614)}
\gppoint{gp mark 7}{(2.659,4.614)}
\gppoint{gp mark 7}{(2.659,4.614)}
\gppoint{gp mark 7}{(2.659,4.614)}
\gppoint{gp mark 7}{(2.659,4.614)}
\gppoint{gp mark 7}{(2.659,4.614)}
\gppoint{gp mark 7}{(2.659,4.614)}
\gppoint{gp mark 7}{(2.659,4.614)}
\gppoint{gp mark 7}{(2.659,4.614)}
\gppoint{gp mark 7}{(2.659,4.614)}
\gppoint{gp mark 7}{(2.659,4.614)}
\gppoint{gp mark 7}{(2.659,4.614)}
\gppoint{gp mark 7}{(2.659,4.614)}
\gppoint{gp mark 7}{(2.659,4.614)}
\gppoint{gp mark 7}{(2.659,4.614)}
\gppoint{gp mark 7}{(2.659,4.614)}
\gppoint{gp mark 7}{(2.659,4.614)}
\gppoint{gp mark 7}{(2.659,4.614)}
\gppoint{gp mark 7}{(2.659,4.614)}
\gppoint{gp mark 7}{(2.659,4.614)}
\gppoint{gp mark 7}{(2.659,4.614)}
\gppoint{gp mark 7}{(2.659,4.614)}
\gppoint{gp mark 7}{(2.659,4.614)}
\gppoint{gp mark 7}{(2.659,4.614)}
\gppoint{gp mark 7}{(2.601,5.137)}
\gppoint{gp mark 7}{(2.780,5.427)}
\gppoint{gp mark 7}{(3.003,5.734)}
\gppoint{gp mark 7}{(2.562,5.158)}
\gppoint{gp mark 7}{(2.562,5.158)}
\gppoint{gp mark 7}{(2.665,4.485)}
\gppoint{gp mark 7}{(2.971,4.485)}
\gppoint{gp mark 7}{(2.234,5.013)}
\gppoint{gp mark 7}{(2.259,5.061)}
\gppoint{gp mark 7}{(2.259,5.061)}
\gppoint{gp mark 7}{(2.422,4.914)}
\gppoint{gp mark 7}{(2.847,5.552)}
\gppoint{gp mark 7}{(2.772,5.483)}
\gppoint{gp mark 7}{(2.522,5.085)}
\gppoint{gp mark 7}{(2.614,5.099)}
\gppoint{gp mark 7}{(2.676,5.119)}
\gppoint{gp mark 7}{(2.653,5.085)}
\gppoint{gp mark 7}{(2.990,5.099)}
\gppoint{gp mark 7}{(3.144,5.113)}
\gppoint{gp mark 7}{(2.534,5.069)}
\gppoint{gp mark 7}{(3.066,5.313)}
\gppoint{gp mark 7}{(3.495,5.571)}
\gppoint{gp mark 7}{(2.471,5.495)}
\gppoint{gp mark 7}{(2.471,5.495)}
\gppoint{gp mark 7}{(2.471,5.491)}
\gppoint{gp mark 7}{(2.471,5.491)}
\gppoint{gp mark 7}{(2.471,5.002)}
\gppoint{gp mark 7}{(2.471,5.002)}
\gppoint{gp mark 7}{(2.462,5.153)}
\gppoint{gp mark 7}{(2.760,5.398)}
\gppoint{gp mark 7}{(2.875,5.494)}
\gppoint{gp mark 7}{(2.462,5.153)}
\gppoint{gp mark 7}{(2.462,5.153)}
\gppoint{gp mark 7}{(2.597,5.002)}
\gppoint{gp mark 7}{(2.597,4.990)}
\gppoint{gp mark 7}{(3.280,5.661)}
\gppoint{gp mark 7}{(2.424,5.085)}
\gppoint{gp mark 7}{(2.300,4.848)}
\gpcolor{color=gp lt color border}
\draw[gp path] (1.424,6.631)--(1.424,4.485)--(3.570,4.485)--(3.570,6.631)--cycle;
%% coordinates of the plot area
\gpdefrectangularnode{gp plot 1}{\pgfpoint{1.424cm}{4.485cm}}{\pgfpoint{3.570cm}{6.631cm}}
\draw[gp path] (5.730,4.533)--(5.910,4.533);
\node[gp node right] at (5.546,4.533) {{\supertiny $10^{-1}$}};
\draw[gp path] (5.730,4.875)--(5.910,4.875);
\node[gp node right] at (5.546,4.875) {{\supertiny $10^{0}$}};
\draw[gp path] (5.730,5.216)--(5.910,5.216);
\node[gp node right] at (5.546,5.216) {{\supertiny $10^{1}$}};
\draw[gp path] (5.730,5.558)--(5.910,5.558);
\node[gp node right] at (5.546,5.558) {{\supertiny $10^{2}$}};
\draw[gp path] (5.730,5.900)--(5.910,5.900);
\node[gp node right] at (5.546,5.900) {{\supertiny $10^{3}$}};
\draw[gp path] (5.730,6.241)--(5.910,6.241);
\node[gp node right] at (5.546,6.241) {{\supertiny $10^{4}$}};
\draw[gp path] (5.730,6.583)--(5.910,6.583);
\node[gp node right] at (5.546,6.583) {{\supertiny $10^{5}$}};
\draw[gp path] (5.730,4.533)--(5.730,4.713);
\node[gp node center] at (5.730,4.225) {{\supertiny $10^{-1}$}};
\draw[gp path] (6.072,4.533)--(6.072,4.713);
\node[gp node center] at (6.072,4.225) {{\supertiny $10^{0}$}};
\draw[gp path] (6.413,4.533)--(6.413,4.713);
\node[gp node center] at (6.413,4.225) {{\supertiny $10^{1}$}};
\draw[gp path] (6.755,4.533)--(6.755,4.713);
\node[gp node center] at (6.755,4.225) {{\supertiny $10^{2}$}};
\draw[gp path] (7.097,4.533)--(7.097,4.713);
\node[gp node center] at (7.097,4.225) {{\supertiny $10^{3}$}};
\draw[gp path] (7.438,4.533)--(7.438,4.713);
\node[gp node center] at (7.438,4.225) {{\supertiny $10^{4}$}};
\draw[gp path] (7.780,4.533)--(7.780,4.713);
\node[gp node center] at (7.780,4.225) {{\supertiny $10^{5}$}};
\draw[gp path] (5.730,6.583)--(5.730,4.533)--(7.780,4.533)--(7.780,6.583)--cycle;
\node[gp node center,rotate=-270] at (4.472,5.558) {Safety over integers};
\node[gp node center] at (6.755,3.763) {Safety over rationals};
\gpcolor{color=gp lt color 0}
\gpsetlinetype{gp lt plot 0}
\draw[gp path] (5.730,4.533)--(5.751,4.554)--(5.771,4.574)--(5.792,4.595)--(5.813,4.616)%
  --(5.834,4.637)--(5.854,4.657)--(5.875,4.678)--(5.896,4.699)--(5.916,4.719)--(5.937,4.740)%
  --(5.958,4.761)--(5.978,4.781)--(5.999,4.802)--(6.020,4.823)--(6.041,4.844)--(6.061,4.864)%
  --(6.082,4.885)--(6.103,4.906)--(6.123,4.926)--(6.144,4.947)--(6.165,4.968)--(6.186,4.989)%
  --(6.206,5.009)--(6.227,5.030)--(6.248,5.051)--(6.268,5.071)--(6.289,5.092)--(6.310,5.113)%
  --(6.331,5.134)--(6.351,5.154)--(6.372,5.175)--(6.393,5.196)--(6.413,5.216)--(6.434,5.237)%
  --(6.455,5.258)--(6.475,5.278)--(6.496,5.299)--(6.517,5.320)--(6.538,5.341)--(6.558,5.361)%
  --(6.579,5.382)--(6.600,5.403)--(6.620,5.423)--(6.641,5.444)--(6.662,5.465)--(6.683,5.486)%
  --(6.703,5.506)--(6.724,5.527)--(6.745,5.548)--(6.765,5.568)--(6.786,5.589)--(6.807,5.610)%
  --(6.827,5.630)--(6.848,5.651)--(6.869,5.672)--(6.890,5.693)--(6.910,5.713)--(6.931,5.734)%
  --(6.952,5.755)--(6.972,5.775)--(6.993,5.796)--(7.014,5.817)--(7.035,5.838)--(7.055,5.858)%
  --(7.076,5.879)--(7.097,5.900)--(7.117,5.920)--(7.138,5.941)--(7.159,5.962)--(7.179,5.982)%
  --(7.200,6.003)--(7.221,6.024)--(7.242,6.045)--(7.262,6.065)--(7.283,6.086)--(7.304,6.107)%
  --(7.324,6.127)--(7.345,6.148)--(7.366,6.169)--(7.387,6.190)--(7.407,6.210)--(7.428,6.231)%
  --(7.449,6.252)--(7.469,6.272)--(7.490,6.293)--(7.511,6.314)--(7.532,6.335)--(7.552,6.355)%
  --(7.573,6.376)--(7.594,6.397)--(7.614,6.417)--(7.635,6.438)--(7.656,6.459)--(7.676,6.479)%
  --(7.697,6.500)--(7.718,6.521)--(7.739,6.542)--(7.759,6.562);
\gpcolor{color=gp lt color 2}
\gppoint{gp mark 7}{(5.790,4.593)}
\gppoint{gp mark 7}{(5.780,4.583)}
\gppoint{gp mark 7}{(5.903,4.719)}
\gppoint{gp mark 7}{(5.817,4.620)}
\gppoint{gp mark 7}{(5.969,4.763)}
\gppoint{gp mark 7}{(6.440,5.325)}
\gppoint{gp mark 7}{(5.840,4.628)}
\gppoint{gp mark 7}{(5.790,4.593)}
\gppoint{gp mark 7}{(5.800,4.603)}
\gppoint{gp mark 7}{(5.809,4.593)}
\gppoint{gp mark 7}{(5.780,4.603)}
\gppoint{gp mark 7}{(5.790,4.593)}
\gppoint{gp mark 7}{(5.790,4.603)}
\gppoint{gp mark 7}{(5.790,4.593)}
\gppoint{gp mark 7}{(5.769,4.593)}
\gppoint{gp mark 7}{(5.809,4.603)}
\gppoint{gp mark 7}{(5.817,4.620)}
\gppoint{gp mark 7}{(5.800,4.593)}
\gppoint{gp mark 7}{(5.790,4.583)}
\gppoint{gp mark 7}{(5.790,4.583)}
\gppoint{gp mark 7}{(5.780,4.583)}
\gppoint{gp mark 7}{(5.790,4.612)}
\gppoint{gp mark 7}{(5.800,4.593)}
\gppoint{gp mark 7}{(5.790,4.593)}
\gppoint{gp mark 7}{(5.780,4.583)}
\gppoint{gp mark 7}{(5.790,4.593)}
\gppoint{gp mark 7}{(5.790,4.593)}
\gppoint{gp mark 7}{(5.790,4.603)}
\gppoint{gp mark 7}{(5.780,4.593)}
\gppoint{gp mark 7}{(6.272,5.075)}
\gppoint{gp mark 7}{(6.288,5.098)}
\gppoint{gp mark 7}{(6.286,5.117)}
\gppoint{gp mark 7}{(6.273,5.072)}
\gppoint{gp mark 7}{(6.277,5.115)}
\gppoint{gp mark 7}{(6.279,5.112)}
\gppoint{gp mark 7}{(6.280,5.070)}
\gppoint{gp mark 7}{(6.288,5.122)}
\gppoint{gp mark 7}{(6.291,5.130)}
\gppoint{gp mark 7}{(6.279,5.076)}
\gppoint{gp mark 7}{(6.285,5.124)}
\gppoint{gp mark 7}{(6.272,5.118)}
\gppoint{gp mark 7}{(6.526,5.329)}
\gppoint{gp mark 7}{(6.526,5.327)}
\gppoint{gp mark 7}{(6.525,5.328)}
\gppoint{gp mark 7}{(6.525,5.328)}
\gppoint{gp mark 7}{(6.530,5.444)}
\gppoint{gp mark 7}{(6.525,5.328)}
\gppoint{gp mark 7}{(6.525,5.325)}
\gppoint{gp mark 7}{(6.527,5.328)}
\gppoint{gp mark 7}{(6.529,5.328)}
\gppoint{gp mark 7}{(6.531,5.448)}
\gppoint{gp mark 7}{(6.530,5.325)}
\gppoint{gp mark 7}{(6.525,5.327)}
\gppoint{gp mark 7}{(6.527,5.326)}
\gppoint{gp mark 7}{(6.528,5.329)}
\gppoint{gp mark 7}{(6.531,5.444)}
\gppoint{gp mark 7}{(6.527,5.329)}
\gppoint{gp mark 7}{(6.527,5.329)}
\gppoint{gp mark 7}{(6.524,5.334)}
\gppoint{gp mark 7}{(6.524,5.327)}
\gppoint{gp mark 7}{(6.535,5.442)}
\gppoint{gp mark 7}{(6.526,5.328)}
\gppoint{gp mark 7}{(6.526,5.325)}
\gppoint{gp mark 7}{(6.523,5.327)}
\gppoint{gp mark 7}{(6.529,5.329)}
\gppoint{gp mark 7}{(6.529,5.445)}
\gppoint{gp mark 7}{(6.525,5.328)}
\gppoint{gp mark 7}{(6.525,5.327)}
\gppoint{gp mark 7}{(6.524,5.327)}
\gppoint{gp mark 7}{(6.525,5.326)}
\gppoint{gp mark 7}{(6.530,5.444)}
\gppoint{gp mark 7}{(6.528,5.327)}
\gppoint{gp mark 7}{(6.528,5.327)}
\gppoint{gp mark 7}{(6.535,5.441)}
\gppoint{gp mark 7}{(6.527,5.326)}
\gppoint{gp mark 7}{(6.527,5.328)}
\gppoint{gp mark 7}{(6.531,5.447)}
\gppoint{gp mark 7}{(6.527,5.327)}
\gppoint{gp mark 7}{(6.527,5.328)}
\gppoint{gp mark 7}{(6.524,5.327)}
\gppoint{gp mark 7}{(6.526,5.328)}
\gppoint{gp mark 7}{(6.534,5.448)}
\gppoint{gp mark 7}{(5.916,4.710)}
\gppoint{gp mark 7}{(6.054,4.852)}
\gppoint{gp mark 7}{(6.263,5.020)}
\gppoint{gp mark 7}{(5.883,4.675)}
\gppoint{gp mark 7}{(5.893,4.691)}
\gppoint{gp mark 7}{(5.888,4.701)}
\gppoint{gp mark 7}{(5.860,4.675)}
\gppoint{gp mark 7}{(5.924,4.731)}
\gppoint{gp mark 7}{(6.175,4.973)}
\gppoint{gp mark 7}{(5.833,4.643)}
\gppoint{gp mark 7}{(6.042,4.822)}
\gppoint{gp mark 7}{(6.216,5.029)}
\gppoint{gp mark 7}{(5.825,4.612)}
\gppoint{gp mark 7}{(5.825,4.628)}
\gppoint{gp mark 7}{(5.800,4.612)}
\gppoint{gp mark 7}{(5.833,4.612)}
\gppoint{gp mark 7}{(6.202,4.982)}
\gppoint{gp mark 7}{(6.061,4.864)}
\gppoint{gp mark 7}{(5.872,4.603)}
\gppoint{gp mark 7}{(5.888,4.719)}
\gppoint{gp mark 7}{(5.969,4.766)}
\gppoint{gp mark 7}{(5.907,4.701)}
\gppoint{gp mark 7}{(6.090,4.885)}
\gppoint{gp mark 7}{(6.191,4.994)}
\gppoint{gp mark 7}{(5.866,4.675)}
\gppoint{gp mark 7}{(6.510,5.307)}
\gppoint{gp mark 7}{(7.385,6.186)}
\gppoint{gp mark 7}{(5.883,4.686)}
\gppoint{gp mark 7}{(5.883,4.680)}
\gppoint{gp mark 7}{(5.877,4.636)}
\gppoint{gp mark 7}{(5.883,4.650)}
\gppoint{gp mark 7}{(5.877,4.675)}
\gppoint{gp mark 7}{(5.877,4.675)}
\gppoint{gp mark 7}{(5.893,4.675)}
\gppoint{gp mark 7}{(5.883,4.686)}
\gppoint{gp mark 7}{(5.840,4.628)}
\gppoint{gp mark 7}{(5.860,4.663)}
\gppoint{gp mark 7}{(6.062,4.843)}
\gppoint{gp mark 7}{(6.205,4.974)}
\gppoint{gp mark 7}{(5.860,4.650)}
\gppoint{gp mark 7}{(5.866,4.669)}
\gppoint{gp mark 7}{(5.872,4.686)}
\gppoint{gp mark 7}{(5.872,4.675)}
\gppoint{gp mark 7}{(6.458,5.256)}
\gppoint{gp mark 7}{(5.790,4.603)}
\gppoint{gp mark 7}{(5.817,4.620)}
\gppoint{gp mark 7}{(5.817,4.620)}
\gppoint{gp mark 7}{(5.809,4.612)}
\gppoint{gp mark 7}{(5.809,4.583)}
\gppoint{gp mark 7}{(5.790,4.603)}
\gppoint{gp mark 7}{(5.809,4.620)}
\gppoint{gp mark 7}{(6.283,5.100)}
\gppoint{gp mark 7}{(5.825,4.620)}
\gppoint{gp mark 7}{(6.031,4.840)}
\gppoint{gp mark 7}{(6.381,5.199)}
\gppoint{gp mark 7}{(5.854,4.663)}
\gppoint{gp mark 7}{(6.019,4.820)}
\gppoint{gp mark 7}{(5.825,4.620)}
\gppoint{gp mark 7}{(5.920,4.719)}
\gppoint{gp mark 7}{(5.790,4.593)}
\gppoint{gp mark 7}{(5.956,4.749)}
\gppoint{gp mark 7}{(5.860,4.650)}
\gppoint{gp mark 7}{(6.077,4.879)}
\gppoint{gp mark 7}{(5.883,4.686)}
\gppoint{gp mark 7}{(6.307,5.243)}
\gppoint{gp mark 7}{(6.302,5.238)}
\gppoint{gp mark 7}{(5.883,4.657)}
\gppoint{gp mark 7}{(6.126,4.931)}
\gppoint{gp mark 7}{(5.809,4.628)}
\gppoint{gp mark 7}{(6.005,4.820)}
\gppoint{gp mark 7}{(6.192,5.000)}
\gppoint{gp mark 7}{(5.800,4.593)}
\gppoint{gp mark 7}{(5.877,4.680)}
\gppoint{gp mark 7}{(6.102,4.908)}
\gppoint{gp mark 7}{(5.817,4.620)}
\gppoint{gp mark 7}{(6.132,4.920)}
\gppoint{gp mark 7}{(5.924,4.746)}
\gppoint{gp mark 7}{(5.939,4.746)}
\gppoint{gp mark 7}{(6.129,4.932)}
\gppoint{gp mark 7}{(6.127,4.924)}
\gppoint{gp mark 7}{(5.800,4.628)}
\gppoint{gp mark 7}{(5.866,4.669)}
\gppoint{gp mark 7}{(5.800,4.593)}
\gppoint{gp mark 7}{(5.916,4.719)}
\gppoint{gp mark 7}{(6.019,4.824)}
\gppoint{gp mark 7}{(5.825,4.620)}
\gppoint{gp mark 7}{(5.960,4.759)}
\gppoint{gp mark 7}{(5.809,4.612)}
\gppoint{gp mark 7}{(5.888,4.691)}
\gppoint{gp mark 7}{(5.825,4.628)}
\gppoint{gp mark 7}{(5.860,4.669)}
\gppoint{gp mark 7}{(5.928,4.739)}
\gppoint{gp mark 7}{(5.817,4.628)}
\gppoint{gp mark 7}{(5.986,4.796)}
\gppoint{gp mark 7}{(5.883,4.691)}
\gppoint{gp mark 7}{(6.296,5.244)}
\gpcolor{color=gp lt color border}
\gpsetlinetype{gp lt border}
\draw[gp path] (5.730,6.583)--(5.730,4.533)--(7.780,4.533)--(7.780,6.583)--cycle;
%% coordinates of the plot area
\gpdefrectangularnode{gp plot 2}{\pgfpoint{5.730cm}{4.533cm}}{\pgfpoint{7.780cm}{6.583cm}}
\draw[gp path] (9.897,4.533)--(10.077,4.533);
\node[gp node right] at (9.713,4.533) {{\supertiny $10^{-1}$}};
\draw[gp path] (9.897,4.875)--(10.077,4.875);
\node[gp node right] at (9.713,4.875) {{\supertiny $10^{0}$}};
\draw[gp path] (9.897,5.216)--(10.077,5.216);
\node[gp node right] at (9.713,5.216) {{\supertiny $10^{1}$}};
\draw[gp path] (9.897,5.558)--(10.077,5.558);
\node[gp node right] at (9.713,5.558) {{\supertiny $10^{2}$}};
\draw[gp path] (9.897,5.900)--(10.077,5.900);
\node[gp node right] at (9.713,5.900) {{\supertiny $10^{3}$}};
\draw[gp path] (9.897,6.241)--(10.077,6.241);
\node[gp node right] at (9.713,6.241) {{\supertiny $10^{4}$}};
\draw[gp path] (9.897,6.583)--(10.077,6.583);
\node[gp node right] at (9.713,6.583) {{\supertiny $10^{5}$}};
\draw[gp path] (9.897,4.533)--(9.897,4.713);
\node[gp node center] at (9.897,4.225) {{\supertiny $10^{-1}$}};
\draw[gp path] (10.239,4.533)--(10.239,4.713);
\node[gp node center] at (10.239,4.225) {{\supertiny $10^{0}$}};
\draw[gp path] (10.580,4.533)--(10.580,4.713);
\node[gp node center] at (10.580,4.225) {{\supertiny $10^{1}$}};
\draw[gp path] (10.922,4.533)--(10.922,4.713);
\node[gp node center] at (10.922,4.225) {{\supertiny $10^{2}$}};
\draw[gp path] (11.264,4.533)--(11.264,4.713);
\node[gp node center] at (11.264,4.225) {{\supertiny $10^{3}$}};
\draw[gp path] (11.605,4.533)--(11.605,4.713);
\node[gp node center] at (11.605,4.225) {{\supertiny $10^{4}$}};
\draw[gp path] (11.947,4.533)--(11.947,4.713);
\node[gp node center] at (11.947,4.225) {{\supertiny $10^{5}$}};
\draw[gp path] (9.897,6.583)--(9.897,4.533)--(11.947,4.533)--(11.947,6.583)--cycle;
\node[gp node center,rotate=-270] at (8.639,5.558) {Ref.\ over integers};
\node[gp node center] at (10.922,3.763) {Safety over integers};
\gpcolor{color=gp lt color 0}
\gpsetlinetype{gp lt plot 0}
\draw[gp path] (9.897,4.533)--(9.918,4.554)--(9.938,4.574)--(9.959,4.595)--(9.980,4.616)%
  --(10.001,4.637)--(10.021,4.657)--(10.042,4.678)--(10.063,4.699)--(10.083,4.719)--(10.104,4.740)%
  --(10.125,4.761)--(10.145,4.781)--(10.166,4.802)--(10.187,4.823)--(10.208,4.844)--(10.228,4.864)%
  --(10.249,4.885)--(10.270,4.906)--(10.290,4.926)--(10.311,4.947)--(10.332,4.968)--(10.353,4.989)%
  --(10.373,5.009)--(10.394,5.030)--(10.415,5.051)--(10.435,5.071)--(10.456,5.092)--(10.477,5.113)%
  --(10.498,5.134)--(10.518,5.154)--(10.539,5.175)--(10.560,5.196)--(10.580,5.216)--(10.601,5.237)%
  --(10.622,5.258)--(10.642,5.278)--(10.663,5.299)--(10.684,5.320)--(10.705,5.341)--(10.725,5.361)%
  --(10.746,5.382)--(10.767,5.403)--(10.787,5.423)--(10.808,5.444)--(10.829,5.465)--(10.850,5.486)%
  --(10.870,5.506)--(10.891,5.527)--(10.912,5.548)--(10.932,5.568)--(10.953,5.589)--(10.974,5.610)%
  --(10.994,5.630)--(11.015,5.651)--(11.036,5.672)--(11.057,5.693)--(11.077,5.713)--(11.098,5.734)%
  --(11.119,5.755)--(11.139,5.775)--(11.160,5.796)--(11.181,5.817)--(11.202,5.838)--(11.222,5.858)%
  --(11.243,5.879)--(11.264,5.900)--(11.284,5.920)--(11.305,5.941)--(11.326,5.962)--(11.346,5.982)%
  --(11.367,6.003)--(11.388,6.024)--(11.409,6.045)--(11.429,6.065)--(11.450,6.086)--(11.471,6.107)%
  --(11.491,6.127)--(11.512,6.148)--(11.533,6.169)--(11.554,6.190)--(11.574,6.210)--(11.595,6.231)%
  --(11.616,6.252)--(11.636,6.272)--(11.657,6.293)--(11.678,6.314)--(11.699,6.335)--(11.719,6.355)%
  --(11.740,6.376)--(11.761,6.397)--(11.781,6.417)--(11.802,6.438)--(11.823,6.459)--(11.843,6.479)%
  --(11.864,6.500)--(11.885,6.521)--(11.906,6.542)--(11.926,6.562);
\gpcolor{color=gp lt color 2}
\gppoint{gp mark 7}{(9.957,4.742)}
\gppoint{gp mark 7}{(9.947,4.749)}
\gppoint{gp mark 7}{(9.947,4.742)}
\gppoint{gp mark 7}{(9.947,4.753)}
\gppoint{gp mark 7}{(9.957,4.749)}
\gppoint{gp mark 7}{(9.967,4.822)}
\gppoint{gp mark 7}{(10.808,5.558)}
\gppoint{gp mark 7}{(10.811,5.552)}
\gppoint{gp mark 7}{(10.812,5.550)}
\gppoint{gp mark 7}{(10.039,4.840)}
\gppoint{gp mark 7}{(10.039,4.838)}
\gppoint{gp mark 7}{(10.007,4.824)}
\gppoint{gp mark 7}{(10.186,4.999)}
\gppoint{gp mark 7}{(10.393,5.374)}
\gppoint{gp mark 7}{(10.050,4.902)}
\gppoint{gp mark 7}{(10.044,4.894)}
\gppoint{gp mark 7}{(10.000,4.945)}
\gppoint{gp mark 7}{(10.014,4.943)}
\gppoint{gp mark 7}{(10.039,5.092)}
\gppoint{gp mark 7}{(10.050,5.087)}
\gppoint{gp mark 7}{(9.992,4.811)}
\gppoint{gp mark 7}{(9.967,4.794)}
\gppoint{gp mark 7}{(9.984,4.791)}
\gppoint{gp mark 7}{(9.984,4.794)}
\gppoint{gp mark 7}{(9.976,4.783)}
\gppoint{gp mark 7}{(9.947,4.778)}
\gppoint{gp mark 7}{(9.967,4.786)}
\gppoint{gp mark 7}{(9.984,4.804)}
\gppoint{gp mark 7}{(10.204,5.000)}
\gppoint{gp mark 7}{(10.050,4.920)}
\gppoint{gp mark 7}{(10.021,4.854)}
\gppoint{gp mark 7}{(10.055,4.852)}
\gpcolor{color=gp lt color border}
\gpsetlinetype{gp lt border}
\draw[gp path] (9.897,6.583)--(9.897,4.533)--(11.947,4.533)--(11.947,6.583)--cycle;
%% coordinates of the plot area
\gpdefrectangularnode{gp plot 3}{\pgfpoint{9.897cm}{4.533cm}}{\pgfpoint{11.947cm}{6.583cm}}
\draw[gp path] (1.423,0.985)--(1.603,0.985);
\node[gp node right] at (1.239,0.985) {{\supertiny $10^{0}$}};
\draw[gp path] (1.423,1.414)--(1.603,1.414);
\node[gp node right] at (1.239,1.414) {{\supertiny $10^{1}$}};
\draw[gp path] (1.423,1.844)--(1.603,1.844);
\node[gp node right] at (1.239,1.844) {{\supertiny $10^{2}$}};
\draw[gp path] (1.423,2.273)--(1.603,2.273);
\node[gp node right] at (1.239,2.273) {{\supertiny $10^{3}$}};
\draw[gp path] (1.423,2.703)--(1.603,2.703);
\node[gp node right] at (1.239,2.703) {{\supertiny $10^{4}$}};
\draw[gp path] (1.423,3.132)--(1.603,3.132);
\node[gp node right] at (1.239,3.132) {{\supertiny $10^{5}$}};
\draw[gp path] (1.423,0.985)--(1.423,1.165);
\node[gp node center] at (1.423,0.677) {{\supertiny $10^{0}$}};
\draw[gp path] (1.853,0.985)--(1.853,1.165);
\node[gp node center] at (1.853,0.677) {{\supertiny $10^{1}$}};
\draw[gp path] (2.282,0.985)--(2.282,1.165);
\node[gp node center] at (2.282,0.677) {{\supertiny $10^{2}$}};
\draw[gp path] (2.712,0.985)--(2.712,1.165);
\node[gp node center] at (2.712,0.677) {{\supertiny $10^{3}$}};
\draw[gp path] (3.141,0.985)--(3.141,1.165);
\node[gp node center] at (3.141,0.677) {{\supertiny $10^{4}$}};
\draw[gp path] (3.571,0.985)--(3.571,1.165);
\node[gp node center] at (3.571,0.677) {{\supertiny $10^{5}$}};
\draw[gp path] (1.423,3.132)--(1.423,0.985)--(3.571,0.985)--(3.571,3.132)--cycle;
\node[gp node center,rotate=-270] at (0.349,2.058) {IIC};
\node[gp node center] at (2.497,0.215) {Petrinizer};
\gpcolor{color=gp lt color 0}
\gpsetlinetype{gp lt plot 0}
\draw[gp path] (1.423,0.985)--(1.445,1.007)--(1.466,1.028)--(1.488,1.050)--(1.510,1.072)%
  --(1.531,1.093)--(1.553,1.115)--(1.575,1.137)--(1.597,1.158)--(1.618,1.180)--(1.640,1.202)%
  --(1.662,1.224)--(1.683,1.245)--(1.705,1.267)--(1.727,1.289)--(1.748,1.310)--(1.770,1.332)%
  --(1.792,1.354)--(1.814,1.375)--(1.835,1.397)--(1.857,1.419)--(1.879,1.440)--(1.900,1.462)%
  --(1.922,1.484)--(1.944,1.505)--(1.965,1.527)--(1.987,1.549)--(2.009,1.571)--(2.031,1.592)%
  --(2.052,1.614)--(2.074,1.636)--(2.096,1.657)--(2.117,1.679)--(2.139,1.701)--(2.161,1.722)%
  --(2.182,1.744)--(2.204,1.766)--(2.226,1.787)--(2.247,1.809)--(2.269,1.831)--(2.291,1.852)%
  --(2.313,1.874)--(2.334,1.896)--(2.356,1.918)--(2.378,1.939)--(2.399,1.961)--(2.421,1.983)%
  --(2.443,2.004)--(2.464,2.026)--(2.486,2.048)--(2.508,2.069)--(2.530,2.091)--(2.551,2.113)%
  --(2.573,2.134)--(2.595,2.156)--(2.616,2.178)--(2.638,2.199)--(2.660,2.221)--(2.681,2.243)%
  --(2.703,2.265)--(2.725,2.286)--(2.747,2.308)--(2.768,2.330)--(2.790,2.351)--(2.812,2.373)%
  --(2.833,2.395)--(2.855,2.416)--(2.877,2.438)--(2.898,2.460)--(2.920,2.481)--(2.942,2.503)%
  --(2.963,2.525)--(2.985,2.546)--(3.007,2.568)--(3.029,2.590)--(3.050,2.612)--(3.072,2.633)%
  --(3.094,2.655)--(3.115,2.677)--(3.137,2.698)--(3.159,2.720)--(3.180,2.742)--(3.202,2.763)%
  --(3.224,2.785)--(3.246,2.807)--(3.267,2.828)--(3.289,2.850)--(3.311,2.872)--(3.332,2.893)%
  --(3.354,2.915)--(3.376,2.937)--(3.397,2.959)--(3.419,2.980)--(3.441,3.002)--(3.463,3.024)%
  --(3.484,3.045)--(3.506,3.067)--(3.528,3.089)--(3.549,3.110);
\gpcolor{color=gp lt color 2}
\gppoint{gp mark 7}{(1.853,1.414)}
\gppoint{gp mark 7}{(1.853,1.414)}
\gppoint{gp mark 7}{(2.359,2.856)}
\gppoint{gp mark 7}{(2.031,2.200)}
\gppoint{gp mark 7}{(2.454,3.046)}
\gppoint{gp mark 7}{(2.454,3.046)}
\gppoint{gp mark 7}{(2.157,2.451)}
\gppoint{gp mark 7}{(1.682,1.722)}
\gppoint{gp mark 7}{(1.552,1.244)}
\gppoint{gp mark 7}{(1.682,1.448)}
\gppoint{gp mark 7}{(1.682,1.502)}
\gppoint{gp mark 7}{(1.682,1.502)}
\gppoint{gp mark 7}{(1.853,2.420)}
\gppoint{gp mark 7}{(1.853,1.585)}
\gppoint{gp mark 7}{(1.682,2.007)}
\gppoint{gp mark 7}{(1.887,1.797)}
\gppoint{gp mark 7}{(1.915,1.855)}
\gppoint{gp mark 7}{(1.833,1.805)}
\gppoint{gp mark 7}{(1.962,1.964)}
\gppoint{gp mark 7}{(1.682,1.954)}
\gppoint{gp mark 7}{(1.552,1.114)}
\gppoint{gp mark 7}{(1.552,1.114)}
\gppoint{gp mark 7}{(1.552,1.114)}
\gppoint{gp mark 7}{(1.552,1.114)}
\gppoint{gp mark 7}{(1.423,0.985)}
\gppoint{gp mark 7}{(1.423,0.985)}
\gppoint{gp mark 7}{(1.952,1.975)}
\gppoint{gp mark 7}{(2.000,2.031)}
\gppoint{gp mark 7}{(2.000,2.031)}
\gppoint{gp mark 7}{(1.853,2.050)}
\gppoint{gp mark 7}{(2.008,2.245)}
\gppoint{gp mark 7}{(2.430,2.191)}
\gppoint{gp mark 7}{(2.430,2.191)}
\gppoint{gp mark 7}{(1.940,2.393)}
\gppoint{gp mark 7}{(1.940,2.393)}
\gppoint{gp mark 7}{(2.092,2.436)}
\gppoint{gp mark 7}{(2.092,2.467)}
\gppoint{gp mark 7}{(1.940,2.086)}
\gppoint{gp mark 7}{(1.928,2.017)}
\gppoint{gp mark 7}{(2.024,1.585)}
\gppoint{gp mark 7}{(1.786,1.348)}
\gpcolor{color=gp lt color border}
\gpsetlinetype{gp lt border}
\draw[gp path] (1.423,3.132)--(1.423,0.985)--(3.571,0.985)--(3.571,3.132)--cycle;
%% coordinates of the plot area
\gpdefrectangularnode{gp plot 4}{\pgfpoint{1.423cm}{0.985cm}}{\pgfpoint{3.571cm}{3.132cm}}
\draw[gp path] (5.730,1.033)--(5.910,1.033);
\node[gp node right] at (5.546,1.033) {{\supertiny $10^{-1}$}};
\draw[gp path] (5.730,1.375)--(5.910,1.375);
\node[gp node right] at (5.546,1.375) {{\supertiny $10^{0}$}};
\draw[gp path] (5.730,1.717)--(5.910,1.717);
\node[gp node right] at (5.546,1.717) {{\supertiny $10^{1}$}};
\draw[gp path] (5.730,2.059)--(5.910,2.059);
\node[gp node right] at (5.546,2.059) {{\supertiny $10^{2}$}};
\draw[gp path] (5.730,2.400)--(5.910,2.400);
\node[gp node right] at (5.546,2.400) {{\supertiny $10^{3}$}};
\draw[gp path] (5.730,2.742)--(5.910,2.742);
\node[gp node right] at (5.546,2.742) {{\supertiny $10^{4}$}};
\draw[gp path] (5.730,3.084)--(5.910,3.084);
\node[gp node right] at (5.546,3.084) {{\supertiny $10^{5}$}};
\draw[gp path] (5.730,1.033)--(5.730,1.213);
\node[gp node center] at (5.730,0.725) {{\supertiny $10^{-1}$}};
\draw[gp path] (6.072,1.033)--(6.072,1.213);
\node[gp node center] at (6.072,0.725) {{\supertiny $10^{0}$}};
\draw[gp path] (6.413,1.033)--(6.413,1.213);
\node[gp node center] at (6.413,0.725) {{\supertiny $10^{1}$}};
\draw[gp path] (6.755,1.033)--(6.755,1.213);
\node[gp node center] at (6.755,0.725) {{\supertiny $10^{2}$}};
\draw[gp path] (7.097,1.033)--(7.097,1.213);
\node[gp node center] at (7.097,0.725) {{\supertiny $10^{3}$}};
\draw[gp path] (7.438,1.033)--(7.438,1.213);
\node[gp node center] at (7.438,0.725) {{\supertiny $10^{4}$}};
\draw[gp path] (7.780,1.033)--(7.780,1.213);
\node[gp node center] at (7.780,0.725) {{\supertiny $10^{5}$}};
\draw[gp path] (5.730,3.084)--(5.730,1.033)--(7.780,1.033)--(7.780,3.084)--cycle;
\node[gp node center,rotate=-270] at (4.472,2.058) {Ref.\ w/ invariants};
\node[gp node center] at (6.755,0.263) {Ref.\ w/o invariants};
\gpcolor{color=gp lt color 0}
\gpsetlinetype{gp lt plot 0}
\draw[gp path] (5.730,1.033)--(5.751,1.054)--(5.771,1.074)--(5.792,1.095)--(5.813,1.116)%
  --(5.834,1.137)--(5.854,1.157)--(5.875,1.178)--(5.896,1.199)--(5.916,1.219)--(5.937,1.240)%
  --(5.958,1.261)--(5.978,1.282)--(5.999,1.302)--(6.020,1.323)--(6.041,1.344)--(6.061,1.364)%
  --(6.082,1.385)--(6.103,1.406)--(6.123,1.427)--(6.144,1.447)--(6.165,1.468)--(6.186,1.489)%
  --(6.206,1.509)--(6.227,1.530)--(6.248,1.551)--(6.268,1.572)--(6.289,1.592)--(6.310,1.613)%
  --(6.331,1.634)--(6.351,1.655)--(6.372,1.675)--(6.393,1.696)--(6.413,1.717)--(6.434,1.737)%
  --(6.455,1.758)--(6.475,1.779)--(6.496,1.800)--(6.517,1.820)--(6.538,1.841)--(6.558,1.862)%
  --(6.579,1.882)--(6.600,1.903)--(6.620,1.924)--(6.641,1.945)--(6.662,1.965)--(6.683,1.986)%
  --(6.703,2.007)--(6.724,2.027)--(6.745,2.048)--(6.765,2.069)--(6.786,2.090)--(6.807,2.110)%
  --(6.827,2.131)--(6.848,2.152)--(6.869,2.172)--(6.890,2.193)--(6.910,2.214)--(6.931,2.235)%
  --(6.952,2.255)--(6.972,2.276)--(6.993,2.297)--(7.014,2.317)--(7.035,2.338)--(7.055,2.359)%
  --(7.076,2.380)--(7.097,2.400)--(7.117,2.421)--(7.138,2.442)--(7.159,2.462)--(7.179,2.483)%
  --(7.200,2.504)--(7.221,2.525)--(7.242,2.545)--(7.262,2.566)--(7.283,2.587)--(7.304,2.608)%
  --(7.324,2.628)--(7.345,2.649)--(7.366,2.670)--(7.387,2.690)--(7.407,2.711)--(7.428,2.732)%
  --(7.449,2.753)--(7.469,2.773)--(7.490,2.794)--(7.511,2.815)--(7.532,2.835)--(7.552,2.856)%
  --(7.573,2.877)--(7.594,2.898)--(7.614,2.918)--(7.635,2.939)--(7.656,2.960)--(7.676,2.980)%
  --(7.697,3.001)--(7.718,3.022)--(7.739,3.043)--(7.759,3.063);
\gpcolor{color=gp lt color 2}
\gppoint{gp mark 7}{(5.943,1.423)}
\gppoint{gp mark 7}{(5.943,1.426)}
\gppoint{gp mark 7}{(5.912,1.378)}
\gppoint{gp mark 7}{(5.817,1.289)}
\gppoint{gp mark 7}{(5.960,1.419)}
\gppoint{gp mark 7}{(6.439,2.747)}
\gppoint{gp mark 7}{(5.840,1.309)}
\gppoint{gp mark 7}{(5.790,1.278)}
\gppoint{gp mark 7}{(5.790,1.283)}
\gppoint{gp mark 7}{(5.790,1.345)}
\gppoint{gp mark 7}{(5.809,1.289)}
\gppoint{gp mark 7}{(5.825,1.294)}
\gppoint{gp mark 7}{(5.800,1.283)}
\gppoint{gp mark 7}{(5.950,1.334)}
\gppoint{gp mark 7}{(5.790,1.278)}
\gppoint{gp mark 7}{(5.950,1.340)}
\gppoint{gp mark 7}{(5.943,1.344)}
\gppoint{gp mark 7}{(5.780,1.272)}
\gppoint{gp mark 7}{(6.023,1.388)}
\gppoint{gp mark 7}{(5.809,1.278)}
\gppoint{gp mark 7}{(6.285,1.722)}
\gppoint{gp mark 7}{(6.269,1.710)}
\gppoint{gp mark 7}{(6.282,1.723)}
\gppoint{gp mark 7}{(6.263,1.712)}
\gppoint{gp mark 7}{(6.526,1.959)}
\gppoint{gp mark 7}{(6.530,1.957)}
\gppoint{gp mark 7}{(6.527,1.955)}
\gppoint{gp mark 7}{(6.530,1.956)}
\gppoint{gp mark 7}{(6.527,1.956)}
\gppoint{gp mark 7}{(6.528,1.957)}
\gppoint{gp mark 7}{(6.526,1.956)}
\gppoint{gp mark 7}{(6.527,1.955)}
\gppoint{gp mark 7}{(6.526,1.957)}
\gppoint{gp mark 7}{(6.529,1.957)}
\gppoint{gp mark 7}{(6.527,1.957)}
\gppoint{gp mark 7}{(6.529,1.957)}
\gppoint{gp mark 7}{(6.525,1.957)}
\gppoint{gp mark 7}{(6.528,1.957)}
\gppoint{gp mark 7}{(6.528,1.957)}
\gppoint{gp mark 7}{(6.528,1.955)}
\gppoint{gp mark 7}{(6.531,1.955)}
\gppoint{gp mark 7}{(6.526,1.959)}
\gppoint{gp mark 7}{(6.528,1.956)}
\gppoint{gp mark 7}{(6.527,1.956)}
\gppoint{gp mark 7}{(6.527,1.957)}
\gppoint{gp mark 7}{(6.525,1.956)}
\gppoint{gp mark 7}{(6.528,1.956)}
\gppoint{gp mark 7}{(6.524,1.956)}
\gppoint{gp mark 7}{(6.528,1.958)}
\gppoint{gp mark 7}{(6.530,1.957)}
\gppoint{gp mark 7}{(6.527,1.956)}
\gppoint{gp mark 7}{(6.528,1.956)}
\gppoint{gp mark 7}{(6.525,1.958)}
\gppoint{gp mark 7}{(6.528,1.954)}
\gppoint{gp mark 7}{(6.525,1.953)}
\gppoint{gp mark 7}{(6.524,1.956)}
\gppoint{gp mark 7}{(5.912,1.376)}
\gppoint{gp mark 7}{(6.056,1.508)}
\gppoint{gp mark 7}{(6.272,1.697)}
\gppoint{gp mark 7}{(5.893,1.364)}
\gppoint{gp mark 7}{(5.898,1.359)}
\gppoint{gp mark 7}{(5.932,1.394)}
\gppoint{gp mark 7}{(6.174,1.617)}
\gppoint{gp mark 7}{(5.769,1.294)}
\gppoint{gp mark 7}{(5.825,1.297)}
\gppoint{gp mark 7}{(5.817,1.289)}
\gppoint{gp mark 7}{(5.790,1.301)}
\gppoint{gp mark 7}{(6.198,1.629)}
\gppoint{gp mark 7}{(6.059,1.511)}
\gppoint{gp mark 7}{(5.840,1.342)}
\gppoint{gp mark 7}{(5.916,1.381)}
\gppoint{gp mark 7}{(5.963,1.420)}
\gppoint{gp mark 7}{(5.907,1.369)}
\gppoint{gp mark 7}{(6.090,1.542)}
\gppoint{gp mark 7}{(6.191,1.644)}
\gppoint{gp mark 7}{(5.866,1.338)}
\gppoint{gp mark 7}{(6.505,1.922)}
\gppoint{gp mark 7}{(7.383,2.790)}
\gppoint{gp mark 7}{(6.146,1.491)}
\gppoint{gp mark 7}{(6.142,1.493)}
\gppoint{gp mark 7}{(6.144,1.491)}
\gppoint{gp mark 7}{(6.140,1.491)}
\gppoint{gp mark 7}{(5.877,1.340)}
\gppoint{gp mark 7}{(5.866,1.349)}
\gppoint{gp mark 7}{(5.860,1.318)}
\gppoint{gp mark 7}{(6.044,1.497)}
\gppoint{gp mark 7}{(6.189,1.627)}
\gppoint{gp mark 7}{(5.866,1.336)}
\gppoint{gp mark 7}{(5.860,1.336)}
\gppoint{gp mark 7}{(5.877,1.347)}
\gppoint{gp mark 7}{(5.877,1.345)}
\gppoint{gp mark 7}{(6.452,1.883)}
\gppoint{gp mark 7}{(5.920,1.381)}
\gppoint{gp mark 7}{(5.860,1.336)}
\gpcolor{color=gp lt color border}
\gpsetlinetype{gp lt border}
\draw[gp path] (5.730,3.084)--(5.730,1.033)--(7.780,1.033)--(7.780,3.084)--cycle;
%% coordinates of the plot area
\gpdefrectangularnode{gp plot 5}{\pgfpoint{5.730cm}{1.033cm}}{\pgfpoint{7.780cm}{3.084cm}}
\draw[gp path] (9.897,1.033)--(10.077,1.033);
\node[gp node right] at (9.713,1.033) {{\supertiny $10^{-1}$}};
\draw[gp path] (9.897,1.375)--(10.077,1.375);
\node[gp node right] at (9.713,1.375) {{\supertiny $10^{0}$}};
\draw[gp path] (9.897,1.717)--(10.077,1.717);
\node[gp node right] at (9.713,1.717) {{\supertiny $10^{1}$}};
\draw[gp path] (9.897,2.059)--(10.077,2.059);
\node[gp node right] at (9.713,2.059) {{\supertiny $10^{2}$}};
\draw[gp path] (9.897,2.400)--(10.077,2.400);
\node[gp node right] at (9.713,2.400) {{\supertiny $10^{3}$}};
\draw[gp path] (9.897,2.742)--(10.077,2.742);
\node[gp node right] at (9.713,2.742) {{\supertiny $10^{4}$}};
\draw[gp path] (9.897,3.084)--(10.077,3.084);
\node[gp node right] at (9.713,3.084) {{\supertiny $10^{5}$}};
\draw[gp path] (9.897,1.033)--(9.897,1.213);
\node[gp node center] at (9.897,0.725) {{\supertiny $10^{-1}$}};
\draw[gp path] (10.239,1.033)--(10.239,1.213);
\node[gp node center] at (10.239,0.725) {{\supertiny $10^{0}$}};
\draw[gp path] (10.580,1.033)--(10.580,1.213);
\node[gp node center] at (10.580,0.725) {{\supertiny $10^{1}$}};
\draw[gp path] (10.922,1.033)--(10.922,1.213);
\node[gp node center] at (10.922,0.725) {{\supertiny $10^{2}$}};
\draw[gp path] (11.264,1.033)--(11.264,1.213);
\node[gp node center] at (11.264,0.725) {{\supertiny $10^{3}$}};
\draw[gp path] (11.605,1.033)--(11.605,1.213);
\node[gp node center] at (11.605,0.725) {{\supertiny $10^{4}$}};
\draw[gp path] (11.947,1.033)--(11.947,1.213);
\node[gp node center] at (11.947,0.725) {{\supertiny $10^{5}$}};
\draw[gp path] (9.897,3.084)--(9.897,1.033)--(11.947,1.033)--(11.947,3.084)--cycle;
\node[gp node center,rotate=-270] at (8.639,2.058) {Inv.\ w/ minimization};
\node[gp node center] at (10.922,0.263) {Inv.\ w/o minimization};
\gpcolor{color=gp lt color 0}
\gpsetlinetype{gp lt plot 0}
\draw[gp path] (9.897,1.033)--(9.918,1.054)--(9.938,1.074)--(9.959,1.095)--(9.980,1.116)%
  --(10.001,1.137)--(10.021,1.157)--(10.042,1.178)--(10.063,1.199)--(10.083,1.219)--(10.104,1.240)%
  --(10.125,1.261)--(10.145,1.282)--(10.166,1.302)--(10.187,1.323)--(10.208,1.344)--(10.228,1.364)%
  --(10.249,1.385)--(10.270,1.406)--(10.290,1.427)--(10.311,1.447)--(10.332,1.468)--(10.353,1.489)%
  --(10.373,1.509)--(10.394,1.530)--(10.415,1.551)--(10.435,1.572)--(10.456,1.592)--(10.477,1.613)%
  --(10.498,1.634)--(10.518,1.655)--(10.539,1.675)--(10.560,1.696)--(10.580,1.717)--(10.601,1.737)%
  --(10.622,1.758)--(10.642,1.779)--(10.663,1.800)--(10.684,1.820)--(10.705,1.841)--(10.725,1.862)%
  --(10.746,1.882)--(10.767,1.903)--(10.787,1.924)--(10.808,1.945)--(10.829,1.965)--(10.850,1.986)%
  --(10.870,2.007)--(10.891,2.027)--(10.912,2.048)--(10.932,2.069)--(10.953,2.090)--(10.974,2.110)%
  --(10.994,2.131)--(11.015,2.152)--(11.036,2.172)--(11.057,2.193)--(11.077,2.214)--(11.098,2.235)%
  --(11.119,2.255)--(11.139,2.276)--(11.160,2.297)--(11.181,2.317)--(11.202,2.338)--(11.222,2.359)%
  --(11.243,2.380)--(11.264,2.400)--(11.284,2.421)--(11.305,2.442)--(11.326,2.462)--(11.346,2.483)%
  --(11.367,2.504)--(11.388,2.525)--(11.409,2.545)--(11.429,2.566)--(11.450,2.587)--(11.471,2.608)%
  --(11.491,2.628)--(11.512,2.649)--(11.533,2.670)--(11.554,2.690)--(11.574,2.711)--(11.595,2.732)%
  --(11.616,2.753)--(11.636,2.773)--(11.657,2.794)--(11.678,2.815)--(11.699,2.835)--(11.719,2.856)%
  --(11.740,2.877)--(11.761,2.898)--(11.781,2.918)--(11.802,2.939)--(11.823,2.960)--(11.843,2.980)%
  --(11.864,3.001)--(11.885,3.022)--(11.906,3.043)--(11.926,3.063);
\gpcolor{color=gp lt color 2}
\gppoint{gp mark 7}{(10.286,1.444)}
\gppoint{gp mark 7}{(10.290,1.440)}
\gppoint{gp mark 7}{(10.242,1.404)}
\gppoint{gp mark 7}{(10.153,1.320)}
\gppoint{gp mark 7}{(10.283,1.447)}
\gppoint{gp mark 7}{(11.610,2.778)}
\gppoint{gp mark 7}{(10.172,1.332)}
\gppoint{gp mark 7}{(10.142,1.304)}
\gppoint{gp mark 7}{(10.147,1.291)}
\gppoint{gp mark 7}{(10.209,1.373)}
\gppoint{gp mark 7}{(10.153,1.309)}
\gppoint{gp mark 7}{(10.158,1.318)}
\gppoint{gp mark 7}{(10.147,1.301)}
\gppoint{gp mark 7}{(10.198,1.347)}
\gppoint{gp mark 7}{(10.142,1.289)}
\gppoint{gp mark 7}{(10.204,1.354)}
\gppoint{gp mark 7}{(10.207,1.356)}
\gppoint{gp mark 7}{(10.136,1.299)}
\gppoint{gp mark 7}{(10.251,1.397)}
\gppoint{gp mark 7}{(10.142,1.297)}
\gppoint{gp mark 7}{(10.585,1.742)}
\gppoint{gp mark 7}{(10.574,1.742)}
\gppoint{gp mark 7}{(10.586,1.746)}
\gppoint{gp mark 7}{(10.575,1.743)}
\gppoint{gp mark 7}{(10.822,1.992)}
\gppoint{gp mark 7}{(10.820,1.995)}
\gppoint{gp mark 7}{(10.819,1.992)}
\gppoint{gp mark 7}{(10.819,1.994)}
\gppoint{gp mark 7}{(10.819,1.994)}
\gppoint{gp mark 7}{(10.821,1.997)}
\gppoint{gp mark 7}{(10.819,1.994)}
\gppoint{gp mark 7}{(10.818,1.999)}
\gppoint{gp mark 7}{(10.820,1.989)}
\gppoint{gp mark 7}{(10.820,1.994)}
\gppoint{gp mark 7}{(10.821,1.997)}
\gppoint{gp mark 7}{(10.820,1.990)}
\gppoint{gp mark 7}{(10.820,1.996)}
\gppoint{gp mark 7}{(10.820,1.988)}
\gppoint{gp mark 7}{(10.821,1.982)}
\gppoint{gp mark 7}{(10.819,1.987)}
\gppoint{gp mark 7}{(10.819,1.992)}
\gppoint{gp mark 7}{(10.823,1.989)}
\gppoint{gp mark 7}{(10.819,1.987)}
\gppoint{gp mark 7}{(10.820,1.986)}
\gppoint{gp mark 7}{(10.820,1.994)}
\gppoint{gp mark 7}{(10.820,1.991)}
\gppoint{gp mark 7}{(10.819,1.999)}
\gppoint{gp mark 7}{(10.819,1.999)}
\gppoint{gp mark 7}{(10.822,1.992)}
\gppoint{gp mark 7}{(10.821,1.995)}
\gppoint{gp mark 7}{(10.820,2.000)}
\gppoint{gp mark 7}{(10.820,2.003)}
\gppoint{gp mark 7}{(10.821,1.987)}
\gppoint{gp mark 7}{(10.817,1.994)}
\gppoint{gp mark 7}{(10.817,1.999)}
\gppoint{gp mark 7}{(10.819,1.994)}
\gppoint{gp mark 7}{(10.240,1.464)}
\gppoint{gp mark 7}{(10.372,1.628)}
\gppoint{gp mark 7}{(10.560,1.931)}
\gppoint{gp mark 7}{(10.228,1.470)}
\gppoint{gp mark 7}{(10.223,1.464)}
\gppoint{gp mark 7}{(10.258,1.514)}
\gppoint{gp mark 7}{(10.481,1.889)}
\gppoint{gp mark 7}{(10.158,1.309)}
\gppoint{gp mark 7}{(10.160,1.324)}
\gppoint{gp mark 7}{(10.153,1.322)}
\gppoint{gp mark 7}{(10.165,1.352)}
\gppoint{gp mark 7}{(10.493,1.739)}
\gppoint{gp mark 7}{(10.375,1.597)}
\gppoint{gp mark 7}{(10.206,1.403)}
\gppoint{gp mark 7}{(10.244,1.478)}
\gppoint{gp mark 7}{(10.284,1.538)}
\gppoint{gp mark 7}{(10.233,1.493)}
\gppoint{gp mark 7}{(10.406,1.895)}
\gppoint{gp mark 7}{(10.508,2.083)}
\gppoint{gp mark 7}{(10.202,1.417)}
\gppoint{gp mark 7}{(10.786,2.080)}
\gppoint{gp mark 7}{(11.654,2.807)}
\gppoint{gp mark 7}{(10.355,1.638)}
\gppoint{gp mark 7}{(10.357,1.636)}
\gppoint{gp mark 7}{(10.354,1.575)}
\gppoint{gp mark 7}{(10.354,1.578)}
\gppoint{gp mark 7}{(10.204,1.396)}
\gppoint{gp mark 7}{(10.213,1.397)}
\gppoint{gp mark 7}{(10.181,1.376)}
\gppoint{gp mark 7}{(10.360,1.640)}
\gppoint{gp mark 7}{(10.491,1.795)}
\gppoint{gp mark 7}{(10.200,1.390)}
\gppoint{gp mark 7}{(10.200,1.389)}
\gppoint{gp mark 7}{(10.211,1.446)}
\gppoint{gp mark 7}{(10.209,1.453)}
\gppoint{gp mark 7}{(10.746,2.310)}
\gppoint{gp mark 7}{(10.244,1.408)}
\gppoint{gp mark 7}{(10.200,1.361)}
\gpcolor{color=gp lt color border}
\gpsetlinetype{gp lt border}
\draw[gp path] (9.897,3.084)--(9.897,1.033)--(11.947,1.033)--(11.947,3.084)--cycle;
%% coordinates of the plot area
\gpdefrectangularnode{gp plot 6}{\pgfpoint{9.897cm}{1.033cm}}{\pgfpoint{11.947cm}{3.084cm}}
\end{tikzpicture}
%% gnuplot variables

  \caption[Invariant sizes and time overhead of Petrinizer options.]{Graph on the top left shows a relation of sizes of constructed
  invariants to the number of places in the corresponding Petri nets. Graph
  on the bottom left shows comparison in size of invariants produced by Petrinizer and
  IIC\@. Axes represent size on a logarithmic scale. Each dot represents
  one example.
  The four graphs in the center and on the right show time
  overhead of integer arithmetic, trap refinement, invariant
  construction and invariant minimization. Axes represent time in
  seconds on a logarithmic scale. Each dot represents execution time
  on one example. The graph on the top right only shows examples for
  which at least one trap appeared in the refinement. Similarly, the
  bottom center and bottom right graphs only show safe examples.}
\label{fig:petrinizer-various}
\end{figure}

\paragraph{Invariant sizes.} 
We measure the size of inductive invariants produced by Pe\-tri\-ni\-zer without
minimization.
We took the number of atomic (non-zero) terms appearing in an invariant's linear expressions
as a measure of its size. When we relate sizes of invariants to number of places
in the corresponding Petri net (top left graph in Fig.~\ref{fig:petrinizer-various}), we
see that invariants are usually very succinct. As an example, the
largest invariant had 814 atomic terms, and the corresponding Petri net, coming
from the Erlang suite, had 4,763 places. For the largest Petri net, 
with 66,950 places, the constructed invariant had 339 atomic
terms.

The added benefit of minimization is negligible: there are only four examples where the invariant
was reduced, and the reduction was about 2–3\%. 
Thus, invariant minimization does not pay off for these examples.

We also compared sizes of constructed invariants with
sizes of invariants produced by IIC~\cite{KloosMNP13}. IIC's invariants
are expressed as CNF formulas over atoms of the form $x<a$,
for a variable $x$ and a constant $a$. As a measure of size for these formulas,
we took the number of atoms they contain. As the bottom left graph in
Fig.~\ref{fig:petrinizer-various} shows, when compared to IIC's invariants, Petrinizer's invariants
are never larger, and are often orders of magnitude smaller.

\begin{table}[t]
  \centering
  \caption[Mean and median times in seconds for each tool.]
  {Mean and median times in seconds for each tool. We report
  times for safe examples, as well as for all examples. Memory-out cases were
  set to the timeout value of 100,000 s. Symbols
  $\mathbb{Q}$ and $\mathbb{Z}$ denote rational and integer numbers.}
\label{table:mean-median-times}
  \scriptsize
  \begin{tabular}{lrrrrrr}
    \toprule
    Method/tool & Safety/$\mathbb{Q}$ & Safety/$\mathbb{Z}$ & Ref./$\mathbb{Q}$ &
    Ref./$\mathbb{Z}$ & Safety+inv. & Safety+inv.min. \\
    \midrule
    Mean (safe)   & 69.26 & 70.20 & 69.36 & 72.20 & 168.46 & 203.05 \\
    Median (safe) & 2.45 & 2.23 & 2.35 & 3.81 & 3.70 & 4.03 \\
    \midrule
    Mean (all)    & 45.17 & 46.04 & 45.52 & 47.70 & 109.23 & 131.58 \\
    Median (all)  & 0.44 & 0.43 & 0.90 & 0.93 & 0.66 & 1.00 \\
    \bottomrule
  \end{tabular}
  
  \vspace{1ex}
  
  \begin{tabular}{lrrrrr}
    \toprule
    Method/tool &
    Ref.+inv. & Ref.+inv.min. & IIC & BFC & MIST \\
    \midrule
    Mean (safe)   & 228.88 & 275.12 & 56954.09 & 47126.12 & 69196.77 \\
    Median (safe) & 5.96 & 6.30 & 100000.00 & 1642.43 & 100000.00 \\
    \midrule
    Mean (all)    & 148.57 & 178.45 & 44089.93 & 31017.80 & 61586.56 \\
    Median (all)  & 1.37 & 1.94 & 138.00 & 0.77 & 100000.00 \\
    \bottomrule
  \end{tabular}
\end{table}

\paragraph{Performance.} To ensure accuracy and fairness, all experiments were
performed on identical machines, equipped with Intel Xeon 2.66 GHz CPUs and 48 GB of memory, running Linux 3.2.48.1
in 64-bit mode. Execution time was limited to 100,000 seconds (27
hours, 46 minutes and 40 seconds), and memory to 2 GB\@.

Due to dissimilarities between the compared tools, selecting a fair
measure of time was non-trivial. On the one hand, as
Petrinizer communicates with Z3 via temporary files, it spends a
considerable amount of time doing I/O operations.  On the other hand,
as BFC performs both a forward and a backward search, it naturally
splits the work into two threads, and runs them in parallel on two CPU
cores. In both cases, the actual elapsed time does not quite
correspond to the amount of computational effort we wanted to measure.
Therefore, for the measure of time we selected the \emph{user time},
as reported by the \emph{time} utility on Linux. User time measures
the total CPU time spent executing the process and its children. In
the case of Petrinizer, it excludes the I/O overhead, and in the case
of BFC, it includes total CPU time spent on both CPU cores.

We report mean and median times measured for each tool in
Table~\ref{table:mean-median-times}.

\paragraph{Time overhead of Petrinizer's methods.} Before comparing Petrinizer
with other tools, we analyze time overhead of integer arithmetic, trap refinement, invariant construction, and invariant minimization.
The four graphs in the center and on the right in
Fig.~\ref{fig:petrinizer-various} summarize the results. The top
central graph shows that the difference in performance between integer
and rational arithmetic is negligible.

The top right graph in Fig.~\ref{fig:petrinizer-various} shows that traps
incur a significant overhead. This is not too surprising as, each time
a trap is found, the main system has to be updated with a new trap
constraint and solved again. Thus the actual overhead depends on the number
of traps that appear during the refinement. In the experiments, there
were 32 examples for refinement with integer arithmetic where traps
appeared at least once. The maximal number of traps in a single example was 9.
In the examples where traps appear once, we see a slowdown of 2–3$\times$. In
the extreme cases with 9 traps we see slowdowns of 10–16$\times$.

In the case of invariant construction, as shown on the bottom central graph in
Fig.~\ref{fig:petrinizer-various}, the overhead is more
uniform and predictable. The reason is that constructing the invariant
involves solving the dual form of the main system as many times as
there are disjuncts in the property violation constraint. In most
cases, the property violation constraint has one disjunct. A single example
with many disjuncts, having 8989 of them, appears on the graph
as an outlier.

In the case of invariant minimization, as the bottom right
graph in Fig.~\ref{fig:petrinizer-various} shows, time overhead is
quite severe.
The underlying data contains examples of slowdowns of up to 30$\times$.

\begin{figure}[t]
  \centering \begin{tikzpicture}[gnuplot]
%% generated with GNUPLOT 4.6p3 (Lua 5.1; terminal rev. 99, script rev. 100)
%% Mon May 12 23:25:45 2014
\path (0.000,0.000) rectangle (12.500,7.000);
\gpcolor{color=gp lt color border}
\gpsetlinetype{gp lt border}
\gpsetlinewidth{1.00}
\draw[gp path] (1.564,4.533)--(1.744,4.533);
\node[gp node right] at (1.380,4.533) {{\supertiny $10^{-2}$}};
\draw[gp path] (1.564,4.826)--(1.744,4.826);
\node[gp node right] at (1.380,4.826) {{\supertiny $10^{-1}$}};
\draw[gp path] (1.564,5.119)--(1.744,5.119);
\node[gp node right] at (1.380,5.119) {{\supertiny $10^{0}$}};
\draw[gp path] (1.564,5.412)--(1.744,5.412);
\node[gp node right] at (1.380,5.412) {{\supertiny $10^{1}$}};
\draw[gp path] (1.564,5.704)--(1.744,5.704);
\node[gp node right] at (1.380,5.704) {{\supertiny $10^{2}$}};
\draw[gp path] (1.564,5.997)--(1.744,5.997);
\node[gp node right] at (1.380,5.997) {{\supertiny $10^{3}$}};
\draw[gp path] (1.564,6.290)--(1.744,6.290);
\node[gp node right] at (1.380,6.290) {{\supertiny $10^{4}$}};
\draw[gp path] (1.564,6.583)--(1.744,6.583);
\node[gp node right] at (1.380,6.583) {{\supertiny $10^{5}$}};
\draw[gp path] (1.564,4.533)--(1.564,4.713);
\node[gp node center] at (1.564,4.225) {{\supertiny $10^{-2}$}};
\draw[gp path] (1.857,4.533)--(1.857,4.713);
\node[gp node center] at (1.857,4.225) {{\supertiny $10^{-1}$}};
\draw[gp path] (2.150,4.533)--(2.150,4.713);
\node[gp node center] at (2.150,4.225) {{\supertiny $10^{0}$}};
\draw[gp path] (2.443,4.533)--(2.443,4.713);
\node[gp node center] at (2.443,4.225) {{\supertiny $10^{1}$}};
\draw[gp path] (2.735,4.533)--(2.735,4.713);
\node[gp node center] at (2.735,4.225) {{\supertiny $10^{2}$}};
\draw[gp path] (3.028,4.533)--(3.028,4.713);
\node[gp node center] at (3.028,4.225) {{\supertiny $10^{3}$}};
\draw[gp path] (3.321,4.533)--(3.321,4.713);
\node[gp node center] at (3.321,4.225) {{\supertiny $10^{4}$}};
\draw[gp path] (3.614,4.533)--(3.614,4.713);
\node[gp node center] at (3.614,4.225) {{\supertiny $10^{5}$}};
\draw[gp path] (1.564,6.583)--(1.564,4.533)--(3.614,4.533)--(3.614,6.583)--cycle;
\node[gp node center,rotate=-270] at (0.490,5.558) {IIC};
\node[gp node center] at (2.589,3.763) {Ref.\ over integers};
\gpcolor{color=gp lt color 0}
\gpsetlinetype{gp lt plot 0}
\draw[gp path] (1.564,4.533)--(1.585,4.554)--(1.605,4.574)--(1.626,4.595)--(1.647,4.616)%
  --(1.668,4.637)--(1.688,4.657)--(1.709,4.678)--(1.730,4.699)--(1.750,4.719)--(1.771,4.740)%
  --(1.792,4.761)--(1.812,4.781)--(1.833,4.802)--(1.854,4.823)--(1.875,4.844)--(1.895,4.864)%
  --(1.916,4.885)--(1.937,4.906)--(1.957,4.926)--(1.978,4.947)--(1.999,4.968)--(2.020,4.989)%
  --(2.040,5.009)--(2.061,5.030)--(2.082,5.051)--(2.102,5.071)--(2.123,5.092)--(2.144,5.113)%
  --(2.165,5.134)--(2.185,5.154)--(2.206,5.175)--(2.227,5.196)--(2.247,5.216)--(2.268,5.237)%
  --(2.289,5.258)--(2.309,5.278)--(2.330,5.299)--(2.351,5.320)--(2.372,5.341)--(2.392,5.361)%
  --(2.413,5.382)--(2.434,5.403)--(2.454,5.423)--(2.475,5.444)--(2.496,5.465)--(2.517,5.486)%
  --(2.537,5.506)--(2.558,5.527)--(2.579,5.548)--(2.599,5.568)--(2.620,5.589)--(2.641,5.610)%
  --(2.661,5.630)--(2.682,5.651)--(2.703,5.672)--(2.724,5.693)--(2.744,5.713)--(2.765,5.734)%
  --(2.786,5.755)--(2.806,5.775)--(2.827,5.796)--(2.848,5.817)--(2.869,5.838)--(2.889,5.858)%
  --(2.910,5.879)--(2.931,5.900)--(2.951,5.920)--(2.972,5.941)--(2.993,5.962)--(3.013,5.982)%
  --(3.034,6.003)--(3.055,6.024)--(3.076,6.045)--(3.096,6.065)--(3.117,6.086)--(3.138,6.107)%
  --(3.158,6.127)--(3.179,6.148)--(3.200,6.169)--(3.221,6.190)--(3.241,6.210)--(3.262,6.231)%
  --(3.283,6.252)--(3.303,6.272)--(3.324,6.293)--(3.345,6.314)--(3.366,6.335)--(3.386,6.355)%
  --(3.407,6.376)--(3.428,6.397)--(3.448,6.417)--(3.469,6.438)--(3.490,6.459)--(3.510,6.479)%
  --(3.531,6.500)--(3.552,6.521)--(3.573,6.542)--(3.593,6.562);
\gpcolor{color=gp lt color 2}
\gpsetpointsize{2.00}
\gppoint{gp mark 7}{(2.009,6.068)}
\gppoint{gp mark 7}{(1.924,4.957)}
\gppoint{gp mark 7}{(2.054,6.476)}
\gppoint{gp mark 7}{(2.534,6.484)}
\gppoint{gp mark 7}{(1.951,5.293)}
\gppoint{gp mark 7}{(1.968,4.947)}
\gppoint{gp mark 7}{(1.880,5.011)}
\gppoint{gp mark 7}{(1.973,5.075)}
\gppoint{gp mark 7}{(1.978,4.978)}
\gppoint{gp mark 7}{(1.917,4.533)}
\gppoint{gp mark 7}{(2.327,5.025)}
\gppoint{gp mark 7}{(2.396,6.583)}
\gppoint{gp mark 7}{(2.398,6.583)}
\gppoint{gp mark 7}{(2.320,5.008)}
\gppoint{gp mark 7}{(2.398,5.287)}
\gppoint{gp mark 7}{(2.394,5.689)}
\gppoint{gp mark 7}{(2.322,5.014)}
\gppoint{gp mark 7}{(2.404,5.285)}
\gppoint{gp mark 7}{(2.395,6.583)}
\gppoint{gp mark 7}{(2.323,5.011)}
\gppoint{gp mark 7}{(2.405,5.286)}
\gppoint{gp mark 7}{(2.397,5.688)}
\gppoint{gp mark 7}{(2.536,6.583)}
\gppoint{gp mark 7}{(2.537,6.583)}
\gppoint{gp mark 7}{(2.539,6.583)}
\gppoint{gp mark 7}{(2.537,6.583)}
\gppoint{gp mark 7}{(2.659,6.583)}
\gppoint{gp mark 7}{(2.537,6.583)}
\gppoint{gp mark 7}{(2.538,6.583)}
\gppoint{gp mark 7}{(2.539,6.583)}
\gppoint{gp mark 7}{(2.541,6.583)}
\gppoint{gp mark 7}{(2.656,6.583)}
\gppoint{gp mark 7}{(2.537,6.583)}
\gppoint{gp mark 7}{(2.537,6.583)}
\gppoint{gp mark 7}{(2.538,6.583)}
\gppoint{gp mark 7}{(2.537,6.583)}
\gppoint{gp mark 7}{(2.657,6.583)}
\gppoint{gp mark 7}{(2.534,6.583)}
\gppoint{gp mark 7}{(2.538,6.583)}
\gppoint{gp mark 7}{(2.533,6.583)}
\gppoint{gp mark 7}{(2.537,6.583)}
\gppoint{gp mark 7}{(2.656,6.583)}
\gppoint{gp mark 7}{(2.538,6.583)}
\gppoint{gp mark 7}{(2.534,6.583)}
\gppoint{gp mark 7}{(2.537,6.583)}
\gppoint{gp mark 7}{(2.539,6.583)}
\gppoint{gp mark 7}{(2.656,6.583)}
\gppoint{gp mark 7}{(2.534,6.583)}
\gppoint{gp mark 7}{(2.536,6.583)}
\gppoint{gp mark 7}{(2.535,6.583)}
\gppoint{gp mark 7}{(2.539,6.583)}
\gppoint{gp mark 7}{(2.735,6.583)}
\gppoint{gp mark 7}{(2.537,6.583)}
\gppoint{gp mark 7}{(2.536,6.583)}
\gppoint{gp mark 7}{(2.653,6.583)}
\gppoint{gp mark 7}{(2.538,6.583)}
\gppoint{gp mark 7}{(2.539,6.583)}
\gppoint{gp mark 7}{(2.730,6.583)}
\gppoint{gp mark 7}{(2.539,6.583)}
\gppoint{gp mark 7}{(2.536,6.583)}
\gppoint{gp mark 7}{(2.538,6.583)}
\gppoint{gp mark 7}{(2.537,6.583)}
\gppoint{gp mark 7}{(2.729,6.583)}
\gppoint{gp mark 7}{(2.009,6.583)}
\gppoint{gp mark 7}{(2.133,6.583)}
\gppoint{gp mark 7}{(2.296,6.583)}
\gppoint{gp mark 7}{(2.120,5.482)}
\gppoint{gp mark 7}{(1.997,6.583)}
\gppoint{gp mark 7}{(2.001,6.583)}
\gppoint{gp mark 7}{(2.118,5.033)}
\gppoint{gp mark 7}{(2.023,4.673)}
\gppoint{gp mark 7}{(2.235,5.176)}
\gppoint{gp mark 7}{(2.106,5.085)}
\gppoint{gp mark 7}{(2.256,6.583)}
\gppoint{gp mark 7}{(2.578,6.583)}
\gppoint{gp mark 7}{(1.857,4.886)}
\gppoint{gp mark 7}{(1.932,4.886)}
\gppoint{gp mark 7}{(1.917,4.877)}
\gppoint{gp mark 7}{(1.938,4.838)}
\gppoint{gp mark 7}{(2.253,6.583)}
\gppoint{gp mark 7}{(2.142,6.583)}
\gppoint{gp mark 7}{(1.951,6.583)}
\gppoint{gp mark 7}{(2.016,6.583)}
\gppoint{gp mark 7}{(2.056,6.583)}
\gppoint{gp mark 7}{(2.001,6.583)}
\gppoint{gp mark 7}{(2.157,6.583)}
\gppoint{gp mark 7}{(2.233,6.583)}
\gppoint{gp mark 7}{(1.968,6.537)}
\gppoint{gp mark 7}{(2.521,6.583)}
\gppoint{gp mark 7}{(2.173,6.583)}
\gppoint{gp mark 7}{(2.166,6.583)}
\gppoint{gp mark 7}{(2.210,5.740)}
\gppoint{gp mark 7}{(2.208,5.745)}
\gppoint{gp mark 7}{(1.983,5.658)}
\gppoint{gp mark 7}{(1.978,5.672)}
\gppoint{gp mark 7}{(2.336,5.837)}
\gppoint{gp mark 7}{(2.332,5.853)}
\gppoint{gp mark 7}{(2.095,4.992)}
\gppoint{gp mark 7}{(1.968,6.285)}
\gppoint{gp mark 7}{(2.139,6.583)}
\gppoint{gp mark 7}{(2.261,6.583)}
\gppoint{gp mark 7}{(1.973,6.386)}
\gppoint{gp mark 7}{(1.951,6.583)}
\gppoint{gp mark 7}{(1.978,5.847)}
\gppoint{gp mark 7}{(1.968,5.540)}
\gppoint{gp mark 7}{(2.478,6.583)}
\gppoint{gp mark 7}{(2.080,4.709)}
\gppoint{gp mark 7}{(2.078,4.673)}
\gppoint{gp mark 7}{(2.080,4.709)}
\gppoint{gp mark 7}{(2.396,6.583)}
\gppoint{gp mark 7}{(2.257,5.102)}
\gppoint{gp mark 7}{(2.467,5.988)}
\gppoint{gp mark 7}{(2.160,4.942)}
\gppoint{gp mark 7}{(2.016,4.709)}
\gppoint{gp mark 7}{(2.106,4.797)}
\gppoint{gp mark 7}{(2.013,4.673)}
\gppoint{gp mark 7}{(2.210,6.583)}
\gppoint{gp mark 7}{(2.189,4.812)}
\gppoint{gp mark 7}{(2.487,6.583)}
\gppoint{gp mark 7}{(2.488,6.583)}
\gppoint{gp mark 7}{(2.132,4.849)}
\gppoint{gp mark 7}{(2.249,6.583)}
\gppoint{gp mark 7}{(2.151,6.583)}
\gppoint{gp mark 7}{(2.299,6.583)}
\gppoint{gp mark 7}{(2.036,4.932)}
\gppoint{gp mark 7}{(2.232,6.583)}
\gppoint{gp mark 7}{(1.978,4.533)}
\gppoint{gp mark 7}{(2.258,6.253)}
\gppoint{gp mark 7}{(2.095,5.075)}
\gppoint{gp mark 7}{(2.097,5.075)}
\gppoint{gp mark 7}{(2.254,6.121)}
\gppoint{gp mark 7}{(2.253,6.583)}
\gppoint{gp mark 7}{(2.069,4.709)}
\gppoint{gp mark 7}{(2.163,5.149)}
\gppoint{gp mark 7}{(2.111,5.459)}
\gppoint{gp mark 7}{(1.973,4.533)}
\gppoint{gp mark 7}{(2.131,4.974)}
\gppoint{gp mark 7}{(1.988,4.533)}
\gppoint{gp mark 7}{(2.023,4.838)}
\gppoint{gp mark 7}{(2.085,5.420)}
\gppoint{gp mark 7}{(1.988,4.533)}
\gppoint{gp mark 7}{(2.135,5.591)}
\gppoint{gp mark 7}{(2.045,4.812)}
\gppoint{gp mark 7}{(2.489,6.583)}
\gpcolor{color=gp lt color border}
\gpsetlinetype{gp lt border}
\draw[gp path] (1.564,6.583)--(1.564,4.533)--(3.614,4.533)--(3.614,6.583)--cycle;
%% coordinates of the plot area
\gpdefrectangularnode{gp plot 1}{\pgfpoint{1.564cm}{4.533cm}}{\pgfpoint{3.614cm}{6.583cm}}
\draw[gp path] (5.730,4.533)--(5.910,4.533);
\node[gp node right] at (5.546,4.533) {{\supertiny $10^{-2}$}};
\draw[gp path] (5.730,4.826)--(5.910,4.826);
\node[gp node right] at (5.546,4.826) {{\supertiny $10^{-1}$}};
\draw[gp path] (5.730,5.119)--(5.910,5.119);
\node[gp node right] at (5.546,5.119) {{\supertiny $10^{0}$}};
\draw[gp path] (5.730,5.412)--(5.910,5.412);
\node[gp node right] at (5.546,5.412) {{\supertiny $10^{1}$}};
\draw[gp path] (5.730,5.704)--(5.910,5.704);
\node[gp node right] at (5.546,5.704) {{\supertiny $10^{2}$}};
\draw[gp path] (5.730,5.997)--(5.910,5.997);
\node[gp node right] at (5.546,5.997) {{\supertiny $10^{3}$}};
\draw[gp path] (5.730,6.290)--(5.910,6.290);
\node[gp node right] at (5.546,6.290) {{\supertiny $10^{4}$}};
\draw[gp path] (5.730,6.583)--(5.910,6.583);
\node[gp node right] at (5.546,6.583) {{\supertiny $10^{5}$}};
\draw[gp path] (5.730,4.533)--(5.730,4.713);
\node[gp node center] at (5.730,4.225) {{\supertiny $10^{-2}$}};
\draw[gp path] (6.023,4.533)--(6.023,4.713);
\node[gp node center] at (6.023,4.225) {{\supertiny $10^{-1}$}};
\draw[gp path] (6.316,4.533)--(6.316,4.713);
\node[gp node center] at (6.316,4.225) {{\supertiny $10^{0}$}};
\draw[gp path] (6.609,4.533)--(6.609,4.713);
\node[gp node center] at (6.609,4.225) {{\supertiny $10^{1}$}};
\draw[gp path] (6.901,4.533)--(6.901,4.713);
\node[gp node center] at (6.901,4.225) {{\supertiny $10^{2}$}};
\draw[gp path] (7.194,4.533)--(7.194,4.713);
\node[gp node center] at (7.194,4.225) {{\supertiny $10^{3}$}};
\draw[gp path] (7.487,4.533)--(7.487,4.713);
\node[gp node center] at (7.487,4.225) {{\supertiny $10^{4}$}};
\draw[gp path] (7.780,4.533)--(7.780,4.713);
\node[gp node center] at (7.780,4.225) {{\supertiny $10^{5}$}};
\draw[gp path] (5.730,6.583)--(5.730,4.533)--(7.780,4.533)--(7.780,6.583)--cycle;
\node[gp node center,rotate=-270] at (4.656,5.558) {BFC};
\node[gp node center] at (6.755,3.763) {Ref.\ over integers};
\gpcolor{color=gp lt color 0}
\gpsetlinetype{gp lt plot 0}
\draw[gp path] (5.730,4.533)--(5.751,4.554)--(5.771,4.574)--(5.792,4.595)--(5.813,4.616)%
  --(5.834,4.637)--(5.854,4.657)--(5.875,4.678)--(5.896,4.699)--(5.916,4.719)--(5.937,4.740)%
  --(5.958,4.761)--(5.978,4.781)--(5.999,4.802)--(6.020,4.823)--(6.041,4.844)--(6.061,4.864)%
  --(6.082,4.885)--(6.103,4.906)--(6.123,4.926)--(6.144,4.947)--(6.165,4.968)--(6.186,4.989)%
  --(6.206,5.009)--(6.227,5.030)--(6.248,5.051)--(6.268,5.071)--(6.289,5.092)--(6.310,5.113)%
  --(6.331,5.134)--(6.351,5.154)--(6.372,5.175)--(6.393,5.196)--(6.413,5.216)--(6.434,5.237)%
  --(6.455,5.258)--(6.475,5.278)--(6.496,5.299)--(6.517,5.320)--(6.538,5.341)--(6.558,5.361)%
  --(6.579,5.382)--(6.600,5.403)--(6.620,5.423)--(6.641,5.444)--(6.662,5.465)--(6.683,5.486)%
  --(6.703,5.506)--(6.724,5.527)--(6.745,5.548)--(6.765,5.568)--(6.786,5.589)--(6.807,5.610)%
  --(6.827,5.630)--(6.848,5.651)--(6.869,5.672)--(6.890,5.693)--(6.910,5.713)--(6.931,5.734)%
  --(6.952,5.755)--(6.972,5.775)--(6.993,5.796)--(7.014,5.817)--(7.035,5.838)--(7.055,5.858)%
  --(7.076,5.879)--(7.097,5.900)--(7.117,5.920)--(7.138,5.941)--(7.159,5.962)--(7.179,5.982)%
  --(7.200,6.003)--(7.221,6.024)--(7.242,6.045)--(7.262,6.065)--(7.283,6.086)--(7.304,6.107)%
  --(7.324,6.127)--(7.345,6.148)--(7.366,6.169)--(7.387,6.190)--(7.407,6.210)--(7.428,6.231)%
  --(7.449,6.252)--(7.469,6.272)--(7.490,6.293)--(7.511,6.314)--(7.532,6.335)--(7.552,6.355)%
  --(7.573,6.376)--(7.594,6.397)--(7.614,6.417)--(7.635,6.438)--(7.656,6.459)--(7.676,6.479)%
  --(7.697,6.500)--(7.718,6.521)--(7.739,6.542)--(7.759,6.562);
\gpcolor{color=gp lt color 2}
\gppoint{gp mark 7}{(6.175,5.325)}
\gppoint{gp mark 7}{(6.090,4.533)}
\gppoint{gp mark 7}{(6.220,5.657)}
\gppoint{gp mark 7}{(6.700,5.667)}
\gppoint{gp mark 7}{(6.117,4.849)}
\gppoint{gp mark 7}{(6.134,6.583)}
\gppoint{gp mark 7}{(6.139,6.583)}
\gppoint{gp mark 7}{(6.074,4.533)}
\gppoint{gp mark 7}{(6.083,5.312)}
\gppoint{gp mark 7}{(6.129,5.070)}
\gppoint{gp mark 7}{(6.090,5.072)}
\gppoint{gp mark 7}{(6.098,5.561)}
\gppoint{gp mark 7}{(6.046,5.337)}
\gppoint{gp mark 7}{(6.139,5.247)}
\gppoint{gp mark 7}{(6.493,6.052)}
\gppoint{gp mark 7}{(6.562,6.049)}
\gppoint{gp mark 7}{(6.564,6.049)}
\gppoint{gp mark 7}{(6.486,6.056)}
\gppoint{gp mark 7}{(6.564,6.071)}
\gppoint{gp mark 7}{(6.560,6.060)}
\gppoint{gp mark 7}{(6.488,6.061)}
\gppoint{gp mark 7}{(6.570,6.056)}
\gppoint{gp mark 7}{(6.561,6.072)}
\gppoint{gp mark 7}{(6.489,6.046)}
\gppoint{gp mark 7}{(6.571,6.050)}
\gppoint{gp mark 7}{(6.563,6.056)}
\gppoint{gp mark 7}{(6.702,6.583)}
\gppoint{gp mark 7}{(6.703,6.583)}
\gppoint{gp mark 7}{(6.705,6.583)}
\gppoint{gp mark 7}{(6.703,6.583)}
\gppoint{gp mark 7}{(6.825,6.583)}
\gppoint{gp mark 7}{(6.703,6.583)}
\gppoint{gp mark 7}{(6.704,6.583)}
\gppoint{gp mark 7}{(6.705,6.583)}
\gppoint{gp mark 7}{(6.707,6.583)}
\gppoint{gp mark 7}{(6.822,6.583)}
\gppoint{gp mark 7}{(6.703,6.583)}
\gppoint{gp mark 7}{(6.703,6.583)}
\gppoint{gp mark 7}{(6.704,6.583)}
\gppoint{gp mark 7}{(6.703,6.583)}
\gppoint{gp mark 7}{(6.823,6.583)}
\gppoint{gp mark 7}{(6.700,6.583)}
\gppoint{gp mark 7}{(6.704,6.583)}
\gppoint{gp mark 7}{(6.699,6.583)}
\gppoint{gp mark 7}{(6.703,6.583)}
\gppoint{gp mark 7}{(6.822,6.583)}
\gppoint{gp mark 7}{(6.704,6.583)}
\gppoint{gp mark 7}{(6.700,6.583)}
\gppoint{gp mark 7}{(6.703,6.583)}
\gppoint{gp mark 7}{(6.705,6.583)}
\gppoint{gp mark 7}{(6.822,6.583)}
\gppoint{gp mark 7}{(6.700,6.583)}
\gppoint{gp mark 7}{(6.702,6.583)}
\gppoint{gp mark 7}{(6.701,6.583)}
\gppoint{gp mark 7}{(6.705,6.583)}
\gppoint{gp mark 7}{(6.901,6.583)}
\gppoint{gp mark 7}{(6.703,6.583)}
\gppoint{gp mark 7}{(6.702,6.583)}
\gppoint{gp mark 7}{(6.819,6.583)}
\gppoint{gp mark 7}{(6.704,6.583)}
\gppoint{gp mark 7}{(6.705,6.583)}
\gppoint{gp mark 7}{(6.896,6.583)}
\gppoint{gp mark 7}{(6.705,6.583)}
\gppoint{gp mark 7}{(6.702,6.583)}
\gppoint{gp mark 7}{(6.704,6.583)}
\gppoint{gp mark 7}{(6.703,6.583)}
\gppoint{gp mark 7}{(6.895,6.583)}
\gppoint{gp mark 7}{(6.175,5.672)}
\gppoint{gp mark 7}{(6.299,6.583)}
\gppoint{gp mark 7}{(6.462,6.583)}
\gppoint{gp mark 7}{(6.286,4.533)}
\gppoint{gp mark 7}{(6.163,5.079)}
\gppoint{gp mark 7}{(6.167,5.080)}
\gppoint{gp mark 7}{(6.189,5.271)}
\gppoint{gp mark 7}{(6.401,6.583)}
\gppoint{gp mark 7}{(6.422,4.797)}
\gppoint{gp mark 7}{(6.744,5.454)}
\gppoint{gp mark 7}{(6.419,6.583)}
\gppoint{gp mark 7}{(6.308,6.583)}
\gppoint{gp mark 7}{(6.117,5.100)}
\gppoint{gp mark 7}{(6.182,5.497)}
\gppoint{gp mark 7}{(6.222,6.583)}
\gppoint{gp mark 7}{(6.167,5.131)}
\gppoint{gp mark 7}{(6.323,5.507)}
\gppoint{gp mark 7}{(6.399,6.583)}
\gppoint{gp mark 7}{(6.134,4.989)}
\gppoint{gp mark 7}{(6.687,6.583)}
\gppoint{gp mark 7}{(7.440,6.583)}
\gppoint{gp mark 7}{(6.376,4.533)}
\gppoint{gp mark 7}{(6.149,4.533)}
\gppoint{gp mark 7}{(6.144,4.533)}
\gppoint{gp mark 7}{(6.498,4.533)}
\gppoint{gp mark 7}{(6.134,5.221)}
\gppoint{gp mark 7}{(6.305,6.583)}
\gppoint{gp mark 7}{(6.427,6.583)}
\gppoint{gp mark 7}{(6.139,5.153)}
\gppoint{gp mark 7}{(6.117,5.131)}
\gppoint{gp mark 7}{(6.644,6.583)}
\gppoint{gp mark 7}{(6.562,4.780)}
\gppoint{gp mark 7}{(6.423,5.011)}
\gppoint{gp mark 7}{(6.633,6.060)}
\gppoint{gp mark 7}{(6.326,4.533)}
\gppoint{gp mark 7}{(6.182,5.552)}
\gppoint{gp mark 7}{(6.179,4.533)}
\gppoint{gp mark 7}{(6.376,4.812)}
\gppoint{gp mark 7}{(6.653,5.125)}
\gppoint{gp mark 7}{(6.654,5.090)}
\gppoint{gp mark 7}{(6.415,4.780)}
\gppoint{gp mark 7}{(6.317,4.780)}
\gppoint{gp mark 7}{(6.465,4.989)}
\gppoint{gp mark 7}{(6.398,4.877)}
\gppoint{gp mark 7}{(6.424,5.020)}
\gppoint{gp mark 7}{(6.263,4.533)}
\gppoint{gp mark 7}{(6.420,4.947)}
\gppoint{gp mark 7}{(6.419,4.761)}
\gppoint{gp mark 7}{(6.134,5.256)}
\gppoint{gp mark 7}{(6.235,4.761)}
\gppoint{gp mark 7}{(6.329,5.014)}
\gppoint{gp mark 7}{(6.277,4.738)}
\gppoint{gp mark 7}{(6.154,4.533)}
\gppoint{gp mark 7}{(6.251,4.838)}
\gppoint{gp mark 7}{(6.154,4.533)}
\gppoint{gp mark 7}{(6.301,4.877)}
\gppoint{gp mark 7}{(6.655,5.060)}
\gpcolor{color=gp lt color border}
\gpsetlinetype{gp lt border}
\draw[gp path] (5.730,6.583)--(5.730,4.533)--(7.780,4.533)--(7.780,6.583)--cycle;
%% coordinates of the plot area
\gpdefrectangularnode{gp plot 2}{\pgfpoint{5.730cm}{4.533cm}}{\pgfpoint{7.780cm}{6.583cm}}
\draw[gp path] (9.897,4.533)--(10.077,4.533);
\node[gp node right] at (9.713,4.533) {{\supertiny $10^{-2}$}};
\draw[gp path] (9.897,4.826)--(10.077,4.826);
\node[gp node right] at (9.713,4.826) {{\supertiny $10^{-1}$}};
\draw[gp path] (9.897,5.119)--(10.077,5.119);
\node[gp node right] at (9.713,5.119) {{\supertiny $10^{0}$}};
\draw[gp path] (9.897,5.412)--(10.077,5.412);
\node[gp node right] at (9.713,5.412) {{\supertiny $10^{1}$}};
\draw[gp path] (9.897,5.704)--(10.077,5.704);
\node[gp node right] at (9.713,5.704) {{\supertiny $10^{2}$}};
\draw[gp path] (9.897,5.997)--(10.077,5.997);
\node[gp node right] at (9.713,5.997) {{\supertiny $10^{3}$}};
\draw[gp path] (9.897,6.290)--(10.077,6.290);
\node[gp node right] at (9.713,6.290) {{\supertiny $10^{4}$}};
\draw[gp path] (9.897,6.583)--(10.077,6.583);
\node[gp node right] at (9.713,6.583) {{\supertiny $10^{5}$}};
\draw[gp path] (9.897,4.533)--(9.897,4.713);
\node[gp node center] at (9.897,4.225) {{\supertiny $10^{-2}$}};
\draw[gp path] (10.190,4.533)--(10.190,4.713);
\node[gp node center] at (10.190,4.225) {{\supertiny $10^{-1}$}};
\draw[gp path] (10.483,4.533)--(10.483,4.713);
\node[gp node center] at (10.483,4.225) {{\supertiny $10^{0}$}};
\draw[gp path] (10.776,4.533)--(10.776,4.713);
\node[gp node center] at (10.776,4.225) {{\supertiny $10^{1}$}};
\draw[gp path] (11.068,4.533)--(11.068,4.713);
\node[gp node center] at (11.068,4.225) {{\supertiny $10^{2}$}};
\draw[gp path] (11.361,4.533)--(11.361,4.713);
\node[gp node center] at (11.361,4.225) {{\supertiny $10^{3}$}};
\draw[gp path] (11.654,4.533)--(11.654,4.713);
\node[gp node center] at (11.654,4.225) {{\supertiny $10^{4}$}};
\draw[gp path] (11.947,4.533)--(11.947,4.713);
\node[gp node center] at (11.947,4.225) {{\supertiny $10^{5}$}};
\draw[gp path] (9.897,6.583)--(9.897,4.533)--(11.947,4.533)--(11.947,6.583)--cycle;
\node[gp node center,rotate=-270] at (8.823,5.558) {MIST};
\node[gp node center] at (10.922,3.763) {Ref.\ over integers};
\gpcolor{color=gp lt color 0}
\gpsetlinetype{gp lt plot 0}
\draw[gp path] (9.897,4.533)--(9.918,4.554)--(9.938,4.574)--(9.959,4.595)--(9.980,4.616)%
  --(10.001,4.637)--(10.021,4.657)--(10.042,4.678)--(10.063,4.699)--(10.083,4.719)--(10.104,4.740)%
  --(10.125,4.761)--(10.145,4.781)--(10.166,4.802)--(10.187,4.823)--(10.208,4.844)--(10.228,4.864)%
  --(10.249,4.885)--(10.270,4.906)--(10.290,4.926)--(10.311,4.947)--(10.332,4.968)--(10.353,4.989)%
  --(10.373,5.009)--(10.394,5.030)--(10.415,5.051)--(10.435,5.071)--(10.456,5.092)--(10.477,5.113)%
  --(10.498,5.134)--(10.518,5.154)--(10.539,5.175)--(10.560,5.196)--(10.580,5.216)--(10.601,5.237)%
  --(10.622,5.258)--(10.642,5.278)--(10.663,5.299)--(10.684,5.320)--(10.705,5.341)--(10.725,5.361)%
  --(10.746,5.382)--(10.767,5.403)--(10.787,5.423)--(10.808,5.444)--(10.829,5.465)--(10.850,5.486)%
  --(10.870,5.506)--(10.891,5.527)--(10.912,5.548)--(10.932,5.568)--(10.953,5.589)--(10.974,5.610)%
  --(10.994,5.630)--(11.015,5.651)--(11.036,5.672)--(11.057,5.693)--(11.077,5.713)--(11.098,5.734)%
  --(11.119,5.755)--(11.139,5.775)--(11.160,5.796)--(11.181,5.817)--(11.202,5.838)--(11.222,5.858)%
  --(11.243,5.879)--(11.264,5.900)--(11.284,5.920)--(11.305,5.941)--(11.326,5.962)--(11.346,5.982)%
  --(11.367,6.003)--(11.388,6.024)--(11.409,6.045)--(11.429,6.065)--(11.450,6.086)--(11.471,6.107)%
  --(11.491,6.127)--(11.512,6.148)--(11.533,6.169)--(11.554,6.190)--(11.574,6.210)--(11.595,6.231)%
  --(11.616,6.252)--(11.636,6.272)--(11.657,6.293)--(11.678,6.314)--(11.699,6.335)--(11.719,6.355)%
  --(11.740,6.376)--(11.761,6.397)--(11.781,6.417)--(11.802,6.438)--(11.823,6.459)--(11.843,6.479)%
  --(11.864,6.500)--(11.885,6.521)--(11.906,6.542)--(11.926,6.562);
\gpcolor{color=gp lt color 2}
\gppoint{gp mark 7}{(10.257,5.543)}
\gppoint{gp mark 7}{(10.867,5.439)}
\gppoint{gp mark 7}{(10.284,6.054)}
\gppoint{gp mark 7}{(10.241,4.533)}
\gppoint{gp mark 7}{(10.301,6.583)}
\gppoint{gp mark 7}{(10.306,5.497)}
\gppoint{gp mark 7}{(10.250,4.826)}
\gppoint{gp mark 7}{(10.296,6.583)}
\gppoint{gp mark 7}{(10.301,4.738)}
\gppoint{gp mark 7}{(10.257,4.761)}
\gppoint{gp mark 7}{(10.265,4.780)}
\gppoint{gp mark 7}{(10.213,6.583)}
\gppoint{gp mark 7}{(10.306,6.536)}
\gppoint{gp mark 7}{(10.311,5.154)}
\gppoint{gp mark 7}{(10.250,4.812)}
\gppoint{gp mark 7}{(10.375,4.812)}
\gppoint{gp mark 7}{(10.437,4.709)}
\gppoint{gp mark 7}{(10.241,5.036)}
\gppoint{gp mark 7}{(10.660,5.390)}
\gppoint{gp mark 7}{(10.729,6.045)}
\gppoint{gp mark 7}{(10.731,6.583)}
\gppoint{gp mark 7}{(10.653,5.388)}
\gppoint{gp mark 7}{(10.731,5.516)}
\gppoint{gp mark 7}{(10.727,5.532)}
\gppoint{gp mark 7}{(10.655,5.388)}
\gppoint{gp mark 7}{(10.737,5.508)}
\gppoint{gp mark 7}{(10.728,6.583)}
\gppoint{gp mark 7}{(10.656,5.388)}
\gppoint{gp mark 7}{(10.738,5.509)}
\gppoint{gp mark 7}{(10.730,5.538)}
\gppoint{gp mark 7}{(10.869,6.583)}
\gppoint{gp mark 7}{(10.870,6.583)}
\gppoint{gp mark 7}{(10.872,6.583)}
\gppoint{gp mark 7}{(10.870,6.583)}
\gppoint{gp mark 7}{(10.992,6.583)}
\gppoint{gp mark 7}{(10.870,6.583)}
\gppoint{gp mark 7}{(10.871,6.583)}
\gppoint{gp mark 7}{(10.872,6.583)}
\gppoint{gp mark 7}{(10.874,6.583)}
\gppoint{gp mark 7}{(10.989,6.583)}
\gppoint{gp mark 7}{(10.870,6.583)}
\gppoint{gp mark 7}{(10.870,6.583)}
\gppoint{gp mark 7}{(10.871,6.583)}
\gppoint{gp mark 7}{(10.870,6.583)}
\gppoint{gp mark 7}{(10.990,6.583)}
\gppoint{gp mark 7}{(10.867,6.583)}
\gppoint{gp mark 7}{(10.871,6.583)}
\gppoint{gp mark 7}{(10.866,6.583)}
\gppoint{gp mark 7}{(10.870,6.583)}
\gppoint{gp mark 7}{(10.989,6.583)}
\gppoint{gp mark 7}{(10.871,6.583)}
\gppoint{gp mark 7}{(10.867,6.583)}
\gppoint{gp mark 7}{(10.870,6.583)}
\gppoint{gp mark 7}{(10.872,6.583)}
\gppoint{gp mark 7}{(10.989,6.583)}
\gppoint{gp mark 7}{(10.867,6.583)}
\gppoint{gp mark 7}{(10.869,6.583)}
\gppoint{gp mark 7}{(10.868,6.583)}
\gppoint{gp mark 7}{(10.872,6.583)}
\gppoint{gp mark 7}{(11.068,6.583)}
\gppoint{gp mark 7}{(10.870,6.583)}
\gppoint{gp mark 7}{(10.869,6.583)}
\gppoint{gp mark 7}{(10.986,6.583)}
\gppoint{gp mark 7}{(10.871,6.583)}
\gppoint{gp mark 7}{(10.872,6.583)}
\gppoint{gp mark 7}{(11.063,6.583)}
\gppoint{gp mark 7}{(10.872,6.583)}
\gppoint{gp mark 7}{(10.869,6.583)}
\gppoint{gp mark 7}{(10.871,6.583)}
\gppoint{gp mark 7}{(10.870,6.583)}
\gppoint{gp mark 7}{(11.062,6.583)}
\gppoint{gp mark 7}{(10.342,6.583)}
\gppoint{gp mark 7}{(10.466,6.583)}
\gppoint{gp mark 7}{(10.453,6.583)}
\gppoint{gp mark 7}{(10.330,6.583)}
\gppoint{gp mark 7}{(10.334,6.583)}
\gppoint{gp mark 7}{(10.451,6.583)}
\gppoint{gp mark 7}{(10.356,4.957)}
\gppoint{gp mark 7}{(10.568,6.583)}
\gppoint{gp mark 7}{(10.439,6.583)}
\gppoint{gp mark 7}{(10.911,6.583)}
\gppoint{gp mark 7}{(10.190,6.583)}
\gppoint{gp mark 7}{(10.265,6.583)}
\gppoint{gp mark 7}{(10.250,6.583)}
\gppoint{gp mark 7}{(10.271,5.972)}
\gppoint{gp mark 7}{(10.586,6.583)}
\gppoint{gp mark 7}{(10.475,6.583)}
\gppoint{gp mark 7}{(10.284,6.583)}
\gppoint{gp mark 7}{(10.349,6.583)}
\gppoint{gp mark 7}{(10.389,6.583)}
\gppoint{gp mark 7}{(10.334,6.583)}
\gppoint{gp mark 7}{(10.301,6.583)}
\gppoint{gp mark 7}{(10.506,6.583)}
\gppoint{gp mark 7}{(10.499,6.583)}
\gppoint{gp mark 7}{(10.543,6.583)}
\gppoint{gp mark 7}{(10.541,6.583)}
\gppoint{gp mark 7}{(10.316,6.583)}
\gppoint{gp mark 7}{(10.311,6.583)}
\gppoint{gp mark 7}{(10.669,6.583)}
\gppoint{gp mark 7}{(10.665,6.583)}
\gppoint{gp mark 7}{(10.428,6.583)}
\gppoint{gp mark 7}{(10.301,6.583)}
\gppoint{gp mark 7}{(10.472,6.583)}
\gppoint{gp mark 7}{(10.594,6.583)}
\gppoint{gp mark 7}{(10.306,6.583)}
\gppoint{gp mark 7}{(10.284,6.583)}
\gppoint{gp mark 7}{(10.311,6.583)}
\gppoint{gp mark 7}{(10.301,6.583)}
\gppoint{gp mark 7}{(10.404,5.040)}
\gppoint{gp mark 7}{(10.400,5.080)}
\gppoint{gp mark 7}{(10.407,5.084)}
\gppoint{gp mark 7}{(10.311,5.085)}
\gppoint{gp mark 7}{(10.729,6.583)}
\gppoint{gp mark 7}{(10.422,5.265)}
\gppoint{gp mark 7}{(10.590,5.923)}
\gppoint{gp mark 7}{(10.800,6.583)}
\gppoint{gp mark 7}{(10.346,5.070)}
\gppoint{gp mark 7}{(10.493,5.557)}
\gppoint{gp mark 7}{(10.321,5.154)}
\gppoint{gp mark 7}{(10.349,5.200)}
\gppoint{gp mark 7}{(10.296,4.780)}
\gppoint{gp mark 7}{(10.439,5.467)}
\gppoint{gp mark 7}{(10.346,6.399)}
\gppoint{gp mark 7}{(10.543,6.583)}
\gppoint{gp mark 7}{(10.522,6.144)}
\gppoint{gp mark 7}{(10.820,6.583)}
\gppoint{gp mark 7}{(10.821,6.583)}
\gppoint{gp mark 7}{(10.465,6.300)}
\gppoint{gp mark 7}{(10.582,6.583)}
\gppoint{gp mark 7}{(10.338,5.496)}
\gppoint{gp mark 7}{(10.369,5.968)}
\gppoint{gp mark 7}{(10.311,6.337)}
\gppoint{gp mark 7}{(10.428,6.583)}
\gppoint{gp mark 7}{(10.430,6.583)}
\gppoint{gp mark 7}{(10.587,6.583)}
\gppoint{gp mark 7}{(10.586,6.583)}
\gppoint{gp mark 7}{(10.301,4.999)}
\gppoint{gp mark 7}{(10.306,5.161)}
\gppoint{gp mark 7}{(10.402,5.740)}
\gppoint{gp mark 7}{(10.316,5.209)}
\gppoint{gp mark 7}{(10.306,5.109)}
\gppoint{gp mark 7}{(10.464,5.755)}
\gppoint{gp mark 7}{(10.468,6.583)}
\gppoint{gp mark 7}{(10.378,6.512)}
\gppoint{gp mark 7}{(10.822,6.583)}
\gpcolor{color=gp lt color border}
\gpsetlinetype{gp lt border}
\draw[gp path] (9.897,6.583)--(9.897,4.533)--(11.947,4.533)--(11.947,6.583)--cycle;
%% coordinates of the plot area
\gpdefrectangularnode{gp plot 3}{\pgfpoint{9.897cm}{4.533cm}}{\pgfpoint{11.947cm}{6.583cm}}
\draw[gp path] (1.564,1.033)--(1.744,1.033);
\node[gp node right] at (1.380,1.033) {{\supertiny $10^{-2}$}};
\draw[gp path] (1.564,1.326)--(1.744,1.326);
\node[gp node right] at (1.380,1.326) {{\supertiny $10^{-1}$}};
\draw[gp path] (1.564,1.619)--(1.744,1.619);
\node[gp node right] at (1.380,1.619) {{\supertiny $10^{0}$}};
\draw[gp path] (1.564,1.912)--(1.744,1.912);
\node[gp node right] at (1.380,1.912) {{\supertiny $10^{1}$}};
\draw[gp path] (1.564,2.205)--(1.744,2.205);
\node[gp node right] at (1.380,2.205) {{\supertiny $10^{2}$}};
\draw[gp path] (1.564,2.498)--(1.744,2.498);
\node[gp node right] at (1.380,2.498) {{\supertiny $10^{3}$}};
\draw[gp path] (1.564,2.791)--(1.744,2.791);
\node[gp node right] at (1.380,2.791) {{\supertiny $10^{4}$}};
\draw[gp path] (1.564,3.084)--(1.744,3.084);
\node[gp node right] at (1.380,3.084) {{\supertiny $10^{5}$}};
\draw[gp path] (1.564,1.033)--(1.564,1.213);
\node[gp node center] at (1.564,0.725) {{\supertiny $10^{-2}$}};
\draw[gp path] (1.857,1.033)--(1.857,1.213);
\node[gp node center] at (1.857,0.725) {{\supertiny $10^{-1}$}};
\draw[gp path] (2.150,1.033)--(2.150,1.213);
\node[gp node center] at (2.150,0.725) {{\supertiny $10^{0}$}};
\draw[gp path] (2.443,1.033)--(2.443,1.213);
\node[gp node center] at (2.443,0.725) {{\supertiny $10^{1}$}};
\draw[gp path] (2.735,1.033)--(2.735,1.213);
\node[gp node center] at (2.735,0.725) {{\supertiny $10^{2}$}};
\draw[gp path] (3.028,1.033)--(3.028,1.213);
\node[gp node center] at (3.028,0.725) {{\supertiny $10^{3}$}};
\draw[gp path] (3.321,1.033)--(3.321,1.213);
\node[gp node center] at (3.321,0.725) {{\supertiny $10^{4}$}};
\draw[gp path] (3.614,1.033)--(3.614,1.213);
\node[gp node center] at (3.614,0.725) {{\supertiny $10^{5}$}};
\draw[gp path] (1.564,3.084)--(1.564,1.033)--(3.614,1.033)--(3.614,3.084)--cycle;
\node[gp node center,rotate=-270] at (0.490,2.058) {IIC};
\node[gp node center] at (2.589,0.263) {Ref.\ w/ invariants};
\gpcolor{color=gp lt color 0}
\gpsetlinetype{gp lt plot 0}
\draw[gp path] (1.564,1.033)--(1.585,1.054)--(1.605,1.074)--(1.626,1.095)--(1.647,1.116)%
  --(1.668,1.137)--(1.688,1.157)--(1.709,1.178)--(1.730,1.199)--(1.750,1.219)--(1.771,1.240)%
  --(1.792,1.261)--(1.812,1.282)--(1.833,1.302)--(1.854,1.323)--(1.875,1.344)--(1.895,1.364)%
  --(1.916,1.385)--(1.937,1.406)--(1.957,1.427)--(1.978,1.447)--(1.999,1.468)--(2.020,1.489)%
  --(2.040,1.509)--(2.061,1.530)--(2.082,1.551)--(2.102,1.572)--(2.123,1.592)--(2.144,1.613)%
  --(2.165,1.634)--(2.185,1.655)--(2.206,1.675)--(2.227,1.696)--(2.247,1.717)--(2.268,1.737)%
  --(2.289,1.758)--(2.309,1.779)--(2.330,1.800)--(2.351,1.820)--(2.372,1.841)--(2.392,1.862)%
  --(2.413,1.882)--(2.434,1.903)--(2.454,1.924)--(2.475,1.945)--(2.496,1.965)--(2.517,1.986)%
  --(2.537,2.007)--(2.558,2.027)--(2.579,2.048)--(2.599,2.069)--(2.620,2.090)--(2.641,2.110)%
  --(2.661,2.131)--(2.682,2.152)--(2.703,2.172)--(2.724,2.193)--(2.744,2.214)--(2.765,2.235)%
  --(2.786,2.255)--(2.806,2.276)--(2.827,2.297)--(2.848,2.317)--(2.869,2.338)--(2.889,2.359)%
  --(2.910,2.380)--(2.931,2.400)--(2.951,2.421)--(2.972,2.442)--(2.993,2.462)--(3.013,2.483)%
  --(3.034,2.504)--(3.055,2.525)--(3.076,2.545)--(3.096,2.566)--(3.117,2.587)--(3.138,2.608)%
  --(3.158,2.628)--(3.179,2.649)--(3.200,2.670)--(3.221,2.690)--(3.241,2.711)--(3.262,2.732)%
  --(3.283,2.753)--(3.303,2.773)--(3.324,2.794)--(3.345,2.815)--(3.366,2.835)--(3.386,2.856)%
  --(3.407,2.877)--(3.428,2.898)--(3.448,2.918)--(3.469,2.939)--(3.490,2.960)--(3.510,2.980)%
  --(3.531,3.001)--(3.552,3.022)--(3.573,3.043)--(3.593,3.063);
\gpcolor{color=gp lt color 2}
\gppoint{gp mark 7}{(2.152,2.569)}
\gppoint{gp mark 7}{(2.076,1.457)}
\gppoint{gp mark 7}{(2.188,2.977)}
\gppoint{gp mark 7}{(3.326,2.985)}
\gppoint{gp mark 7}{(2.093,1.793)}
\gppoint{gp mark 7}{(1.968,1.448)}
\gppoint{gp mark 7}{(2.071,1.512)}
\gppoint{gp mark 7}{(1.973,1.575)}
\gppoint{gp mark 7}{(1.978,1.478)}
\gppoint{gp mark 7}{(2.067,1.033)}
\gppoint{gp mark 7}{(2.447,1.526)}
\gppoint{gp mark 7}{(2.377,3.084)}
\gppoint{gp mark 7}{(2.384,3.084)}
\gppoint{gp mark 7}{(2.437,1.509)}
\gppoint{gp mark 7}{(2.382,1.787)}
\gppoint{gp mark 7}{(2.381,2.190)}
\gppoint{gp mark 7}{(2.448,1.515)}
\gppoint{gp mark 7}{(2.381,1.786)}
\gppoint{gp mark 7}{(2.380,3.084)}
\gppoint{gp mark 7}{(2.438,1.512)}
\gppoint{gp mark 7}{(2.382,1.787)}
\gppoint{gp mark 7}{(2.381,2.189)}
\gppoint{gp mark 7}{(2.650,3.084)}
\gppoint{gp mark 7}{(2.648,3.084)}
\gppoint{gp mark 7}{(2.647,3.084)}
\gppoint{gp mark 7}{(2.647,3.084)}
\gppoint{gp mark 7}{(2.580,3.084)}
\gppoint{gp mark 7}{(2.647,3.084)}
\gppoint{gp mark 7}{(2.648,3.084)}
\gppoint{gp mark 7}{(2.647,3.084)}
\gppoint{gp mark 7}{(2.647,3.084)}
\gppoint{gp mark 7}{(2.684,3.084)}
\gppoint{gp mark 7}{(2.648,3.084)}
\gppoint{gp mark 7}{(2.648,3.084)}
\gppoint{gp mark 7}{(2.649,3.084)}
\gppoint{gp mark 7}{(2.648,3.084)}
\gppoint{gp mark 7}{(2.580,3.084)}
\gppoint{gp mark 7}{(2.648,3.084)}
\gppoint{gp mark 7}{(2.648,3.084)}
\gppoint{gp mark 7}{(2.649,3.084)}
\gppoint{gp mark 7}{(2.647,3.084)}
\gppoint{gp mark 7}{(2.580,3.084)}
\gppoint{gp mark 7}{(2.647,3.084)}
\gppoint{gp mark 7}{(2.650,3.084)}
\gppoint{gp mark 7}{(2.647,3.084)}
\gppoint{gp mark 7}{(2.648,3.084)}
\gppoint{gp mark 7}{(2.578,3.084)}
\gppoint{gp mark 7}{(2.648,3.084)}
\gppoint{gp mark 7}{(2.648,3.084)}
\gppoint{gp mark 7}{(2.647,3.084)}
\gppoint{gp mark 7}{(2.647,3.084)}
\gppoint{gp mark 7}{(2.579,3.084)}
\gppoint{gp mark 7}{(2.649,3.084)}
\gppoint{gp mark 7}{(2.649,3.084)}
\gppoint{gp mark 7}{(2.580,3.084)}
\gppoint{gp mark 7}{(2.648,3.084)}
\gppoint{gp mark 7}{(2.648,3.084)}
\gppoint{gp mark 7}{(2.579,3.084)}
\gppoint{gp mark 7}{(2.649,3.084)}
\gppoint{gp mark 7}{(2.646,3.084)}
\gppoint{gp mark 7}{(2.645,3.084)}
\gppoint{gp mark 7}{(2.647,3.084)}
\gppoint{gp mark 7}{(2.579,3.084)}
\gppoint{gp mark 7}{(2.151,3.084)}
\gppoint{gp mark 7}{(2.264,3.084)}
\gppoint{gp mark 7}{(2.425,3.084)}
\gppoint{gp mark 7}{(2.123,1.982)}
\gppoint{gp mark 7}{(2.140,3.084)}
\gppoint{gp mark 7}{(2.136,3.084)}
\gppoint{gp mark 7}{(2.124,1.533)}
\gppoint{gp mark 7}{(2.166,1.173)}
\gppoint{gp mark 7}{(2.357,1.676)}
\gppoint{gp mark 7}{(2.101,1.586)}
\gppoint{gp mark 7}{(2.257,3.084)}
\gppoint{gp mark 7}{(2.474,3.084)}
\gppoint{gp mark 7}{(2.080,1.386)}
\gppoint{gp mark 7}{(2.083,1.386)}
\gppoint{gp mark 7}{(2.076,1.378)}
\gppoint{gp mark 7}{(2.087,1.338)}
\gppoint{gp mark 7}{(2.367,3.084)}
\gppoint{gp mark 7}{(2.266,3.084)}
\gppoint{gp mark 7}{(2.121,3.084)}
\gppoint{gp mark 7}{(2.155,3.084)}
\gppoint{gp mark 7}{(2.189,3.084)}
\gppoint{gp mark 7}{(2.145,3.084)}
\gppoint{gp mark 7}{(2.293,3.084)}
\gppoint{gp mark 7}{(2.380,3.084)}
\gppoint{gp mark 7}{(2.118,3.038)}
\gppoint{gp mark 7}{(2.619,3.084)}
\gppoint{gp mark 7}{(2.249,3.084)}
\gppoint{gp mark 7}{(2.251,3.084)}
\gppoint{gp mark 7}{(2.249,2.240)}
\gppoint{gp mark 7}{(2.249,2.246)}
\gppoint{gp mark 7}{(2.120,2.158)}
\gppoint{gp mark 7}{(2.128,2.172)}
\gppoint{gp mark 7}{(2.251,2.338)}
\gppoint{gp mark 7}{(2.248,2.353)}
\gppoint{gp mark 7}{(2.074,1.492)}
\gppoint{gp mark 7}{(2.101,2.786)}
\gppoint{gp mark 7}{(2.254,3.084)}
\gppoint{gp mark 7}{(2.366,3.084)}
\gppoint{gp mark 7}{(2.116,2.887)}
\gppoint{gp mark 7}{(2.116,3.084)}
\gppoint{gp mark 7}{(2.126,2.348)}
\gppoint{gp mark 7}{(2.124,2.041)}
\gppoint{gp mark 7}{(2.585,3.084)}
\gppoint{gp mark 7}{(2.080,1.209)}
\gppoint{gp mark 7}{(2.087,1.173)}
\gppoint{gp mark 7}{(2.087,1.209)}
\gppoint{gp mark 7}{(2.390,3.084)}
\gppoint{gp mark 7}{(2.172,1.603)}
\gppoint{gp mark 7}{(2.452,2.489)}
\gppoint{gp mark 7}{(2.156,1.443)}
\gppoint{gp mark 7}{(2.155,1.209)}
\gppoint{gp mark 7}{(2.106,1.298)}
\gppoint{gp mark 7}{(2.013,1.173)}
\gppoint{gp mark 7}{(2.209,3.084)}
\gppoint{gp mark 7}{(2.143,1.313)}
\gppoint{gp mark 7}{(2.394,3.084)}
\gppoint{gp mark 7}{(2.399,3.084)}
\gppoint{gp mark 7}{(2.045,1.349)}
\gppoint{gp mark 7}{(2.250,3.084)}
\gppoint{gp mark 7}{(2.233,3.084)}
\gppoint{gp mark 7}{(2.315,3.084)}
\gppoint{gp mark 7}{(2.039,1.432)}
\gppoint{gp mark 7}{(2.233,3.084)}
\gppoint{gp mark 7}{(1.988,1.033)}
\gppoint{gp mark 7}{(2.258,2.754)}
\gppoint{gp mark 7}{(2.091,1.575)}
\gppoint{gp mark 7}{(2.097,1.575)}
\gppoint{gp mark 7}{(2.253,2.621)}
\gppoint{gp mark 7}{(2.252,3.084)}
\gppoint{gp mark 7}{(2.074,1.209)}
\gppoint{gp mark 7}{(2.158,1.649)}
\gppoint{gp mark 7}{(2.113,1.960)}
\gppoint{gp mark 7}{(1.973,1.033)}
\gppoint{gp mark 7}{(2.039,1.474)}
\gppoint{gp mark 7}{(1.997,1.033)}
\gppoint{gp mark 7}{(2.033,1.338)}
\gppoint{gp mark 7}{(2.089,1.921)}
\gppoint{gp mark 7}{(1.988,1.033)}
\gppoint{gp mark 7}{(2.133,2.092)}
\gppoint{gp mark 7}{(2.051,1.313)}
\gppoint{gp mark 7}{(2.402,3.084)}
\gpcolor{color=gp lt color border}
\gpsetlinetype{gp lt border}
\draw[gp path] (1.564,3.084)--(1.564,1.033)--(3.614,1.033)--(3.614,3.084)--cycle;
%% coordinates of the plot area
\gpdefrectangularnode{gp plot 4}{\pgfpoint{1.564cm}{1.033cm}}{\pgfpoint{3.614cm}{3.084cm}}
\draw[gp path] (5.730,1.033)--(5.910,1.033);
\node[gp node right] at (5.546,1.033) {{\supertiny $10^{-2}$}};
\draw[gp path] (5.730,1.326)--(5.910,1.326);
\node[gp node right] at (5.546,1.326) {{\supertiny $10^{-1}$}};
\draw[gp path] (5.730,1.619)--(5.910,1.619);
\node[gp node right] at (5.546,1.619) {{\supertiny $10^{0}$}};
\draw[gp path] (5.730,1.912)--(5.910,1.912);
\node[gp node right] at (5.546,1.912) {{\supertiny $10^{1}$}};
\draw[gp path] (5.730,2.205)--(5.910,2.205);
\node[gp node right] at (5.546,2.205) {{\supertiny $10^{2}$}};
\draw[gp path] (5.730,2.498)--(5.910,2.498);
\node[gp node right] at (5.546,2.498) {{\supertiny $10^{3}$}};
\draw[gp path] (5.730,2.791)--(5.910,2.791);
\node[gp node right] at (5.546,2.791) {{\supertiny $10^{4}$}};
\draw[gp path] (5.730,3.084)--(5.910,3.084);
\node[gp node right] at (5.546,3.084) {{\supertiny $10^{5}$}};
\draw[gp path] (5.730,1.033)--(5.730,1.213);
\node[gp node center] at (5.730,0.725) {{\supertiny $10^{-2}$}};
\draw[gp path] (6.023,1.033)--(6.023,1.213);
\node[gp node center] at (6.023,0.725) {{\supertiny $10^{-1}$}};
\draw[gp path] (6.316,1.033)--(6.316,1.213);
\node[gp node center] at (6.316,0.725) {{\supertiny $10^{0}$}};
\draw[gp path] (6.609,1.033)--(6.609,1.213);
\node[gp node center] at (6.609,0.725) {{\supertiny $10^{1}$}};
\draw[gp path] (6.901,1.033)--(6.901,1.213);
\node[gp node center] at (6.901,0.725) {{\supertiny $10^{2}$}};
\draw[gp path] (7.194,1.033)--(7.194,1.213);
\node[gp node center] at (7.194,0.725) {{\supertiny $10^{3}$}};
\draw[gp path] (7.487,1.033)--(7.487,1.213);
\node[gp node center] at (7.487,0.725) {{\supertiny $10^{4}$}};
\draw[gp path] (7.780,1.033)--(7.780,1.213);
\node[gp node center] at (7.780,0.725) {{\supertiny $10^{5}$}};
\draw[gp path] (5.730,3.084)--(5.730,1.033)--(7.780,1.033)--(7.780,3.084)--cycle;
\node[gp node center,rotate=-270] at (4.656,2.058) {BFC};
\node[gp node center] at (6.755,0.263) {Ref.\ w/ invariants};
\gpcolor{color=gp lt color 0}
\gpsetlinetype{gp lt plot 0}
\draw[gp path] (5.730,1.033)--(5.751,1.054)--(5.771,1.074)--(5.792,1.095)--(5.813,1.116)%
  --(5.834,1.137)--(5.854,1.157)--(5.875,1.178)--(5.896,1.199)--(5.916,1.219)--(5.937,1.240)%
  --(5.958,1.261)--(5.978,1.282)--(5.999,1.302)--(6.020,1.323)--(6.041,1.344)--(6.061,1.364)%
  --(6.082,1.385)--(6.103,1.406)--(6.123,1.427)--(6.144,1.447)--(6.165,1.468)--(6.186,1.489)%
  --(6.206,1.509)--(6.227,1.530)--(6.248,1.551)--(6.268,1.572)--(6.289,1.592)--(6.310,1.613)%
  --(6.331,1.634)--(6.351,1.655)--(6.372,1.675)--(6.393,1.696)--(6.413,1.717)--(6.434,1.737)%
  --(6.455,1.758)--(6.475,1.779)--(6.496,1.800)--(6.517,1.820)--(6.538,1.841)--(6.558,1.862)%
  --(6.579,1.882)--(6.600,1.903)--(6.620,1.924)--(6.641,1.945)--(6.662,1.965)--(6.683,1.986)%
  --(6.703,2.007)--(6.724,2.027)--(6.745,2.048)--(6.765,2.069)--(6.786,2.090)--(6.807,2.110)%
  --(6.827,2.131)--(6.848,2.152)--(6.869,2.172)--(6.890,2.193)--(6.910,2.214)--(6.931,2.235)%
  --(6.952,2.255)--(6.972,2.276)--(6.993,2.297)--(7.014,2.317)--(7.035,2.338)--(7.055,2.359)%
  --(7.076,2.380)--(7.097,2.400)--(7.117,2.421)--(7.138,2.442)--(7.159,2.462)--(7.179,2.483)%
  --(7.200,2.504)--(7.221,2.525)--(7.242,2.545)--(7.262,2.566)--(7.283,2.587)--(7.304,2.608)%
  --(7.324,2.628)--(7.345,2.649)--(7.366,2.670)--(7.387,2.690)--(7.407,2.711)--(7.428,2.732)%
  --(7.449,2.753)--(7.469,2.773)--(7.490,2.794)--(7.511,2.815)--(7.532,2.835)--(7.552,2.856)%
  --(7.573,2.877)--(7.594,2.898)--(7.614,2.918)--(7.635,2.939)--(7.656,2.960)--(7.676,2.980)%
  --(7.697,3.001)--(7.718,3.022)--(7.739,3.043)--(7.759,3.063);
\gpcolor{color=gp lt color 2}
\gppoint{gp mark 7}{(6.318,1.825)}
\gppoint{gp mark 7}{(6.242,1.033)}
\gppoint{gp mark 7}{(6.354,2.158)}
\gppoint{gp mark 7}{(7.492,2.168)}
\gppoint{gp mark 7}{(6.259,1.349)}
\gppoint{gp mark 7}{(6.134,3.084)}
\gppoint{gp mark 7}{(6.134,3.084)}
\gppoint{gp mark 7}{(6.237,1.033)}
\gppoint{gp mark 7}{(6.290,1.813)}
\gppoint{gp mark 7}{(6.129,1.570)}
\gppoint{gp mark 7}{(6.242,1.572)}
\gppoint{gp mark 7}{(6.246,2.062)}
\gppoint{gp mark 7}{(6.237,1.837)}
\gppoint{gp mark 7}{(6.139,1.748)}
\gppoint{gp mark 7}{(6.613,2.553)}
\gppoint{gp mark 7}{(6.543,2.550)}
\gppoint{gp mark 7}{(6.550,2.549)}
\gppoint{gp mark 7}{(6.603,2.556)}
\gppoint{gp mark 7}{(6.548,2.571)}
\gppoint{gp mark 7}{(6.547,2.561)}
\gppoint{gp mark 7}{(6.614,2.561)}
\gppoint{gp mark 7}{(6.547,2.557)}
\gppoint{gp mark 7}{(6.546,2.573)}
\gppoint{gp mark 7}{(6.604,2.546)}
\gppoint{gp mark 7}{(6.548,2.551)}
\gppoint{gp mark 7}{(6.547,2.557)}
\gppoint{gp mark 7}{(6.816,3.084)}
\gppoint{gp mark 7}{(6.814,3.084)}
\gppoint{gp mark 7}{(6.813,3.084)}
\gppoint{gp mark 7}{(6.813,3.084)}
\gppoint{gp mark 7}{(6.746,3.084)}
\gppoint{gp mark 7}{(6.813,3.084)}
\gppoint{gp mark 7}{(6.814,3.084)}
\gppoint{gp mark 7}{(6.813,3.084)}
\gppoint{gp mark 7}{(6.813,3.084)}
\gppoint{gp mark 7}{(6.850,3.084)}
\gppoint{gp mark 7}{(6.814,3.084)}
\gppoint{gp mark 7}{(6.814,3.084)}
\gppoint{gp mark 7}{(6.815,3.084)}
\gppoint{gp mark 7}{(6.814,3.084)}
\gppoint{gp mark 7}{(6.746,3.084)}
\gppoint{gp mark 7}{(6.814,3.084)}
\gppoint{gp mark 7}{(6.814,3.084)}
\gppoint{gp mark 7}{(6.815,3.084)}
\gppoint{gp mark 7}{(6.813,3.084)}
\gppoint{gp mark 7}{(6.746,3.084)}
\gppoint{gp mark 7}{(6.813,3.084)}
\gppoint{gp mark 7}{(6.816,3.084)}
\gppoint{gp mark 7}{(6.813,3.084)}
\gppoint{gp mark 7}{(6.814,3.084)}
\gppoint{gp mark 7}{(6.744,3.084)}
\gppoint{gp mark 7}{(6.814,3.084)}
\gppoint{gp mark 7}{(6.814,3.084)}
\gppoint{gp mark 7}{(6.813,3.084)}
\gppoint{gp mark 7}{(6.813,3.084)}
\gppoint{gp mark 7}{(6.745,3.084)}
\gppoint{gp mark 7}{(6.815,3.084)}
\gppoint{gp mark 7}{(6.815,3.084)}
\gppoint{gp mark 7}{(6.746,3.084)}
\gppoint{gp mark 7}{(6.814,3.084)}
\gppoint{gp mark 7}{(6.814,3.084)}
\gppoint{gp mark 7}{(6.745,3.084)}
\gppoint{gp mark 7}{(6.815,3.084)}
\gppoint{gp mark 7}{(6.812,3.084)}
\gppoint{gp mark 7}{(6.811,3.084)}
\gppoint{gp mark 7}{(6.813,3.084)}
\gppoint{gp mark 7}{(6.745,3.084)}
\gppoint{gp mark 7}{(6.317,2.173)}
\gppoint{gp mark 7}{(6.430,3.084)}
\gppoint{gp mark 7}{(6.591,3.084)}
\gppoint{gp mark 7}{(6.289,1.033)}
\gppoint{gp mark 7}{(6.306,1.579)}
\gppoint{gp mark 7}{(6.302,1.581)}
\gppoint{gp mark 7}{(6.332,1.771)}
\gppoint{gp mark 7}{(6.523,3.084)}
\gppoint{gp mark 7}{(6.423,1.298)}
\gppoint{gp mark 7}{(6.640,1.954)}
\gppoint{gp mark 7}{(6.533,3.084)}
\gppoint{gp mark 7}{(6.432,3.084)}
\gppoint{gp mark 7}{(6.287,1.600)}
\gppoint{gp mark 7}{(6.321,1.998)}
\gppoint{gp mark 7}{(6.355,3.084)}
\gppoint{gp mark 7}{(6.311,1.631)}
\gppoint{gp mark 7}{(6.459,2.007)}
\gppoint{gp mark 7}{(6.546,3.084)}
\gppoint{gp mark 7}{(6.284,1.489)}
\gppoint{gp mark 7}{(6.785,3.084)}
\gppoint{gp mark 7}{(7.528,3.084)}
\gppoint{gp mark 7}{(6.415,1.033)}
\gppoint{gp mark 7}{(6.286,1.033)}
\gppoint{gp mark 7}{(6.294,1.033)}
\gppoint{gp mark 7}{(6.414,1.033)}
\gppoint{gp mark 7}{(6.267,1.722)}
\gppoint{gp mark 7}{(6.420,3.084)}
\gppoint{gp mark 7}{(6.532,3.084)}
\gppoint{gp mark 7}{(6.282,1.653)}
\gppoint{gp mark 7}{(6.282,1.631)}
\gppoint{gp mark 7}{(6.751,3.084)}
\gppoint{gp mark 7}{(6.556,1.281)}
\gppoint{gp mark 7}{(6.338,1.512)}
\gppoint{gp mark 7}{(6.618,2.561)}
\gppoint{gp mark 7}{(6.322,1.033)}
\gppoint{gp mark 7}{(6.321,2.053)}
\gppoint{gp mark 7}{(6.179,1.033)}
\gppoint{gp mark 7}{(6.375,1.313)}
\gppoint{gp mark 7}{(6.560,1.625)}
\gppoint{gp mark 7}{(6.565,1.591)}
\gppoint{gp mark 7}{(6.416,1.281)}
\gppoint{gp mark 7}{(6.399,1.281)}
\gppoint{gp mark 7}{(6.481,1.489)}
\gppoint{gp mark 7}{(6.399,1.378)}
\gppoint{gp mark 7}{(6.424,1.520)}
\gppoint{gp mark 7}{(6.263,1.033)}
\gppoint{gp mark 7}{(6.419,1.448)}
\gppoint{gp mark 7}{(6.418,1.261)}
\gppoint{gp mark 7}{(6.282,1.756)}
\gppoint{gp mark 7}{(6.240,1.261)}
\gppoint{gp mark 7}{(6.324,1.515)}
\gppoint{gp mark 7}{(6.279,1.238)}
\gppoint{gp mark 7}{(6.163,1.033)}
\gppoint{gp mark 7}{(6.255,1.338)}
\gppoint{gp mark 7}{(6.154,1.033)}
\gppoint{gp mark 7}{(6.299,1.378)}
\gppoint{gp mark 7}{(6.568,1.560)}
\gpcolor{color=gp lt color border}
\gpsetlinetype{gp lt border}
\draw[gp path] (5.730,3.084)--(5.730,1.033)--(7.780,1.033)--(7.780,3.084)--cycle;
%% coordinates of the plot area
\gpdefrectangularnode{gp plot 5}{\pgfpoint{5.730cm}{1.033cm}}{\pgfpoint{7.780cm}{3.084cm}}
\draw[gp path] (9.897,1.033)--(10.077,1.033);
\node[gp node right] at (9.713,1.033) {{\supertiny $10^{-2}$}};
\draw[gp path] (9.897,1.326)--(10.077,1.326);
\node[gp node right] at (9.713,1.326) {{\supertiny $10^{-1}$}};
\draw[gp path] (9.897,1.619)--(10.077,1.619);
\node[gp node right] at (9.713,1.619) {{\supertiny $10^{0}$}};
\draw[gp path] (9.897,1.912)--(10.077,1.912);
\node[gp node right] at (9.713,1.912) {{\supertiny $10^{1}$}};
\draw[gp path] (9.897,2.205)--(10.077,2.205);
\node[gp node right] at (9.713,2.205) {{\supertiny $10^{2}$}};
\draw[gp path] (9.897,2.498)--(10.077,2.498);
\node[gp node right] at (9.713,2.498) {{\supertiny $10^{3}$}};
\draw[gp path] (9.897,2.791)--(10.077,2.791);
\node[gp node right] at (9.713,2.791) {{\supertiny $10^{4}$}};
\draw[gp path] (9.897,3.084)--(10.077,3.084);
\node[gp node right] at (9.713,3.084) {{\supertiny $10^{5}$}};
\draw[gp path] (9.897,1.033)--(9.897,1.213);
\node[gp node center] at (9.897,0.725) {{\supertiny $10^{-2}$}};
\draw[gp path] (10.190,1.033)--(10.190,1.213);
\node[gp node center] at (10.190,0.725) {{\supertiny $10^{-1}$}};
\draw[gp path] (10.483,1.033)--(10.483,1.213);
\node[gp node center] at (10.483,0.725) {{\supertiny $10^{0}$}};
\draw[gp path] (10.776,1.033)--(10.776,1.213);
\node[gp node center] at (10.776,0.725) {{\supertiny $10^{1}$}};
\draw[gp path] (11.068,1.033)--(11.068,1.213);
\node[gp node center] at (11.068,0.725) {{\supertiny $10^{2}$}};
\draw[gp path] (11.361,1.033)--(11.361,1.213);
\node[gp node center] at (11.361,0.725) {{\supertiny $10^{3}$}};
\draw[gp path] (11.654,1.033)--(11.654,1.213);
\node[gp node center] at (11.654,0.725) {{\supertiny $10^{4}$}};
\draw[gp path] (11.947,1.033)--(11.947,1.213);
\node[gp node center] at (11.947,0.725) {{\supertiny $10^{5}$}};
\draw[gp path] (9.897,3.084)--(9.897,1.033)--(11.947,1.033)--(11.947,3.084)--cycle;
\node[gp node center,rotate=-270] at (8.823,2.058) {MIST};
\node[gp node center] at (10.922,0.263) {Ref.\ w/ invariants};
\gpcolor{color=gp lt color 0}
\gpsetlinetype{gp lt plot 0}
\draw[gp path] (9.897,1.033)--(9.918,1.054)--(9.938,1.074)--(9.959,1.095)--(9.980,1.116)%
  --(10.001,1.137)--(10.021,1.157)--(10.042,1.178)--(10.063,1.199)--(10.083,1.219)--(10.104,1.240)%
  --(10.125,1.261)--(10.145,1.282)--(10.166,1.302)--(10.187,1.323)--(10.208,1.344)--(10.228,1.364)%
  --(10.249,1.385)--(10.270,1.406)--(10.290,1.427)--(10.311,1.447)--(10.332,1.468)--(10.353,1.489)%
  --(10.373,1.509)--(10.394,1.530)--(10.415,1.551)--(10.435,1.572)--(10.456,1.592)--(10.477,1.613)%
  --(10.498,1.634)--(10.518,1.655)--(10.539,1.675)--(10.560,1.696)--(10.580,1.717)--(10.601,1.737)%
  --(10.622,1.758)--(10.642,1.779)--(10.663,1.800)--(10.684,1.820)--(10.705,1.841)--(10.725,1.862)%
  --(10.746,1.882)--(10.767,1.903)--(10.787,1.924)--(10.808,1.945)--(10.829,1.965)--(10.850,1.986)%
  --(10.870,2.007)--(10.891,2.027)--(10.912,2.048)--(10.932,2.069)--(10.953,2.090)--(10.974,2.110)%
  --(10.994,2.131)--(11.015,2.152)--(11.036,2.172)--(11.057,2.193)--(11.077,2.214)--(11.098,2.235)%
  --(11.119,2.255)--(11.139,2.276)--(11.160,2.297)--(11.181,2.317)--(11.202,2.338)--(11.222,2.359)%
  --(11.243,2.380)--(11.264,2.400)--(11.284,2.421)--(11.305,2.442)--(11.326,2.462)--(11.346,2.483)%
  --(11.367,2.504)--(11.388,2.525)--(11.409,2.545)--(11.429,2.566)--(11.450,2.587)--(11.471,2.608)%
  --(11.491,2.628)--(11.512,2.649)--(11.533,2.670)--(11.554,2.690)--(11.574,2.711)--(11.595,2.732)%
  --(11.616,2.753)--(11.636,2.773)--(11.657,2.794)--(11.678,2.815)--(11.699,2.835)--(11.719,2.856)%
  --(11.740,2.877)--(11.761,2.898)--(11.781,2.918)--(11.802,2.939)--(11.823,2.960)--(11.843,2.980)%
  --(11.864,3.001)--(11.885,3.022)--(11.906,3.043)--(11.926,3.063);
\gpcolor{color=gp lt color 2}
\gppoint{gp mark 7}{(10.409,2.043)}
\gppoint{gp mark 7}{(11.659,1.940)}
\gppoint{gp mark 7}{(10.426,2.554)}
\gppoint{gp mark 7}{(10.400,1.033)}
\gppoint{gp mark 7}{(10.301,3.084)}
\gppoint{gp mark 7}{(10.301,1.998)}
\gppoint{gp mark 7}{(10.457,1.326)}
\gppoint{gp mark 7}{(10.296,3.084)}
\gppoint{gp mark 7}{(10.290,1.238)}
\gppoint{gp mark 7}{(10.409,1.261)}
\gppoint{gp mark 7}{(10.413,1.281)}
\gppoint{gp mark 7}{(10.404,3.084)}
\gppoint{gp mark 7}{(10.306,3.037)}
\gppoint{gp mark 7}{(10.311,1.654)}
\gppoint{gp mark 7}{(10.400,1.313)}
\gppoint{gp mark 7}{(10.456,1.313)}
\gppoint{gp mark 7}{(10.494,1.209)}
\gppoint{gp mark 7}{(10.400,1.536)}
\gppoint{gp mark 7}{(10.780,1.891)}
\gppoint{gp mark 7}{(10.710,2.546)}
\gppoint{gp mark 7}{(10.717,3.084)}
\gppoint{gp mark 7}{(10.770,1.888)}
\gppoint{gp mark 7}{(10.715,2.017)}
\gppoint{gp mark 7}{(10.714,2.032)}
\gppoint{gp mark 7}{(10.781,1.888)}
\gppoint{gp mark 7}{(10.714,2.009)}
\gppoint{gp mark 7}{(10.713,3.084)}
\gppoint{gp mark 7}{(10.771,1.888)}
\gppoint{gp mark 7}{(10.715,2.010)}
\gppoint{gp mark 7}{(10.714,2.039)}
\gppoint{gp mark 7}{(10.983,3.084)}
\gppoint{gp mark 7}{(10.981,3.084)}
\gppoint{gp mark 7}{(10.980,3.084)}
\gppoint{gp mark 7}{(10.980,3.084)}
\gppoint{gp mark 7}{(10.913,3.084)}
\gppoint{gp mark 7}{(10.980,3.084)}
\gppoint{gp mark 7}{(10.981,3.084)}
\gppoint{gp mark 7}{(10.980,3.084)}
\gppoint{gp mark 7}{(10.980,3.084)}
\gppoint{gp mark 7}{(11.017,3.084)}
\gppoint{gp mark 7}{(10.981,3.084)}
\gppoint{gp mark 7}{(10.981,3.084)}
\gppoint{gp mark 7}{(10.982,3.084)}
\gppoint{gp mark 7}{(10.981,3.084)}
\gppoint{gp mark 7}{(10.913,3.084)}
\gppoint{gp mark 7}{(10.981,3.084)}
\gppoint{gp mark 7}{(10.981,3.084)}
\gppoint{gp mark 7}{(10.982,3.084)}
\gppoint{gp mark 7}{(10.980,3.084)}
\gppoint{gp mark 7}{(10.913,3.084)}
\gppoint{gp mark 7}{(10.980,3.084)}
\gppoint{gp mark 7}{(10.983,3.084)}
\gppoint{gp mark 7}{(10.980,3.084)}
\gppoint{gp mark 7}{(10.981,3.084)}
\gppoint{gp mark 7}{(10.911,3.084)}
\gppoint{gp mark 7}{(10.981,3.084)}
\gppoint{gp mark 7}{(10.981,3.084)}
\gppoint{gp mark 7}{(10.980,3.084)}
\gppoint{gp mark 7}{(10.980,3.084)}
\gppoint{gp mark 7}{(10.912,3.084)}
\gppoint{gp mark 7}{(10.982,3.084)}
\gppoint{gp mark 7}{(10.982,3.084)}
\gppoint{gp mark 7}{(10.913,3.084)}
\gppoint{gp mark 7}{(10.981,3.084)}
\gppoint{gp mark 7}{(10.981,3.084)}
\gppoint{gp mark 7}{(10.912,3.084)}
\gppoint{gp mark 7}{(10.982,3.084)}
\gppoint{gp mark 7}{(10.979,3.084)}
\gppoint{gp mark 7}{(10.978,3.084)}
\gppoint{gp mark 7}{(10.980,3.084)}
\gppoint{gp mark 7}{(10.912,3.084)}
\gppoint{gp mark 7}{(10.484,3.084)}
\gppoint{gp mark 7}{(10.597,3.084)}
\gppoint{gp mark 7}{(10.456,3.084)}
\gppoint{gp mark 7}{(10.473,3.084)}
\gppoint{gp mark 7}{(10.469,3.084)}
\gppoint{gp mark 7}{(10.457,3.084)}
\gppoint{gp mark 7}{(10.499,1.457)}
\gppoint{gp mark 7}{(10.690,3.084)}
\gppoint{gp mark 7}{(10.434,3.084)}
\gppoint{gp mark 7}{(10.807,3.084)}
\gppoint{gp mark 7}{(10.413,3.084)}
\gppoint{gp mark 7}{(10.416,3.084)}
\gppoint{gp mark 7}{(10.409,3.084)}
\gppoint{gp mark 7}{(10.420,2.473)}
\gppoint{gp mark 7}{(10.700,3.084)}
\gppoint{gp mark 7}{(10.599,3.084)}
\gppoint{gp mark 7}{(10.454,3.084)}
\gppoint{gp mark 7}{(10.488,3.084)}
\gppoint{gp mark 7}{(10.522,3.084)}
\gppoint{gp mark 7}{(10.478,3.084)}
\gppoint{gp mark 7}{(10.451,3.084)}
\gppoint{gp mark 7}{(10.582,3.084)}
\gppoint{gp mark 7}{(10.584,3.084)}
\gppoint{gp mark 7}{(10.582,3.084)}
\gppoint{gp mark 7}{(10.582,3.084)}
\gppoint{gp mark 7}{(10.453,3.084)}
\gppoint{gp mark 7}{(10.461,3.084)}
\gppoint{gp mark 7}{(10.584,3.084)}
\gppoint{gp mark 7}{(10.581,3.084)}
\gppoint{gp mark 7}{(10.407,3.084)}
\gppoint{gp mark 7}{(10.434,3.084)}
\gppoint{gp mark 7}{(10.587,3.084)}
\gppoint{gp mark 7}{(10.699,3.084)}
\gppoint{gp mark 7}{(10.449,3.084)}
\gppoint{gp mark 7}{(10.449,3.084)}
\gppoint{gp mark 7}{(10.459,3.084)}
\gppoint{gp mark 7}{(10.457,3.084)}
\gppoint{gp mark 7}{(10.400,1.541)}
\gppoint{gp mark 7}{(10.397,1.581)}
\gppoint{gp mark 7}{(10.400,1.584)}
\gppoint{gp mark 7}{(10.321,1.586)}
\gppoint{gp mark 7}{(10.723,3.084)}
\gppoint{gp mark 7}{(10.426,1.765)}
\gppoint{gp mark 7}{(10.505,2.424)}
\gppoint{gp mark 7}{(10.785,3.084)}
\gppoint{gp mark 7}{(10.353,1.570)}
\gppoint{gp mark 7}{(10.489,2.057)}
\gppoint{gp mark 7}{(10.316,1.654)}
\gppoint{gp mark 7}{(10.488,1.700)}
\gppoint{gp mark 7}{(10.296,1.281)}
\gppoint{gp mark 7}{(10.439,1.967)}
\gppoint{gp mark 7}{(10.346,2.900)}
\gppoint{gp mark 7}{(10.542,3.084)}
\gppoint{gp mark 7}{(10.476,2.644)}
\gppoint{gp mark 7}{(10.727,3.084)}
\gppoint{gp mark 7}{(10.732,3.084)}
\gppoint{gp mark 7}{(10.378,2.801)}
\gppoint{gp mark 7}{(10.583,3.084)}
\gppoint{gp mark 7}{(10.325,1.996)}
\gppoint{gp mark 7}{(10.372,2.469)}
\gppoint{gp mark 7}{(10.321,2.838)}
\gppoint{gp mark 7}{(10.424,3.084)}
\gppoint{gp mark 7}{(10.430,3.084)}
\gppoint{gp mark 7}{(10.586,3.084)}
\gppoint{gp mark 7}{(10.585,3.084)}
\gppoint{gp mark 7}{(10.449,1.499)}
\gppoint{gp mark 7}{(10.316,1.661)}
\gppoint{gp mark 7}{(10.407,2.240)}
\gppoint{gp mark 7}{(10.311,1.710)}
\gppoint{gp mark 7}{(10.306,1.610)}
\gppoint{gp mark 7}{(10.372,2.256)}
\gppoint{gp mark 7}{(10.466,3.084)}
\gppoint{gp mark 7}{(10.384,3.013)}
\gppoint{gp mark 7}{(10.735,3.084)}
\gpcolor{color=gp lt color border}
\gpsetlinetype{gp lt border}
\draw[gp path] (9.897,3.084)--(9.897,1.033)--(11.947,1.033)--(11.947,3.084)--cycle;
%% coordinates of the plot area
\gpdefrectangularnode{gp plot 6}{\pgfpoint{9.897cm}{1.033cm}}{\pgfpoint{11.947cm}{3.084cm}}
\end{tikzpicture}
%% gnuplot variables

  \caption[Comparison of execution time for Petrinizer vs.~IIC,
  BFC and MIST.\@]{Comparison of execution time for Petrinizer vs.~IIC,
  BFC and MIST\@. Graphs in the top row show comparison in the case without
  invariant construction, and graphs in the bottom row show comparison in
  the case with invariant construction. Axes represent time in seconds on a
  logarithmic scale. Each dot represents execution time on one example.}
\label{fig:performance-vs-tools}
\end{figure}

\paragraph{Comparison with other tools.} The six graphs in
Fig.~\ref{fig:performance-vs-tools} show the comparison of execution times 
for Petrinizer vs.~IIC, BFC, and MIST\@. 
In the comparison, we used
the refinement methods, both with and without invariant construction.
In general, we observe that other tools outperform
Petrinizer on small examples, an effect that can be explained by the
overhead of starting script interpreters and Z3. 
However, on large examples Petrinizer consistently outperforms other tools.
Not only does it finish in all cases within the given time and memory
constraints, it even finishes in under 100
seconds in all but two cases. The two cases are the
large example from the Erlang suite, with 66,950 places and 213,635
transitions and, in the case of invariant construction, the example
from the MIST suite, with 8989 disjuncts in the property violation constraint.

\paragraph{Conclusions.}
%
Marking equations and traps are classical techniques in Petri net theory, but have fallen out of favor
in recent times in comparison with state-space traversal techniques in combination with abstractions
or symbolic representations.
Our experiments demonstrate that, when combined with the power of a modern SMT solver, these techniques
can be surprisingly effective in finding proofs of correctness (inductive invariants) of common benchmark examples arising
out of software verification.

Our results also suggest 
incorporating these techniques into existing tools as a cheap preprocessing step.
A finer integration with these tools is conceivable, where a
satisfying assignment to a system of constraints is used to guide the more
sophisticated search, similar to~\cite{WimmelWolf12}.

% It remains to be seen whether these techniques can be used to check liveness properties of concurrent
% software, which often reduce to Petri net \emph{reachability} problems and are out of scope of other
% state-space traversal techniques. Also, a combination with the technique
% of \cite{WimmelWolf12} should be explored.

\smallskip
\noindent\textit{Acknowledgements.}
%
We thank Emanuele D'Osualdo for help with the Soter tool.
Ledesma-Garza was supported by the
Collaborative Research Center 1480 ``Program and Model Analysis'' funded by the German Research Council.

% Benchmarks

% Properties

% Rate of success

% Invariant sizes

% Performance

% Comparison with other tools

