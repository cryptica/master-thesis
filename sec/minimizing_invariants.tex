\section{Minimizing invariants}
\label{sec:minimizing_invariants}

\paragraph{Minimizing invariants.} Note that the system $\mathcal{S}'$
from Theorem~\ref{thm:dual} may in general have many
solutions, and each solution yields an inductive invariant. Solutions where
$\lambda$ has fewer non-zero components yield shorter 
inductive invariants $I(M)$, assuming terms in $I(M)$
with coefficient zero are left out. 
We can force the number of
non-zero components to be at most $K$ by introducing a vector of $|P|$
variables $Z$, adding for each $p\in P$ constraints
\begin{align*}
  & \lambda(p)>0\implies Z(p)=1 \\
  & \lambda(p)=0 \implies Z(p)=0
\end{align*}
and adding a constraint $\sum_{p\in P} Z(p) \leq K$. By varying $K$, we
can find a solution with the smallest number of non-zero
components in $\lambda$.

% \begin{enumerate}
% \item The constraints given by \safetyref\ are unsat if{}f the
%   constraints by \invariantref\ are satisfiable.
% \item If the constraints by \invariantref\ are sat, then there is some
%   inductive invariant.
% \end{enumerate}

% \\ \mbox{ } \\
% \\ \mbox{ } \\
% \\ \mbox{ } \\
% \\ \mbox{ } \\

% Note we can express a set of trap constraints as a linear equation
% of the form $DM \ge 1$, where $D$ encodes the places in each trap.
% We can combine these with the negation of the property $AM \ge b$
% to obtain a new constraint $A'M \ge b'$, with
% $A' = (D A)^T$ and $b' = (1 b)^T$.
% The new constraint is then equivalent to $DM \ge 1 \land AM \ge b$.

% We can use this constraint instead of the negation of the property
% to obtain a linear constraint $I(M)$ with method Invariant.
% However, to obtain an invariant which only excludes violations of
% the property, we have to add back the trap constraints.
% By this, we obtain the invariant

% $$I'(M) := DM \ge 1 \land I(M)$$

% The invariant $I'(M)$ holds for all reachable markings, as the traps are marked
% initially and therefore also marked in all reachable markings and
% $I(M)$ holds for all reachable markings. 

% Let $M$ be a marking $M$ violating the property, i.e. $AM \ge b$.
% If $DM \ge 1$ does not hold, then $I'(M)$ does not hold.
% Otherwise, $DM \ge 1 \land AM \ge b$ holds and by the
% the properties of $I(M)$ also $I(M)$ and $I'(M)$ do not hold.

% The invariant $I'(M)$ is inductive, as traps are inductive and $I(M)$ is inductive.

\paragraph{Example.}

\begin{figure}[t]
Inductivity constraints
$$
\begin{array}[t]{clclclclclcl}
    -  & p_1 &  +  & p_2 &     &     &  +  & bit_1 &  -  & notbit_1 &  \le  & 0 \\
       &     &  -  & p_2 &  +  & p_3 &     &       &     &          &  \le  & 0 \\
       & p_1 &     &     &  -  & p_3 &  -  & bit_1 &  +  & notbit_1 &  \le  & 0
\end{array}
$$

$$
\begin{array}[t]{clclclclclclclclclcl}
    -  & q_1 &  +  & q_2 &     &     &     &     &     &     &  -  & notbit_2 &  \le  & 0  \\
       &     &  -  & q_2 &  +  & q_3 &     &     &     &     &     &          &  \le  & 0  \\
       &     &     &     &  -  & q_3 &  +  & q_4 &     &     &  +  & notbit_2 &  \le  & 0  \\
       & q_1 &     &     &     &     &  -  & q_4 &     &     &     &          &  \le  & 0  \\
       &     &  -  & q_2 &     &     &     &     &  +  & q_5 &     &          &  \le  & 0  \\
       & q_1 &     &     &     &     &     &     &  -  & q_5 &  +  & notbit_2 &  \le  & 0 
\end{array}
$$

Safety constraint
$$    
p_1 + q_1 + notbit_1 + notbit_2 < target_1 + target_2 + trap_1
$$

Property constraints
\begin{alignat*}{4}
  p_1 & \ge 0 & \qquad q_1 & \ge 0 & \qquad q_4 & \ge 0 & \qquad bit_1 & \ge 0  \\
  p_2 & \ge trap_1 & \qquad q_2 & \ge trap_1 & \qquad q_5 & \ge target_2& \qquad notbit_1 & \ge trap_1 \\
  p_3 & \ge target_1  & \qquad q_3 & \ge trap_1 & & & \qquad notbit_2 & \ge trap_1
\end{alignat*}

Non-negativity constraints
$$
  target_1, target_2, trap_1 \ge 0
$$

\caption[Constraints $\mathcal{S}'$ for Lamport's
    algorithm and the mutual exclusion property.]
    {Constraints $\mathcal{S}'$ for Lamport's
algorithm and the mutual exclusion property. Here, $\lambda=(p_1\ p_2\
p_3\ q_1\ q_2\ q_3\ q_4\ q_5\ bit_1\ notbit_1\ notbit_2)$,
$Y_1=(target_1\ target_2)$ and $Y_2=(trap_1)$.}
\label{fig:example-constraints}
\end{figure}

Consider again Lamport's algorithm and the mutual
exclusion property. Recall that the negation of the property for this example is $p_3 \ge 1 \land q_5 \ge 1$, and the trap constraint is
$p_2 + q_2 + q_3 + notbit_1 + notbit_2 \ge 1$.
%
Fig.~\ref{fig:example-constraints} shows the constraints $\mathcal{S}'$ for this example. 
A possible satisfying assignment sets $q_1$, $q_4$, and
$bit_1$ to 0, $p_2$, $p_3$, and $target_1$ to 2, and all other
variables to 1.
%\begin{alignat*}{5}
%  p_1 &= 1 & \qquad q_1 &= 0 & \qquad q_4 &= 0 & \qquad bit_1 &= 0 &
%  \qquad target_1 &= 2 \\
%  p_2 &= 2 & \qquad q_2 &= 1 & \qquad q_5 &= 1 & \qquad notbit_1 &= 1 &
%  \qquad target_2 &= 1 \\
%  p_3 &= 2 & \qquad q_3 &= 1 & & & notbit_2 &= 1 & \qquad trap_1 &= 1
%\end{alignat*}
The corresponding inductive invariant is:
\begin{align*}
I(M)\ ::=&\ (p_2 + q_2 + q_3 + notbit_1 + notbit_2 \ge 1 ) \; \land \\
&\ (p_1 + 2 p_2 + 2 p_3 + notbit_1 + notbit_2 + q_2 + q_3 + q_5 \le 3)\,.
\end{align*}
%
If we add constraints that bound the number of non-zero components in
$\lambda$ to 7, the SMT solver finds a new solution, setting $p_2$, $p_3$, $notbit_1$, $notbit_2$, $q_2$, $q_3$,
$target_1$, $target_2$, and $trap_1$ to $1$, and all other variables
to $0$. The corresponding inductive invariant for this solution is
\begin{align*}
I'(M)\ ::=&\ 
(p_2 + q_2 + notbit_1 + notbit_2 + q_3 \ge 1 ) \; \land \\
&\ (p_2 + p_3 + notbit_1 + notbit_2 + q_2 + q_3 + q_5 \le 2)\,.
\end{align*}
% The next section shows how to get a simpler one.

\begin{figure}
  \begin{tikzpicture}[
    every path/.style={draw, ->, >=stealth, shorten >=2pt, shorten <=2pt}
    ]
    \node[state] (begin) {BEGIN};
    \node[action, below=of begin] (c) {$C:=\mathcal C(N)$\\
      $D:=\{\}$};
    \node[decision, below=of c] (satc) {$\text{SAT}(C \cup \{\neg P\})$};
    \node[action, below=of satc] (modelc) {$A:=\text{Model}(C \cup \{\neg P\})$\\
      $C_{\theta}:=\text{TrapConditions}(N, A)$};
    \node[decision, below=of modelc] (satctheta) {$\text{SAT}(C_\theta)$};
    \node[action, below=of satctheta] (modelctheta) {$A_\theta:=\text{Model}(C_\theta)$\\
      $\delta:=\Delta(A_\theta)$\\
      $C:=C \cup \{\delta\}$\\
      $D:=D \cup \{\delta\}$};
    \node[print, right=of satctheta] (noinv2) {No Invariant};
    \node[state, below=of noinv2] (end3) {END};

    \node[action, right=of satc] (cprime) {$C':=\mathcal C'(N, \neg P \land D)$};
    \node[decision, right=of cprime] (satcprime) {$\text{SAT}(C')$};
    \node[action, below=of satcprime] (inv) {$A':=\text{Model}(C')$\\
      $I:=D \land \text{Inv}(N, A')$};
    \node[print, below=of inv] (printinv) {Invariant: $I$};
    \node[print, right=of satcprime] (noinv) {No Invariant};
    \node[state, below=of noinv] (end1) {END};
    \node[state, below=of printinv] (end2) {END};

    \draw (cprime) edge (satcprime);
    \draw (satcprime) edge node[above]{NO} (noinv);
    \draw (satcprime) edge node[right]{YES} (inv);
    \draw (inv) edge (printinv);
    \draw (noinv) edge (end1);
    \draw (printinv) edge (end2);

    \draw (begin) edge (c);
    \draw (c) edge coordinate[pos=.5] (edgein) (satc);
    \draw (satc) edge node[above]{NO} (cprime);
    \draw (satc) edge node[right]{YES} (modelc);
    \draw (modelc) edge (satctheta);
    \draw (satctheta) edge node[above]{NO} (noinv2);
    \draw (noinv2) edge (end3);
    \draw (satctheta) edge node[right]{YES} (modelctheta);
    \draw (modelctheta.south) -- ([yshift=-0.5cm] modelctheta.south)
    -| ([xshift=-2cm] modelctheta.west) |- (edgein);
  \end{tikzpicture}
\caption{Diagram for Function Invariant by Refinement}
\label{fig:function_invariant_by_refinement_diagram}
\end{figure}


