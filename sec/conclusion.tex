\chapter{Conclusion}
\label{chap:conclusion}

\paragraph{Summary.}

In this thesis, we presented
a method for checking safety properties via marking equations and traps
and a method for checking liveness properties via transition invariants and S-components.

While both methods have been used in the past, we combined them with the power of a modern
SMT solver and showed that they are very efficient, often outperforming
state-of-the-art model checkers.
These constraint-based approaches offer reduced complexity at the cost of completeness,
but when applied to common benchmarks arising out of software verification, they prove to be
quite complete.

Additionally, we presented a new method to find inductive invariants for safety properties.
These can be used as proofs of correctness and are usually smaller than invariants returned
by other tools.

All in all, this gives us a very useful tool,
both on its own and in combination with other tools as a cheap preprocessing step.

%Marking equations and traps are classical techniques in Petri net theory, but have fallen out of favor
%in recent times in comparison with state-space traversal techniques in combination with abstractions
%or symbolic representations.
%Our experiments demonstrate that, when combined with the power of a modern SMT solver, these techniques
%can be surprisingly effective in finding proofs of correctness (inductive invariants) of common benchmark examples arising
%out of software verification.
% It remains to be seen whether these techniques can be used to check liveness properties of concurrent
% software, which often reduce to Petri net \emph{reachability} problems and are out of scope of other
% state-space traversal techniques. Also, a combination with the technique
% of \cite{WimmelWolf12} should be explored.

\paragraph{Future work.}

With the method \safety\ one can only prove that markings are not reachable, and not that they
are reachable. Following an approach similar to~\cite{WimmelWolf12}, the
constraints could be used to guide a sophisticated search to a given marking.
This would allow the method to positively answer reachability problems.

The liveness properties that can be proven by method \liveness\ are somewhat limited.
Maybe other properties can be encoded similiarly,
or the method could be combined with Büchi nets like in~\cite{EsparzaMelzer97} to allow
any LTL-formula to be specified. 

It would also be interesting to see if the method can be applied to
other benchmarks with different kinds of properties.
For example, liveness properties of concurrent software
often reduce to Petri net reachability problems~\cite{GantyMajumdar12}
which are out of scope of other state-space traversal techniques.
Our method might be able to handle those problems.

