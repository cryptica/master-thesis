\section{Refining transition invariants}
\label{sec:refining_transition_invariants}

As the test with transition invariants is rather weak in general,
it has been strengthened by Esparza and Melzer~\cite{EsparzaMelzer97} with
the use of \emph{S-components}.
An \emph{S-component}~\cite{DeselEsparza95} of a Petri net $(P,T,F,m_0)$
is a subnet $(P',T')$, where
$\pre{p} ∪ \post{p} ⊆ T'$ for all places $p ∈ P'$ and
$|\pre{t} ∩ P'| = 1 = |\pre{t} ∩ P'|$ for all transitions $t ∈ T'$.
All transitions connected to a place in the S-component are also
part of the component, and
each transition of the component takes exactly one token out
of and puts one token back in the places in the component when fired.
Therefore, S-components are also \emph{S-invariants}, and the total
number of tokens in the places never changes, i.e.\ for an
S-component $(P',T')$, we have
$m(S') = m_0(S')$ for all reachable markings $m$.

We can use S-components for excluding some unrealisable transition invariants $X$
returned by the method \liveness. We will call the new method \livenessref.

\begin{lemma}
\label{lem:scom-trans}
For an S-component $(P',T')$ and a semi-positive transition
T-surinvariant $X$,
if the subnet $(P', T' \cap \|X\|)$ contains two transitions $t$ and $t'$
not strongly connected by the flow relation $F'$ induced
by that subnet, and $m_0(P') = 1$,
then $X$ is not realisable.
\end{lemma}
\begin{proof}
Assume $X$ is realisable. Then there exists an
infinite occurence sequence $\tau$ with $\|X\| = \inf(X)$.
There is a suffix of $\tau$ from where on all occuring
transitions occur infinitely often, and as $t,t' ∈ T' \cap \|X\| = \inf(X)$,
they occur in that suffix.
As both $t$ and $t'$ are part of the S-component, we have places
$p ∈ \pre{t} \cap P'$ and $p' ∈ \pre{t} \cap P'$ as well as
$q ∈ \post{t} \cap P'$ and $q' ∈ \post{t} \cap P'$.
There is always exactly one token in $P'$,
and due to the properties of the S-component, the token can only move between places
in $P'$ by transitions in $T'$,
and in the infinite suffix only by transitions in $\|X\|$.
At some point, it will stay in one of the strongly connected components
of the subnet $(P', T' \cap \|X\|)$.

Not all of $p$,$p'$,$q$ and $q'$ can be in the same strongly connected component,
as otherwise $t$ and $t'$ would be strongly connected.
If $p$ and $p'$ are in the same component,
as $t$ and $t'$ fire infinitely often,
the token will move to a different component of $q$ or $q'$ at some point.
If they are in different components, the token will be in one of those.
In both cases, either $p$ or $p'$ will stay unmarked from some point on,
leaving $t$ or $t'$ unable to fire and therefore $X$ is not realisable.
\end{proof}

To find such an S-component for a given $X$ with an SMT solver, we need
to generate constraints capturing these properties.
To capture the aspect of strong connectivity
of a subnet $(P', U)$ ($U$ will be $T' \cap \|X\|$ for an S-component $(P',T')$),
we introduce the relation $\rightsquigarrow_U \subseteq U \times U$.
It is defined as follows: $t \rightsquigarrow_U t'$ iff $t,t' \in U$ and
there exists a place $p \in P'$ such that $t \in \pre{p}$ and $t' \in \post{p}$.
A set $V \subseteq U$ is \emph{closed} under $\rightsquigarrow_U$ if
$t \in V$ and $t \rightsquigarrow t'$ implies $t' \in V'$.

As the relation follows the paths in the subnet,
a closed $V$ represents the transitions for sets of strongly connected components.
Note that $U$ itself is always closed.
With that we get the following lemma~\cite{EsparzaMelzer97}:
\begin{lemma}
\label{lem:closed-conn}
    For a subnet $(P',U)$,
    if there exists a $V \subseteq U$ that is closed under $\rightsquigarrow_U$,
    is non-empty and is not equal to $U$,
    then there exist $t,t' \in U$ that are not strongly connected in the subnet $(P',U)$.
\end{lemma}
\begin{proof}
    As $V$ is not empty and not equal to $U$, there exists $t \in V$ and $t' \in U \backslash V$.
    If there would be a path from $t$ to $t'$, there would be
    $v \rightarrow p \rightarrow u$ along that path with $v \in V$,
    $p \in P'$ and $u \notin V$, which would mean $V$ is not closed.
\end{proof}

With that, we can formulate the constraints that,
given a semi-positive transition surinvariant $X$,
if satisfiable
give us an S-component $N' = (P',S')$ to reject $X$ as unrealisable.

\vspace{-0.25cm}
\begin{itemize}
    \item $N' = (P',T')$ is an S-component.
    \item $V$ is a proper, non-empty subset of $T' \cap X$, i.e.
          $V \subsetneq T' \cap X$ and $V \ne \emptyset$.
    \item $V$ is closed under $\rightsquigarrow_{T' \cap \|X\|}$, i.e.
          $\forall p \in P': \forall t \in \pre{p} \cap X, t' \in \post{p} \cap X :
          t \in V \implies t' \in V$.
    \item $N'$ initially contains one token, i.e. $m_0(P') = 1$.
\end{itemize}
\vspace{-0.25cm}

To solve these constraints with an SMT solver, the free variables $P'$,$T'$ and $V$
could be encoded as boolean variables representing sets.
However, we can make use of the SMT solvers capabilities and encode them
using vectors over integer variables restricted to $\{0,1\}$.
This is more efficient, as we can encode set size like $|P'|$ as $\sum_{p ∈ P} P'(p)$.
Membership $p ∈ P$ can be encoded as $P(p) > 0$.

Then, given a Petri net $N = (P, T, F, m_0)$ and a semi-positive T-surinvariant $X$,
with the free variables $P'$ of size $|P|$, $T'$ of size $|T|$ and $V$ of size $| \|X\| |$,
the refinement constraints $\mathcal{R}(P, T, F, m_0, X)$ are defined by
the following constraints~\ref{eq:scomp-first} to~\ref{eq:scomp-last}.

$N' = (P',T')$ is an S-component:
\begin{align}
\label{eq:scomp-first}
    \begin{aligned}
    &\bigwedge_{p ∈ P} \left[ P'(p) > 0 \implies
        \bigwedge_{t ∈ \pre{p} \cup \post{p}} T'(t) \right] \land \\
    &\bigwedge_{t ∈ T} \left[ T'(p) > 0 \implies
        \sum_{p ∈ \pre{t}} P'(p) = 1 \land \sum_{p ∈ \post{t}} P'(p) = 1 \right]
    \end{aligned}
\end{align}
$V$ is a proper subset of $T' \cap \|X\|$:
\begin{align}
    \bigwedge_{t ∈ X} \left[ V(t) > 0 \implies T'(t) > 0 \right] \land
    \left[ \sum_{t ∈ \|X\|} V(t) < \sum_{t ∈ \|X\|} T'(t) \right]
\end{align}
$V$ is non-empty:
\begin{align}
    \sum_{t ∈ \|X\|} V(t) > 0
\end{align}
$V$ is closed under $\rightsquigarrow_{T' \cap \|X\|}$:
\begin{align}
    \bigwedge_{p ∈ P} \left[ P'(p) > 0 \implies
    \left( \bigwedge_{\substack{t ∈ \pre{p} \cap \|X\|\\ t' ∈ \post{p} \cap \|X\|}} V(t) > 0 \implies V(t') > 0 \right) \right]
\end{align}
$N'$ initially contains one token:
\begin{align}
    \sum_{p ∈ P} m_0(p) \cdot P'(p) = 1
\end{align}
The variables $P'$,$T'$ and $V$ are binary vectors:
\begin{align}
\label{eq:scomp-last}
    \bigwedge_{p ∈ P}  \left[ 0 \le P'(p) \le 1 \right] \land
    \bigwedge_{t ∈ T'} \left[ 0 \le T'(t) \le 1 \right] \land
    \bigwedge_{t ∈ \|X\|} \left[ 0 \le V'(t) \le 1 \right]
\end{align}

When the constraints $\mathcal{R}(P, T, F, m_0, X)$ are satisfiable,
yielding an S-component $(S',T')$, we know that
$X$ is not realisable by Lemma~\ref{lem:scom-trans} and~\ref{lem:closed-conn} and can exclude it.
We can also exclude any other T-surinvariant $Y$ where
$T' \cap \|Y\| = T' \cap \|X\|$, as it would yield the same subnet with
the same partition. The constraint added to exclude those is
\begin{equation}
\left[ \bigwedge_{t ∈ T' \cap \| X \|} t \right] \implies
\left[ \bigvee_{t ∈ T' \backslash \| X \|} t \right],
\end{equation}
which can be read as ``if all the transitions making the subnet not strongly connected fire,
at least one other transition in the subnet needs to fire to make it connected''.
We add these constraints until the set of constraints is unsatisfiable or
no further S-components can be found.

\paragraph{Example.}
Consider again the Petri net for the asynchronous program from
Figure~\ref{fig:asynchronous_net}. Method \liveness\ gave us the
T-surinvariant $X$ with
$X(dg) = X(dh) = X(gf) = X(gt) = 1$ and $X(gt) = X(hf) = 0$.
When solving the constraints $R(P , T , F , m 0 , X)$ for this net and this $X$,
we obtain the subnet on the left of Figure~\ref{fig:asynchronous_net_scomponent}.
The subnet represents the state of the variable $x$, which can only be $\true$ or
$\false$.
On the right of Figure~\ref{fig:asynchronous_net_scomponent}
is the subnet with the intersection with $\|X\|$. If $X$ would be realisable,
$x$ would be tested infinitely often for $\true$ and also for $\false$.
However, $x$ only changes value a finite amount of times, which is shown
through the non-connectivity of the subnet. Therefore $X$ is not realisable.
The constraint $gf \land ht \implies hf \lor gf$ is added to exclude it,
and then the system is unsatisfiable, so we can prove the liveness property
for fair termination.

\begin{figure}[t]
  \begin{center}
    \begin{minipage}[t]{.48\columnwidth}
    \centering
    \begin{tikzpicture}[
        scale=1.0,
        every node/.style={scale=1.0}
      ]
      \draw[box] (1.5,6.25) rectangle (6.5,2.25);
      \node at (4, 2.75) {$(P',T')$};

      \node[place, tokens=1, label=above:$x$] (x) at (4,5.5) {};
      \node[place, label=below:$notx$] (notx) at (4,4) {};

      \node[transition, label=below:$ht$] (ht) at (2.25,5.5) {};
      \node[transition, label=below:$hf$] (hf) at (2.25,4) {};

      \node[transition, label=below:$gt$] (gt) at (5.75,5.5) {};
      \node[transition, label=below:$gf$] (gf) at (5.75,4) {};

      \draw (x) edge[flow, <->] (ht);
      \draw (notx) edge[flow, <->] (hf);

      \draw (x) edge[flow] (gt);
      \draw (gt) edge[flow] (notx);
      \draw (notx) edge[flow, <->] (gf);
    \end{tikzpicture}
    \end{minipage}
    \begin{minipage}[t]{.48\columnwidth}
    \centering
    \begin{tikzpicture}[
        scale=1.0,
        every node/.style={scale=1.0}
      ]
      \draw[box] (1.5,6.25) rectangle (6.5,2.25);
      \node at (4, 2.75) {$(P',T' \cap \|X\|)$};

      \node[place, tokens=1, label=above:$x$] (x) at (4,5.5) {};
      \node[place, label=below:$notx$] (notx) at (4,4) {};

      \node[transition, label=below:$ht$] (ht) at (2.25,5.5) {};

      \node[transition, label=below:$gf$] (gf) at (5.75,4) {};

      \draw (x) edge[flow, <->] (ht);

      \draw (notx) edge[flow, <->] (gf);
    \end{tikzpicture}
    \end{minipage}
  \end{center}
\caption{S-component of the Petri net for the asynchronous program.}
\label{fig:asynchronous_net_scomponent}
\end{figure}



