\section{Transition invariants}
\label{sec:transition_invariants}

We will now introduce the method \liveness, which uses
a variant of semi-positive transition invariants~\cite{EsparzaBruns96}
to prove liveness properties. As an example, consider the
asynchronous program given in Figure~\ref{fig:asynchronous_src}.
Here, the $\algPost$ command is a non-blocking operation for
launching a process in parallel. Assume there is a scheduler
executing the processes in a non-deterministic order.
Initially, the process $\textsc{Main}$ is executed, which
sets $x$ to $true$ and launches $\textsc{H}$.
Process $\textsc{H}$ launches new instances of $\textsc{H}$
and $\textsc{G}$ until $\textsc{G}$ sets $x$ to $\false$.
Assuming a fair scheduler, i.e.\ one that will execute each process
eventually, the program should terminate. This
\emph{fair termination} is the liveness
property we will want to prove.

\begin{figure}[t]
  \hrule
  \centering
  \begin{minipage}[t]{.04\columnwidth}
   \mbox{} \\ \\ \\ $h$:\\ \\ \\
  \end{minipage}
  \begin{minipage}[t]{.26\columnwidth}
    \mbox{}\\
    \algProcedure \textsc{H} \\
    \algBegin \\
    \tabT \algIf $x$ \algThen \\
    \tabTT \algPost\textsc{} H \\
    \tabTT \algPost\textsc{} G \\
    \tabT \algFi \\
    \algEnd
  \end{minipage}
  \begin{minipage}[t]{.04\columnwidth}
   \mbox{} \\ \\ \\ $g$:\\ \\ \\
  \end{minipage}
  \begin{minipage}[t]{.26\columnwidth}
    \mbox{}\\
    \algProcedure \textsc{G} \\
    \algBegin \\
    \tabT $x$ \algAssgn $\false$ \\
    \algEnd
  \end{minipage}
  \begin{minipage}[t]{.04\columnwidth}
   \mbox{} \\ \\ \\ \\ \\ \\
  \end{minipage}
  \begin{minipage}[t]{.26\columnwidth}
    \mbox{}\\
    \algProcedure \textsc{Main} \\
    \algBegin \\
    \tabT $x$ \algAssgn $\true$ \\
    \tabT \algPost\textsc{H} \\
    \algEnd
  \end{minipage}
  \vspace{1.5ex}
  \hrule
\caption{Asynchronous program.}
\label{fig:asynchronous_src}
\end{figure}


Transforming the net into a Petri net gives us the net in
Figure~\ref{fig:asynchronous_net}. It has an added scheduler
$s$, pending instances $ph$ and $pg$, transitions for
dispatching processes $dh$ and $dg$ and exiting the processes
depending on the truth value of $x$.
If the scheduler is fair and continues dispatching instances
$\textsc{H}$ and $\textsc{G}$ 
infinitely, the program should terminate, giving us as the
the liveness property $\psi = (dh \land dg \implies \false)$
or simply $\neg (dh \land dg)$.

\begin{figure}[t]
  \begin{center}
    \begin{tikzpicture}[
        scale=1.0,
        every node/.style={scale=1.0}
      ]
      \draw[box] (-0.25,8.75) rectangle (3.25,2.25);
      \draw[box] (4.75,8.75) rectangle (8.25,2.25);

      \node at (1.5, 2.75) {Process \textsc{H}};
      \node at (6.5, 2.75) {Process \textsc{G}};

      \node[place, tokens=1, label=below:$s$] (s) at (4,8.25) {};
      \node[place, tokens=1, label=below:$x$] (x) at (4,5.5) {};
      \node[place, label=below:$notx$] (notx) at (4,4) {};

      \node[place, tokens=1, label=left:$ph$] (ph) at (2.25,6.75) {};
      \node[place, label=left:$h$] (h) at (0.75,4.75) {};

      \node[place, label=right:$pg$] (pg) at (5.75,6.75) {};
      \node[place, label=right:$g$] (g) at (7.25,4.75) {};

      \node[transition, label=left:$dh$] (dh) at (0.75,7.5) {};
      \node[transition, label=below:$ht$] (ht) at (2.25,5.5) {};
      \node[transition, label=below:$hf$] (hf) at (2.25,4) {};

      \node[transition, label=right:$dg$] (dg) at (7.25,7.5) {};
      \node[transition, label=below:$gt$] (gt) at (5.75,5.5) {};
      \node[transition, label=below:$gf$] (gf) at (5.75,4) {};

      \draw (s) edge[flow] (dh);
      \draw (ph) edge[flow] (dh);
      \draw (dh) edge[flow] (h);
      \draw (h) edge[flow] (ht);
      \draw (h) edge[flow] (hf);
      \draw (x) edge[flow, <->] (ht);
      \draw (notx) edge[flow, <->] (hf);
      \draw (ht) edge[flow] (s);
      \draw (ht) edge[flow] (ph);
      \draw (ht) edge[flow] (pg);
      \draw[rounded corners=4mm, flow] (hf.west) -- (-0.75,4) -- (-0.75,9.25) -| ([xshift=-1mm, yshift=-0.2mm] s.north);

      \draw (s) edge[flow] (dg);
      \draw (pg) edge[flow] (dg);
      \draw (dg) edge[flow] (g);
      \draw (g) edge[flow] (gt);
      \draw (g) edge[flow] (gf);
      \draw (x) edge[flow] (gt);
      \draw (gt) edge[flow] (notx);
      \draw (gt) edge[flow] (s);
      \draw (notx) edge[flow, <->] (gf);
      \draw[rounded corners=4mm, flow] (gf.east) -- (8.75,4) -- (8.75,9.25) -| ([xshift=1mm, yshift=-0.2mm] s.north);
    \end{tikzpicture}
  \end{center}
\caption{Petri net for the asynchronous program.}
\label{fig:asynchronous_net}
\end{figure}



The main idea of method \liveness\ are transition invariants.
Note that we are using a slighty weaker version of invariants,
here called T-surinvariants. This allows the method to be
used on general Petri nets and not only 1-safe nets.
For a Petri net $N = (P, T, F, m_0)$,
a vector $X$ of size $|T|$ is
called a \emph{semi-positive T-surinvariant} if it satisfies the following
constraints:
\begin{align*}
  \mathcal{T}(P, T, F, m_0) ::&
  \begin{cases}
    C X \ge 0 & \text{T-surinvariant} \\
      X \ge 0 & \text{non-negativity conditions} \\
      X \ne 0 & \text{non-zero condition} \\
  \end{cases}\,,
\end{align*}
where $C$ is the incidence matrix of $N$.
For a vector $X$ of $|T|$ variables, we define
its support $\|X\|$ as $\{ t ∈ T \mid X(t) > 0 \}$.

Semi-positive T-invariants have an interesting property relating to infinite
occurence sequences (proof adapted from~\cite{EsparzaBruns96}):
\begin{lemma}
\label{lem:inf-inv}
    If a Petri net $N$ has an infinite occurence sequence $\tau$, then
    there is a semi-positive T-surinvariant $X$ such that
    $\|X\| = \inf(\tau)$.
\end{lemma}
\begin{proof}
    Let $\tau' = m_k \xrightarrow{t_{k+1}} m_{k+1} \xrightarrow{t_{k+2}} m_{k+2} \ldots$ be
    a suffix of $\tau$ such that every transition of $N$ appears either
    never or infinitely often in $\tau'$. The sequence $m_k m_{k+1} m_{k+2} \ldots$
    is an infinite sequence of distinct $|P|$-tuples of natural numbers.
    By Dickson's lemma~\cite{Dickson13}, this sequence contains an
    infinite subsequence $m_{i_1} \le m_{i_2} \le m_{i_3} \le \ldots$.
    Every transition in $\inf(\tau)$ needs to appear eventually after
    $m_{i_1}$, so there is an $m_{i_j}$ such that all $t \in \inf(\tau)$
    appear in the subsequence between $m_{i_1}$ and $m_{i_j}$. With
    $X$ as a vector of transition occurence counts in that subsequence,
    we have $\|X\| = \inf(\tau)$, $C X = m_{i_j} - m_{i_1} \ge 0$,
    $X \ge 0$ and $X \ne 0$ as the sequence is not empty.
\end{proof}
The reverse does not always hold. For a semi-positive T-surinvariant $X$,
if there exists an infinite occurence sequence $\tau$ with $\|X\| = \inf(\tau)$,
we will call $X$ \emph{realisable}.

Method \liveness\ checks if a Petri net $N$ satisfies a liveness
property $\psi$ by checking if the following constraints are satisfiable:
\begin{equation}\label{eq:liveness}
  \mathcal{T}(P, T, F, m_0) \land \neg\psi(\|X\|)
\end{equation}

If the constraints are unsatisfiable, then no infinite occurence
sequence violates $\psi$ and therefore $N \models \psi$.
If there a sequence $\tau$ with $\tau \models \neg\psi$,
then by Lemma~\ref{lem:inf-inv} there is a semi-positive T-surinvariant
$X$ with $\|X\| = \inf(\tau)$, which would satisfy the constraints.

%\begin{theorem}
%    If a Petri net $N$ satisfies a liveness property $\psi$,
%    then $N$ has an infinite occurence sequence $\tau$, then
%    there is a semi-positive T-invariant $X$ such that
%    $\|X\| = \inf(\tau)$.
%\end{theorem}

Consider the Petri net for the asynchronous program from
Figure~\ref{fig:asynchronous_net}.
The constraints $\mathcal{T}(P, T, F, m_0)$ for the net are
given in Figure~\ref{fig:setT}. However, when combined with
the negation for the fair termination property
$\neg\psi = dh \land dg$, which would translate to the constraints
$\mathcal{T}(P, T, F, m_0) \land dh > 0 \land dg > 0$, they are still satisfiable,
for example by assigning $dg = dh = gf = gt = 1$ and $gt = hf = 0$.
Therefore we cannot prove that the property holds.

\begin{figure}
  \begin{minipage}{.79\columnwidth}
    \emph{Quasi-T-invariant} \\ \\[-2em]
    \begin{alignat*}{22}
    &{}-{}& dh &{}-{}& dg &{}+{}& ht &{}+{}& hf &{}+{}& gt &{}+{}& gf &{}\ge{}& 0 \\[-0.1cm]
    &     &    &     &    &     &    &     &    &{}-{}& gt &     &    &{}\ge{}& 0 \\[-0.1cm]
    &     &    &     &    &     &    &     &    &     & gt &     &    &{}\ge{}& 0 \\[-0.1cm]
    &{}-{}& dh &     &    &{}+{}& ht &     &    &     &    &     &    &{}\ge{}& 0 \\[-0.1cm]
    &     & dh &     &    &{}-{}& ht &{}-{}& hf &     &    &     &    &{}\ge{}& 0 \\[-0.1cm]
    &     &    &{}-{}& dg &{}+{}& ht &     &    &     &    &     &    &{}\ge{}& 0 \\[-0.1cm]
    &     &    &     & dg &     &    &     &    &{}-{}& gt &{}-{}& gf &{}\ge{}& 0 \\[-0.1cm]
    \end{alignat*}
  \end{minipage}
  \begin{minipage}{.19\columnwidth}
    \emph{Non-negativity conditions} \\[-2em]
    \begin{align*}
         dh   & \ge 0 \\[-0.1cm]
         dg   & \ge 0 \\[-0.1cm]
         ht   & \ge 0 \\[-0.1cm]
         hf   & \ge 0 \\[-0.1cm]
         gt   & \ge 0 \\[-0.1cm]
         gf   & \ge 0
    \end{align*}
  \end{minipage}
    \emph{Non-zero condition} \\[-2em]
    \begin{align*}
         dh + dg + ht + hf + gt + gf  & > 0
    \end{align*}
  \caption{Constraints $\mathcal{T}$ given by function \liveness\ for
      the asynchronous program.}
\label{fig:setT}
\end{figure}


%Method \safety{} for checking that a property $\varphi$ is
%invariant for a Petri net $N=(P,T,F,m_0)$ consists of checking for
%satisfiability of the constraints
%\begin{equation}\label{eq:safety}
%\mathcal{C}(P,T,F,m_0) \wedge \lnot \varphi(M)\,.
%\end{equation}
%If the constraints are unsatisfiable, then no reachable marking
%violates $\varphi$. To see that this is true, consider the converse:
%If there exists an occurrence sequence $m_0
%\xrightarrow{\sigma} m$ leading to a marking $m$ that violates the
%property, then we can construct a valuation of the variables that assigns $m(p)$ to
%$M(p)$ for each place $p$, and the number of occurrences of $t$ in
%$\sigma$ to $X(t)$ for each transition $t$. This valuation then
%satisfies the constraints.

%We now recall a well-known method, which we call \safety, that provides a sufficient
%condition for a given Petri net $N$ to satisfy a property $\varphi$ 
%by reducing the problem to checking satisfiability of a linear arithmetic formula.
%We illustrate the method on Lamport's 1-bit algorithm for mutual
%exclusion. 



%Before going into details, we state several conventions. For a
%Petri net $N = (P, T, F, m_0)$, we introduce a vector of
%$|P|$ variables $M$, and a vector of $|T|$ variables $X$.
%The vectors $M$ and $X$ will be used to represent the current
%marking and the number of occurrences
%of transitions in the occurrence sequence leading to the current marking, respectively.
%If a place or a transition is given a specific
%name, we use the same name for its associated variable. Given a place
%$p$, the intended meaning of a constraint like $p \geq 3$ is 
%``at the current marking place $p$ must have at least 3 tokens.'' 
%Given a transition $t$, the intended meaning of a constraint like $t \leq 2$ is ``in
%the occurrence sequence leading to the current marking, transition $t$
%must fire at most twice.'' 

%The key idea of the \safety{} method lies in the \emph{marking
%  equation}:
%$$M = m_0 + C X\,,$$
%where the incidence matrix $C$ is a $|P| \times |T|$ matrix given by
%$$C(p,t) = F(t,p) - F(p,t)\,.$$
%For each place $p$, the marking equation contains a constraint
%that formulates a simple token conservation law: the number of tokens in
%$p$ at the current marking is equal to the initial number of tokens $m_0(p)$,
%plus the number of tokens added by the input transitions of $p$, minus
%the number of tokens removed by the output transitions. 
%So, for instance, in Lamport's algorithm the constraint for place $notbit_1$
%is:
%$$notbit_1 = 1 + s_3 + t_5 + t_4 - s_1 - t_4 - t_5 = 1 +s_3
%-s_1\,.$$

%\begin{definition}[Transition invariant]
%A \emph{transition invariant} (also \emph{T-invariant}) of
%a Petri net is a firing vector $X$ with
%$$
%CX = 0.
%$$
%\end{definition}

%\begin{definition}[Transition surinvariant]
%A \emph{semi-positve transition surinvariant} of
%a Petri net is a firing vector $X$ with
%\begin{align*}
%CX &≥ 0 \\
% X &≥ 0 \\
% X &≠ 0.
%\end{align*}
%\end{definition}

%\begin{lemma}
%    If a Petri net has an infinite firing sequence $σ$, then
%    there is a solution $X$ to the constraints $C_T$
%    with $X(t) > 0 \iff t ∈ \inf(σ)$.
%\end{lemma}
%\begin{proof}
%    Let $m_0 \xrightarrow{σ}$ be the infinite firing sequence.
%    A certain set of transitions in $σ$ only fire finitely often,
%    after which the remaining transitions all fire infinitely often.
%    Therefore the sequence can be decomposed into a finite part $σ_0$ and
%    an infinite part $σ'$ such that $σ=σ_0σ'$,
%    $\inf(σ') = \inf(σ)$ and $\fin(σ') = \emptyset$.
%
%    With this we have $m_0 \xrightarrow{σ_0} m_1 \xrightarrow{σ_1}$.
%
%    With this we have
%    $m_0 \xrightarrow{σ_0} m_1 \xrightarrow{σ_1} m_2 \xrightarrow{σ''}$
%    with $σ=σ_0σ_1σ_2σ''$ and $m_2 ≥ m_1$.
%    As $σ_1$ is not empty, we have $X_{σ_1} ≠ 0$ and $X_{σ_1} ≥ 0$.
%    and as $m_2 = m_1 + CX$, we have $CX = m_2 - m_1 ≥ 0$.
%    Therefore $X_{σ_1}$ is a solution to the constraints $C_T$ and
%    $X_{σ_1}(t) > 0 \iff t ∈ \fin(σ_1) \iff t ∈ \inf(σ)$.
%\end{proof}

%\begin{definition}
%For a liveness property $ψ$, the corresponding constraint
%$C_ψ$ is given by
%\begin{align*}
%    C_t           &:= t > 0 \\
%    C_{\neg ψ}    &:= \neg C_ψ \\
%    C_{ψ \land φ} &:= C_ψ \land C_φ \\
%    C_{ψ \lor φ}  &:= C_ψ \lor C_φ
%\end{align*}
%\end{definition}

%\begin{theorem}
%    For a Petri net $N$ and a liveness property $ψ$,
%    If there is no solution to the constraints $C_T \cup \{ C_{¬ψ} \}$,
%    then $N \models ψ$.
%\end{theorem}
%\begin{proof}
%    Assume there is an infinite firing sequence $σ$ such that
%    $σ \models Ψ$.
%\end{proof}
