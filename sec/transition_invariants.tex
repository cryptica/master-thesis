\section{Transition invariants}
\label{sec:transition_invariants}

\begin{definition}[Transition invariant]
A \emph{transition invariant} (also \emph{T-invariant}) of
a Petri net is a firing vector $X$ with
$$
CX = 0.
$$
\end{definition}

\begin{definition}[Quasi transition invariant]
A \emph{semi-positve quasi transition invariant} of
a Petri net is a firing vector $X$ with
\begin{align*}
CX &≥ 0 \\
 X &≥ 0 \\
 X &≠ 0.
\end{align*}
\end{definition}

\begin{lemma}
    If a Petri net has an infinite firing sequence $σ$, then
    there is a solution $X$ to the constraints $C_T$
    with $X(t) > 0 \iff t ∈ \inf(σ)$.
\end{lemma}
\begin{proof}
    Let $m_0 \xrightarrow{σ}$ be the infinite firing sequence.
    A certain set of transitions in $σ$ only fire finitely often,
    after which the remaining transitions all fire infinitely often.
    Therefore the sequence can be decomposed into a finite part $σ_0$ and
    an infinite part $σ'$ such that $σ=σ_0σ'$,
    $\inf(σ') = \inf(σ)$ and $\fin(σ') = \emptyset$.

    With this we have $m_0 \xrightarrow{σ_0} m_1 \xrightarrow{σ_1}$.

    With this we have
    $m_0 \xrightarrow{σ_0} m_1 \xrightarrow{σ_1} m_2 \xrightarrow{σ''}$
    with $σ=σ_0σ_1σ_2σ''$ and $m_2 ≥ m_1$.
    As $σ_1$ is not empty, we have $X_{σ_1} ≠ 0$ and $X_{σ_1} ≥ 0$.
    and as $m_2 = m_1 + CX$, we have $CX = m_2 - m_1 ≥ 0$.
    Therefore $X_{σ_1}$ is a solution to the constraints $C_T$ and
    $X_{σ_1}(t) > 0 \iff t ∈ \fin(σ_1) \iff t ∈ \inf(σ)$.
\end{proof}

\begin{definition}
For a liveness property $ψ$, the corresponding constraint
$C_ψ$ is given by
\begin{align*}
    C_t           &:= t > 0 \\
    C_{\neg ψ}    &:= \neg C_ψ \\
    C_{ψ \land φ} &:= C_ψ \land C_φ \\
    C_{ψ \lor φ}  &:= C_ψ \lor C_φ
\end{align*}
\end{definition}

\begin{theorem}
    For a Petri net $N$ and a liveness property $ψ$,
    If there is no solution to the constraints $C_T \cup \{ C_{¬ψ} \}$,
    then $N \models ψ$.
\end{theorem}
\begin{proof}
    Assume there is an infinite firing sequence $σ$ such that
    $σ \models Ψ$.
\end{proof}
