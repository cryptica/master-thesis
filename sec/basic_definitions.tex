\chapter{Basic definitions}
\label{chap:basic_definitions}

\paragraph{Nets and markings.}
A \emph{Petri net} is a tuple $(P,T,F,m_0)$, where $P$ is a set of \emph{places},
$T$ is a (disjoint) set of \emph{transitions},
$F : (P × T) ∪ (T × P) → \set{0,1}$ is the \emph{flow function}
and $m_0$ is the initial marking.
A \emph{marking} is a function $m : P → \mathbb{N}$,
which describes the number of tokens $m(p)$ in each place $p\in P$.
Assuming an enumeration $p_1, \ldots, p_n$ of $P$, we
often identify $m$ and the vector $(m(p_1), \ldots, m(p_n))$.
For a subset $P' ⊆ P$ of places, we write $m(P') = \sum_{p\in P'} m(p)$.
A \emph{subnet} of a Petri net $(P,T,F,m_0)$ is a tuple $(P',T')$ with
$P' \subseteq P$ and $T' \subseteq T$.
The induced flow relation $F'$ of a subnet is the projection of $F$ onto $(P' × T') ∪ (T' × P')$.

\paragraph{Pre- and post-set.}
For $x\in P\cup T$, the \emph{pre-set} is
$\pre{x} = \set{y ∈ P ∪ T \mid F(y,x) =1}$
and the \emph{post-set} is $\post{x} = \set{y ∈ P ∪ T \mid F(x,y) = 1}$.
We extend the pre- and post-set to a subset of $P ∪ T$ as the union of
the pre- and post-sets of its elements.


\paragraph{Firing transitions.}
A transition $t \in T$ is \emph{enabled at $m$} iff
for all $p \in \pre{t}$, we have $m(p) \ge F(p, t)$.
A transition $t$ enabled at $m$ may \emph{fire},
yielding a new marking $m'$ (denoted $m \xrightarrow{t} m'$),
where $m'(p) = m(p) + F(t,p) - F(p,t)$.
%
A sequence of transitions, $\sigma = t_1 t_2 \ldots t_r$ is an
$\emph{occurrence sequence}$ of $N$ iff there exist markings
$m_1, \ldots, m_r$ such that $m_0 \xrightarrow{t_1} m_1
\xrightarrow{t_2} m_2 \ldots \xrightarrow{t_r} m_r$. The marking
$m_r$ is said to be \emph{reachable} from $m_0$ by the occurrence
of $\sigma$ (denoted $m_0 \xrightarrow{\sigma} m_r$).
An infinite sequence of transitions, $\tau = t_1 t_2 \ldots$
is an $\emph{infinite occurence sequence}$ of $N$ iff
there exist markings $m_1, m_2, \ldots$ such that
$m_0 \xrightarrow{t_1} m_1 \xrightarrow{t_2} m_2 \ldots$.
%every finite prefix $\sigma$ of $\tau$ is an occurence sequence of $N$.
This is denoted by
$m_0 \xrightarrow{\tau}$. For an infinite occurence sequence $\tau$, the set
$\inf(\tau) \subseteq T$ contains the transitions that occur infinitely often in $\tau$.

A \emph{safety property} $\varphi$ is a linear arithmetic constraint over the free variables $P$. 
The property $\varphi$ holds on a marking $m$ iff $m \models \varphi$.
 Examples of safety properties are $s_1 \le 2$, $s_1 + s_2 \ge 1$ and
 $((s_1 \le 1) \land (s_2 \ge 1)) \lor (s_3 \le 1)$.
A Petri net $N$ satisfies a safety property $\varphi$ (denoted by $N \models \varphi$)
iff for all reachable markings $m_0 \xrightarrow{\sigma} m$, we have
$m \models \varphi$.
A property $\varphi$ is an \emph{invariant} of $N$ if it holds for every reachable marking.
A property is inductive if whenever $m\models \varphi$ and $m \xrightarrow{t} m'$ for some $t\in T$
and marking $m'$, we have $m'\models \varphi$.

A \emph{liveness property} $\psi$ is a boolean constraint over the free variables $T$.
An infinite occurence sequence $\tau$ satisfies $\psi$ (denoted by $\tau \models \psi$)
iff, with the interpretation $I$ defined as $I(t) := t \in \inf(\tau)$, we have $I \models \psi$.
A Petri net $N$ satisfies a liveness property $\psi$ (denoted by $N \models \psi$)
iff for all infinite occurence sequences $\tau$ of $N$, we have $\tau \models \psi$.
In other words, $\psi$ specifies which sets of transitions can occur
together infinitely often. For example, the property $\psi = \neg (t_1 \land t_2) \lor t_3$
specifies that there is no sequence where $t_1$ and $t_2$ are fired infinitely often, unless
$t_3$ is also fired infinitely often. Note that the special property $\psi = \false$ is only
satisfied if the net has no infinite firing sequence at all.
This property can be used to specify termination of the net.

% A trap is a set of places $S \subseteq P$ such that $\post{S} \subseteq \pre{S}$.
% If a trap $S$ is marked in $m_0$, i.e. $\sum_{p \in S} m_0(p) > 0$, then it is also marked in all reachable markings.


Petri nets are represented graphically as follows: places and transitions
are represented as circles and boxes, respectively. For $x, y \in P \cup T$,
there is an arc leading from $x$ to $y$ if{}f $F(x,y)=1$.
%
% Petri nets can be used to model the behavior of concurrent programs.
As an example, consider Lamport's 1-bit
algorithm for mutual exclusion~\autocite{Lamport86}, shown in
Fig.~\ref{fig:lamport_src}.
Fig.~\ref{fig:lamport_net} shows a Petri net model for the code.
The two grey blocks model the control flow of the two
processes. For instance, the token in place $p_1$ models the current
position of process 1 at program location $p_1$. The three places in the middle of the
diagram model the current values of the variables. For instance, a
token in place $notbit_1$ indicates that the variable $bit_1$ is currently
set to $\false$.
The mutual exclusion property, which states that the two processes cannot be in the
critical section at the same time, corresponds to the property
$\varphi = (p_3 = 0) \lor (q_5 = 0)$, i.e.\ that
places $p_3$ and $q_5$ cannot both have a token at the same time.

%A liveness property for fairness could be

\newpage

\begin{figure}[t]
  \hrule
  \centering
  \begin{minipage}[t]{.04\columnwidth}
   \mbox{} \\ \\ \\ \\ \\ $p_1$: \\ $p_2$: \\ $p_3$:
  \end{minipage}
  \begin{minipage}[t]{.44\columnwidth}
    \mbox{}\\
    \algProcedure \textsc{Process 1} \\
    \algBegin \\
    \tabT $bit_1$ \algAssgn $\false$ \\
    \tabT \algWhile $\true$\ \algDo \\
    \tabTT $bit_1$ \algAssgn $\true$ \\
    \tabTT \algWhile $bit_2$ \algDo \algSkip \algOd \\
    \tabTT \algComment{critical section} \\
    \tabTT $bit_1$ \algAssgn $\false$ \\
    \tabT \algOd \\
    \algEnd
  \end{minipage}
  \begin{minipage}[t]{.04\columnwidth}
   \mbox{} \\ \\ \\ \\ \\ $q_1$: \\ $q_2$: \\ $q_3$: \\
   $q_4$: \\ \\ \\ $q_5$:
  \end{minipage}
  \begin{minipage}[t]{.44\columnwidth}
    \mbox{}\\
    \algProcedure \textsc{Process 2} \\
    \algBegin \\
    \tabT $bit_2$ \algAssgn $\false$ \\
    \tabT \algWhile $\true$\ \algDo \\
    \tabTT $bit_2$ \algAssgn $\true$ \\
    \tabTT \algIf $bit_1$ \algThen \\
    \tabTTT $bit_2$ \algAssgn $\false$ \\
    \tabTTT \algWhile $bit_1$ \algDo \algSkip \algOd \\
    \tabTTT \algGoto $q_1$ \\
    \tabTT \algFi \\
    \tabTT \algComment{critical section} \\
    \tabTT $bit_2$ \algAssgn $\false$ \\
    \tabT \algOd \\
    \algEnd
  \end{minipage}
  \vspace{1.5ex}
  \hrule
\caption{Lamport's 1-bit algorithm for mutual exclusion~\cite{Lamport86}.}
\label{fig:lamport_src}
\end{figure}

\begin{figure}[t]
  \begin{center}
    \begin{tikzpicture}[
        scale=1.0,
        every node/.style={scale=1.0}
      ]
      \draw[box] (-2,8.5) rectangle (1,0);
      \draw[box] (3,8.5) rectangle (7.5,0);

      \node at (-0.5, 0.5) {First Process};
      \node at (5, 0.5) {Second Process};

      \node[place, label=left:$p_3$] (p3) at (0,7) {};
      \node[transition, label=left:$s_3$] (s3) at (0,6) {};
      \node[place, tokens=1, label=left:$p_1$] (p1) at (0,5) {};
      \node[transition, label=left:$s_1$] (s1) at (0,4) {};
      \node[place, label=left:$p_2$] (p2) at (0,3) {};
      \node[transition, label=left:$s_2$] (s2) at (0,2) {};

      \draw (p3) edge[flow] (s3);
      \draw (s3) edge[flow] (p1);
      \draw (p1) edge[flow] (s1);
      \draw (s1) edge[flow] (p2);
      \draw (p2) edge[flow] (s2);
      \draw[flow, rounded corners=4mm] (s2) |- (-1.5,1) -- (-1.5,8) -| (p3);

      \node[transition, label=above:$t_2$] (t2) at (4,7) {};
      \node[place, label=right:$q_3$] (q3) at (4,6) {};
      \node[transition, label=right:$t_3$] (t3) at (4,5) {};
      \node[place, label=right:$q_4$] (q4) at (4,4) {};
      \node[transition, label=left:$t_4$] (t4) at (4,3) {};

      \node[place, label=right:$q_2$] (q2) at (5.5,7) {};
      \node[transition, label=right:$t_5$] (t5) at (5.5,6) {};
      \node[place, label=right:$q_5$] (q5) at (5.5,5) {};
      \node[transition, label=right:$t_6$] (t6) at (5.5,4) {};
      \node[place, tokens=1, label=right:$q_1$] (q1) at (5.5,3) {};
      \node[transition, label=right:$t_1$] (t1) at (5.5,2) {};

      \draw (q2) edge[flow] (t5);
      \draw (t5) edge[flow] (q5);
      \draw (q5) edge[flow] (t6);
      \draw (t6) edge[flow] (q1);
      \draw (q1) edge[flow] (t1);
      \draw[rounded corners=4mm, flow] (t1) |- (7,1) -- (7,8) -| (q2);

      \draw (q2) edge[flow] (t2);
      \draw (t2) edge[flow] (q3);
      \draw (q3) edge[flow] (t3);
      \draw (t3) edge[flow] (q4);
      \draw (q4) edge[flow] (t4);
      \draw (t4) edge[flow] (q1);

      \node[place, label=above:$bit_1$] (bit1) at (2,6) {};
      \node[place, tokens=1, label=below:$notbit_1$] (notbit1) at (2,4) {};
      \node[place, tokens=1, label=below:$notbit_2$] (notbit2) at (2,2) {};

      \draw (bit1) edge[flow] (s3);
      \draw (s3) edge[flow] (notbit1);
      \draw (notbit1) edge[flow] (s1);
      \draw (s1) edge[flow] (bit1);
      \draw (t2) edge[flow, <->] (bit1);
      \draw (notbit1) edge[flow, <->, bend left=18] (t5);
      \draw (notbit1) edge[flow, <->] (t4);
      \draw (notbit2) edge[flow, <->] (s2);
      \draw (t3) edge[flow] (notbit2);
      \draw (t6) edge[flow, bend left=30] ([xshift=-0.2mm, yshift=1mm] notbit2.east);
      \draw (notbit2) edge[flow] (t1.west);
    \end{tikzpicture}
  \end{center}
\caption{Petri net for Lamport's 1-bit algorithm.}
\label{fig:lamport_net}
\end{figure}


