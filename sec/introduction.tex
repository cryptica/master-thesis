\chapter{Introduction}
\label{chap:introduction}

For program analysis and verification, there are essentially  two different
types of properties, \emph{safety} and \emph{liveness} properties~\cite{Lamport77}.
Safety properties state that nothing bad ever happens, while liveness properties
state that something good eventually happens~\cite{AlpernSchneider85}.
For example, never reaching an error state or a deadlock in a program are
safety properties, while termination or fairness are liveness properties.

Especially for the verification of safety properties of concurrent software, many
papers have proposed and developed techniques 
in the recent years~\cite{KaiserKW10,GantyM12,BouajjaniE14,KaiserKW12,DOsualdoKO13}.
In particular, model checkers based on Petri net coverability have been successfully applied.  
Petri nets are a simple and natural automata-like model for concurrent systems, and 
can model certain programs with an unbounded
number of threads or thread creation. In a nutshell, the places of the
net correspond to program locations, and the number of tokens in a place
models the number of threads that are currently at that location. This
point was first observed in~\cite{GermanS92}, and later revisited in
\cite{DelzannoRB02} and, more implicitly, in~\cite{KaiserKW10,GantyM12}.

The safety problem, i.e.\ whether at least one thread can reach a given program
location (modelling some kind of error), naturally reduces to the
\emph{coverability problem} for Petri nets: given a net $N$ and a marking
$M$, decide whether some reachable marking of $N$ \emph{covers} $M$,
i.e., puts at least as many tokens as $M$ on each place. While the
decidability and EXPSPACE-completeness of the coverability problem
were settled long ago~\cite{KarpM69,Rackoff78}, new algorithmic ideas
have been developed in recent years
\cite{GeeraertsRB06,GantyRB08,ValmariH14,KaiserKW12,KloosMNP13}.
The techniques are based on forward or backward state-space 
exploration, which is accelerated in a number of ways in order to cope 
with the possibly infinite number of states.

Here we will revisit an approach to the coverability problem based
on classical Petri net analysis techniques: the marking equation and
traps~\cite{Murata89,Reisig13}.
The marking equation is a system of linear constraints that can be
easily derived from the net, and whose set of solutions
overapproximates the set of reachable markings. This system
can be supplemented with linear constraints specifying a set of unsafe
markings, and solved using standard linear or integer programming. If
the constraints are infeasible, then all reachable markings are safe. If
not, then one can try different approaches. In~\cite{EsparzaMelzer00}
a solution of the constraints is used to derive an additional constraint 
in the shape of a \emph{trap}: a set of places that, loosely speaking, once 
marked cannot be ``emptied''; the process can be iterated.
More recently,
in~\cite{WimmelWolf12}, Wimmel and Wolf propose to use the solution to guide a
state space exploration searching for an unsafe marking; if the search fails, 
then information gathered during it is used to construct an additional
constraint.
Our methods and results for coverability were shown
in~\cite{EsparzaLMMN14} and will be used in this thesis.
Additionally, we will test some other safety properties besides coverability.

The liveness problem instead asks whether a program always reaches some good state.
For that, one one needs to look at infinite executions.
A traditional approach is the automata-theoretic approach~\cite{VardiWolper86}
using LTL and Büchi automata.
This has been applied to 1-safe Petri nets by Esparza and Melzer~\cite{EsparzaMelzer97}
in the form of \emph{Büchi nets}.
The problem is reduced to deciding emptiness of the Büchi net, i.e.\ checking if the net
has an infinite accepting run. For 1-safe nets, this was shown to be PSPACE-complete.
However, similar to the marking equation, the infinite runs can be
overapproximated by \emph{semi-positive T-invariants} to check liveness~\cite{EsparzaBruns96}.
In~\cite{EsparzaMelzer97} this has been refined to \emph{$T^\star$-invariants} by using
\emph{S-components}~\cite{DeselEsparza95}.
We will use the constraint-based approach described there and
and apply it to general Petri nets.

%Safety The safety property basically states that bad things never happens. The
%definition of a ”bad thing” varies depending on the system but in general it holds
%that a system violates the safety property if it enters an illegal state. Formally, for
%a system S where the state α represents deadlock. Safety with respect to α is the
%assertion that α is false for all states reachable from an initial state S 0.

%Liveness The liveness property states that ”something good eventually happens”
%which means that the system, at some, point will reach a desired state. Since the
%property works with infinite time as displayed by the use of ”eventually”, there is
%no way of disproving a system. Formally the liveness property can be summarized
%with the following. For a system S where β is a desirable state, β is true for some
%state reachable from S 0. Therefore the liveness property is often split into parts that
%can be proved separately. If true the system should enter a desired state at some
%point of the execution. Common parts that will be necessary for our further proof
%is stated below.
%• Fairness: This property states that every process should be executed infinitely
%often.
%• Termination: The property states that from an initial state every behavior of
%the system leads to a configuration where all the guards (all execution paths
%is locked in there final state) are false.

Both the safety and liveness analysis can be done with
constraint-based techniques. While these are known for a while, they have always
suffered from the absence of efficient decision procedures for 
linear arithmetic together with Boolean satisfiability.  
Profiting from recent advances in SMT-solving
technology, we reimplement the techniques of~\cite{EsparzaMelzer00}
and~\cite{EsparzaMelzer97}
on top of the Z3 SMT solver~\cite{MouraB08}, and apply it to a large 
collection of benchmarks.

The techniques are theoretically incomplete, i.e., the set of linear 
constraints derived from the marking equation and traps may be 
feasible even if all reachable markings are safe.
Also the T-invariant-based liveness technique with S-components may not prove all
properties.
Surprisingly we find that, despite this fact, the safety technique
is powerful enough to prove safety of 96 out of a total
of 115 safe benchmarks gathered from current research papers in concurrent software verification. 
In contrast, three different state-of-the-art tools for coverability
proved only 61, 51, or 33 of these 115 cases!
Moreover, and possibly due to the characteristics of the
application domain, even the simplest version of the
technique—based on the marking equation—is successful in 84 cases.
With the liveness technique, we can prove termination in
1834 of 1836 cases on two sets of industrial business process models.
We also perform well on two scalable typical examples. 

As an additional contribution, and inspired
by work on interpolation, we show that a dual version of the classical
set of constraints, equivalent in expressive power, can be used not
only to check safety, but to produce an inductive invariant.
While some existing solvers based on state-space
exploration can also produce such invariants, we show that inductive invariants obtained
through our technique are usually orders
of magnitude smaller.  Additionally, while we can use the 
SMT solver iteratively to minimize the invariant, the tool almost always
provides a minimal one at the first attempt.

% The paper is organized as follows.
% We introduce the basic notation in Section~\ref{sec_preliminaries}.
% We recall the marking equation and traps approach in
% Sections~\ref{sec_method_safety}
% and~\ref{sec_method_safety_by_refinement}.
% We introduce three novel methods to constructing invariants in
% Sections~\ref{sec_method_invariant}
% % \ref{sec_method_invariant_by_refinement},
% and
% \ref{sec_method_invariant_w_minimization}.
% Section~\ref{sec_experiments} presents (1) our experimental evaluation of
% the application of SMT solver \zthree\ to the constraint-based
% approach, and (2) our experimental evaluation of our methods for
% invariant construction.
% Section~\ref{sec_rel_work} reviews related
% work. Section~\ref{sec_conclusions} presents our conclusions.
