\section{The marking equation}
\label{sec:the_marking_equation}

We now recall a well-known method, which we call \safety, that provides a sufficient
condition for a given Petri net $N$ to satisfy a property $\varphi$ 
by reducing the problem to checking satisfiability of a linear arithmetic formula.
We illustrate the method on Lamport's 1-bit algorithm for mutual
exclusion. 

Before going into details, we state several conventions. For a
Petri net $N = (P, T, F, m_0)$, we introduce a vector of
$|P|$ variables $M$, and a vector of $|T|$ variables $X$.
The vectors $M$ and $X$ will be used to represent the current
marking and the number of occurrences
of transitions in the occurrence sequence leading to the current marking, respectively.
If a place or a transition is given a specific
name, we use the same name for its associated variable. Given a place
$p$, the intended meaning of a constraint like $p \geq 3$ is 
``at the current marking place $p$ must have at least 3 tokens.'' 
Given a transition $t$, the intended meaning of a constraint like $t \leq 2$ is ``in
the occurrence sequence leading to the current marking, transition $t$
must fire at most twice.'' 

The key idea of the \safety{} method lies in the \emph{marking
  equation}:
$$M = m_0 + C X\,,$$
where the incidence matrix $C$ is a $|P| \times |T|$ matrix given by
$$C(p,t) = F(t,p) - F(p,t)\,.$$
For each place $p$, the marking equation contains a constraint
that formulates a simple token conservation law: the number of tokens in
$p$ at the current marking is equal to the initial number of tokens $m_0(p)$,
plus the number of tokens added by the input transitions of $p$, minus
the number of tokens removed by the output transitions. 
So, for instance, in Lamport's algorithm the constraint for place $notbit_1$
is:
$$notbit_1 = 1 + s_3 + t_5 + t_4 - s_1 - t_4 - t_5 = 1 +s_3 -s_1\,.$$
We equip the marking equation with the non-negativity conditions,
modeling that the number of tokens in a place, or the number of
occurrences of a transition in an occurrence sequence cannot become
negative.

All together, we get the following set of \emph{marking constraints}:
\begin{align*}
  \mathcal{C}(P, T, F, m_0) ::&
  \begin{cases}
    M = m_0 + C X  & \text{marking equation} \\
    M \ge 0        & \text{non-negativity conditions for places} \\
    X \ge 0        & \text{non-negativity conditions for transitions} \\
  \end{cases}
\end{align*}

Method \safety{} for checking that a property $\varphi$ is
invariant for a Petri net $N=(P,T,F,m_0)$ consists of checking for
satisfiability of the constraints
\begin{equation}\label{eq:safety}
\mathcal{C}(P,T,F,m_0) \wedge \lnot \varphi(M)\,.
\end{equation}
If the constraints are unsatisfiable, then no reachable marking
violates $\varphi$. To see that this is true, consider the converse:
If there exists an occurrence sequence $m_0
\xrightarrow{\sigma} m$ leading to a marking $m$ that violates the
property, then we can construct a valuation of the variables that assigns $m(p)$ to
$M(p)$ for each place $p$, and the number of occurrences of $t$ in
$\sigma$ to $X(t)$ for each transition $t$. This valuation then
satisfies the constraints.

The method does not work in the other direction: If the constraints
\eqref{eq:safety} are satisfiable, we cannot conclude that the
property $\varphi$ is violated.

\begin{figure}
  \begin{minipage}{.79\columnwidth}
    \emph{Place equations} \\ \\[-2em]
    \begin{alignat*}{22}
         p_1   &{}={}& 1 &{}-{}& s1 &     &    &{}+{}& s3 &&&&&&&&&&&&& \\[-0.4em]
         p_2   &{}={}& 0 &{}+{}& s1 &{}-{}& s2 &     &    &&&&&&&&&&&&& \\[-0.4em]
         p_3   &{}={}& 0 &     &    &{}+{}& s2 &{}-{}& s3 &&&&&&&&&&&&& \\[-0.4em]
         bit_1 &{}={}& 0 &{}+{}& s1 &     &    &{}-{}& s3 &&&&&&&&&&&&& \\[-0.4em]
    notbit_1 &{}={}& 1 &{}-{}& s1 &     &    &{}+{}& s3 &&&&&&&&&&&&& \\[-0.4em]
         q_1   &{}={}& 1 &&{}-{}& t1 &     &    &     &    &{}+{}& t4 &     &    &{}+{}& t6 & \\[-0.4em]
         q_2   &{}={}& 0 &&&&&&&{}+{}& t1 &{}-{}& t2 &     &    &     &    &{}-{}& t5 &     &    & \\[-0.4em]
         q_3   &{}={}& 0 &&&&&&&     &    &{}+{}& t2 &{}-{}& t3 &     &    &     &    &     &    & \\[-0.4em]
         q_4   &{}={}& 0 &&&&&&&     &    &     &    &{}+{}& t3 &{}-{}& t4 &     &    &     &    & \\[-0.4em]
         q_5   &{}={}& 0 &&&&&&&     &    &     &    &     &    &     &    &{}+{}& t5 &{}-{}& t6 & \\[-0.4em]
    notbit_2 &{}={}& 1 &&&&&&&{}-{}& t1 &     &    &{}+{}& t3 &     &    &     &    &{}+{}& t6 &
    \end{alignat*}
  \end{minipage}
  \begin{minipage}{.19\columnwidth}
    \emph{Non-negativity conditions} \\[-2em]
    \begin{align*}
         p_1   & \ge 0 \\[-0.4em]
         p_2   & \ge 0 \\[-0.4em]
         p_3   & \ge 0 \\[-0.4em]
         bit_1 & \ge 0 \\[-0.4em]
      notbit_1 & \ge 0 \\[-0.4em]
         q_1   & \ge 0 \\[-0.4em]
         q_2   & \ge 0 \\[-0.4em]
         q_3   & \ge 0 \\[-0.4em]
         q_4   & \ge 0 \\[-0.4em]
         q_5   & \ge 0 \\[-0.4em]
      notbit_2 & \ge 0
    \end{align*}
  \end{minipage}
\caption{Constraints $C$ given by function \safety\ for
         Lamport's 1-bit algorithm}
\label{fig:setC}
\end{figure}


As an example, consider the Lamport's algorithm.
We can successfully prove the property ``if process
1 is at location $p_3$, then ${bit_1}={\true}$''
with method \safety{} by adding the negation $p_3 \geq 1 \wedge bit_1 \neq 1$
to the constraints $\mathcal{C}(P,T,F,m_0)$ given in Figure~\ref{fig:setC},
which makes them unsatisfiable.
However, if we apply it to the mutual
exclusion property, i.e.,~check for satisfiability of
$\mathcal{C}(P,T,F,m_0) \wedge p_3 \geq 1\wedge q_5 \geq 1$, we obtain
a solution, but we cannot conclude that the mutual exclusion property
does not hold.

Note that the marking constraints~\eqref{eq:safety} are interpreted over integer
variables.
As usual in program analysis, one can solve the constraints over rationals to get an approximation
of the method. Solving the constraints over rationals will become
useful in Section~\ref{sec:constructing_invariants}.

\iffalse

\begin{verbatim}
* Method Safety:

  Subprocedure \mathcal{C} constructs state constraints C corresponding to N.
\end{verbatim}

\begin{figure}[t]
  \hrule
  \centering
  \begin{minipage}[t]{.04\columnwidth}
   \mbox{} \\ \\ \\ \\ \\ \\ \\ \\
   \raggedleft 1 \\ 2 \\ 3 \\ 4 \\ 5 \\ 6 %\\ 7 \\ 8 \\ 9 \\ 10 \\
%   11 \\ 12 \\ 13 \\ 14 \\ 15 \\ 16 \\ 17 \\ 18 \\ 19 % \\ 20 \\
%   21 \\ 22 \\ 23 \\ 24 \\ 25 \\ 26 \\ 27 \\ 28 \\ 29 % \\ 30 \\
%   31 \\ 32 \\ 33 \\ 34 \\ 35 \\ 36
  \end{minipage}
  \begin{minipage}[t]{.94\columnwidth}
    \mbox{}\\
    \algMethod \textsc{Safety} \\
    \algInput\\
    \tabT $N$ — Petri net \\
    \tabT $P$ — property \\
    \algVars\\
    \tabT $C$ — state constraints \\
    \algBegin\\
    \tabT $C$ \algAssgn $\mathcal C(N)$ \\
    \tabT \algIf SAT$(C \cup \{\neg P\})$ \algThen \\
    \tabTT \algReturn ``The petri net may not satisfy the property'' \\
    \tabT \algElse \\
    \tabTT \algReturn ``The petri net satisfies the property'' \\
    \tabT \algFi \\
    \algEnd
  \end{minipage}
  \vspace{1.5ex}
  \hrule
\caption{Method \safety}
\label{fig_method_safety}
\end{figure}


\begin{verbatim}
* Property of state constraints C: If C U {\neg P} is unsat then N |= P.

* Place equation:
  
  For a given place s the place equation is

  # of tokens in s = initial number of tokens of place s
                     + # times each input transition of s fires
                     - # times each output transition of s fires

* Non-negativity conditions:

  # of tokens in place s           >= 0
  # of times transition t is fired >= 0

* Subprocedure \mathcal{C}:

  Input:
    N = (S, T, F, M0) : Petri net
  Output:
    C : State constraints

  Pseudocode:

\end{verbatim}

\begin{align*}
  \mathcal{C}(N) =& \left( \bigwedge_{s \in S} \left(
    s = M_0(s) + \sum_{t \in \pre{s}} t - \sum_{t \in \post{s}} t
  \right) \right) \land
    \left( \bigwedge_{s \in S} s \ge 0 \right) \land
    \left( \bigwedge_{t \in T} t \ge 0 \right)
\end{align*}

\newpage

\begin{verbatim}
* Example

  - Property 1: Process 1 and Process 2 are never in their respective critical section at the same time.

  - Property 1: p3 < 1 \/ q5 < 1

  - Petri net:

\end{verbatim}

% \input{fig_lamport_net}

\newpage

\begin{verbatim}
  
\end{verbatim}

%\iffalse
\begin{verbatim}
  - State constraints C:
    p1    = 1 - s1      + s3
    ^       ^   ^         ^
    |       |   |         |
    |       |   |         |
    |       |   |         |
    |       |   |        # of tokens given to p1
    |       |   |    
    |       |  # of tokens taken from p1
    |       |
    |      initial number of tokens in p1
    |
    number of tokens in p1
\end{verbatim}

\begin{verbatim}
- State constraints C:
\end{verbatim}

\begin{figure}
  \begin{minipage}{.79\columnwidth}
    \emph{Place equations} \\ \\[-2em]
    \begin{alignat*}{22}
         p_1   &{}={}& 1 &{}-{}& s1 &     &    &{}+{}& s3 &&&&&&&&&&&&& \\[-0.4em]
         p_2   &{}={}& 0 &{}+{}& s1 &{}-{}& s2 &     &    &&&&&&&&&&&&& \\[-0.4em]
         p_3   &{}={}& 0 &     &    &{}+{}& s2 &{}-{}& s3 &&&&&&&&&&&&& \\[-0.4em]
         bit_1 &{}={}& 0 &{}+{}& s1 &     &    &{}-{}& s3 &&&&&&&&&&&&& \\[-0.4em]
    notbit_1 &{}={}& 1 &{}-{}& s1 &     &    &{}+{}& s3 &&&&&&&&&&&&& \\[-0.4em]
         q_1   &{}={}& 1 &&{}-{}& t1 &     &    &     &    &{}+{}& t4 &     &    &{}+{}& t6 & \\[-0.4em]
         q_2   &{}={}& 0 &&&&&&&{}+{}& t1 &{}-{}& t2 &     &    &     &    &{}-{}& t5 &     &    & \\[-0.4em]
         q_3   &{}={}& 0 &&&&&&&     &    &{}+{}& t2 &{}-{}& t3 &     &    &     &    &     &    & \\[-0.4em]
         q_4   &{}={}& 0 &&&&&&&     &    &     &    &{}+{}& t3 &{}-{}& t4 &     &    &     &    & \\[-0.4em]
         q_5   &{}={}& 0 &&&&&&&     &    &     &    &     &    &     &    &{}+{}& t5 &{}-{}& t6 & \\[-0.4em]
    notbit_2 &{}={}& 1 &&&&&&&{}-{}& t1 &     &    &{}+{}& t3 &     &    &     &    &{}+{}& t6 &
    \end{alignat*}
  \end{minipage}
  \begin{minipage}{.19\columnwidth}
    \emph{Non-negativity conditions} \\[-2em]
    \begin{align*}
         p_1   & \ge 0 \\[-0.4em]
         p_2   & \ge 0 \\[-0.4em]
         p_3   & \ge 0 \\[-0.4em]
         bit_1 & \ge 0 \\[-0.4em]
      notbit_1 & \ge 0 \\[-0.4em]
         q_1   & \ge 0 \\[-0.4em]
         q_2   & \ge 0 \\[-0.4em]
         q_3   & \ge 0 \\[-0.4em]
         q_4   & \ge 0 \\[-0.4em]
         q_5   & \ge 0 \\[-0.4em]
      notbit_2 & \ge 0
    \end{align*}
  \end{minipage}
\caption{Constraints $C$ given by function \safety\ for
         Lamport's 1-bit algorithm}
\label{fig:setC}
\end{figure}


\begin{verbatim}
  - Using property 1
  
  - Negated property 1 \neg P:
\end{verbatim}

\begin{align*}
     p3 \ge 1 & \land q5 \ge 1
\end{align*}

\fi
